%===============================================================================
% LaTeX sjabloon voor de bachelorproef toegepaste informatica aan HOGENT
% Meer info op https://github.com/HoGentTIN/bachproef-latex-sjabloon
%===============================================================================
\documentclass[table,xcdraw]{bachproef-tin}
\usepackage{hogent-thesis-titlepage}

 % Glossary en fix voor glossary met HoGent Theme
\usepackage[nonumberlist]{glossaries}
\renewcommand{\glossarysection}[2][]{}
\setcounter{secnumdepth}{5}
\makeglossaries
%run generate.sh en run hier
\newglossaryentry{datacompressie}
{
	name={datacompressie},
	description={Datacompressie bestaat uit het digitaal opslaan van een bestand met zo weinig mogelijk bits. Enkele belangrijke factoren voor het bepalen van de juiste datacompressie zijn gewenste kwaliteit, bestandsgrootte en snelheid}
}

\newglossaryentry{bit}
{
	name={bit},
	description={Zoals de naam bit, kort voor binary digit, suggereert kan een bit beschouwd worden als een binair signaal. Een bit wordt beschouwd als de kleinste eenheid voor dataopslag. Meer informatie rond bestandsgrootte en dataopslag worden besproken in hoofdstuk \ref{ch:literatuurstudie}, deel \ref{sec:bestandsgrootte-dataopslag}},
	plural={bits}
}

\newglossaryentry{dna-compressie}
{
	name={DNA compressie},
	description={Het menselijke DNA kan digitaal voorgesteld worden door een lange lijst van 5 verschillende karakters, gekend als basen. Deze digitale voorstelling bestaat uit meer dan 3 miljard van deze basen (\cite{dodanaugent2011}). DNA compressie bestaat er uit deze reeks van basen zo efficiënt mogelijk op te slaan zodanig dat performante bewerkingen mogelijk zijn met een zo klein mogelijke bestandsgrootte}
}

\newglossaryentry{afbeeldingscompressie}
{
	name={afbeeldingscompressie},
	description={Afbeeldingscompressie bestaat er uit een afbeelding in zo weinig mogelijk aantal bits op te slaan terwijl een aanvaardbare kwaliteit behouden blijft. Dit kan zowel via lossless als lossy algoritmes. Afbeeldingscompressie wordt uitgebreid besproken in hoofdstuk \ref{ch:afbeeldingscompressie}}
}

\newglossaryentry{videocompressie}
{
	name={videocompressie},
	description={Videocompressie bestaat er uit een videobestand in zo weinig mogelijk aantal bits op te slaan terwijl een accapteerbare kwaliteit behouden blijft. Dit kan zowel via lossless als lossy algoritmes. videcompressie wordt uitgebreid besproken in hoofdstuk \ref{ch:videocompressie}}
}

\newglossaryentry{codec}
{
	name={codec},
	description={De coder-decoder. Binnen datacompressie betekent codec de gebruikte techniek om een bestand te comprimeren. De codec is dus is de technologie verantwoordelijk voor het encoden en decoden van een bestand volgens een bapaald compressiealgoritme. BV; H.265},
	plural={codecs}
}

\newglossaryentry{afbeeldingsformaat}
{
	name={afbeeldingsformaat},
	description={Een afbeeldingsformaat bevat alle gegevens voor het digitaal opslaan van een afbeelding. De vier gekendste categoriën van afbeeldingsformaten zijn: raster, vector, compound en stereo formaten},
	plural={afbeeldingsformaten}
}

\newglossaryentry{container}
{
	name={container},
	description={Binnen datacompressie kan een container vrijwel letterlijk vertaald worden. Het is een verpakking voor alle data die men opslaat en metadata. onder andere de codec plaatst data in deze container. Wanneer er gesproken wordt over bestandsextensies wordt vaak de container bedoeld. Bv; MP4}
}

\newglossaryentry{jpeg}
{
	name={JPEG},
	description={Ook gekend als JPG. JPEG is een afkorting voor Joint Photographic Experts Group. JPEG is een bestandsformaat voor het opslaan van digitale afbeeldingen via lossy compressie. Afbeeldingscompressie en JPEG worden uitgebreid besproken in hoofdstuk \ref{ch:afbeeldingscompressie}}
}

\newglossaryentry{jpeg2000}
{
	name={JPEG2000},
	description={Ook gekend als JPEG2K. JPEG2000 is een bestandsformaat voor het opslaan van digitale afbeeldingen gemaakt als opvolger van JPEG. Net zoals JPEG maakt het gebruik van lossy compressie. Afbeeldingscompressie en JPEG2000 worden uitgebreid besproken in hoofdstuk \ref{ch:afbeeldingscompressie}}
}

\newglossaryentry{png}
{
	name={PNG},
	description={PNG is een afkorting voor Portable Network Graphics. PNG is een bestandsformaat voor het opslaan van digitale afbeeldingen. PNG maakt gebruik van lossless compressie. Afbeeldingscompressie en PNG worden uitgebreid besproken in hoofdstuk \ref{ch:afbeeldingscompressie}}
}

\newglossaryentry{h264-avc}
{
	name={H.264-AVC},
	description={H.264-AVC is één van de gekenste videocodecs die grootschalig gebruikt wordt. AVC is een afkorting van Advanced Video Coding. H.264-AVC wordt uitgebreid besproken in deel \ref{sec:videocompressie-h264-AVC}}
}

\newglossaryentry{h264-svc}
{
	name={H.264-SVC},
	description={H.264-SVC is een videocodec ontwikkelt als extensie op H264-AVC. SVC is een afkorting voor Scalable Video Coding. De nadruk bij deze extensie ligt zoals de naam sugereert op schaalbaarheid. H.264-SVC wordt uitgebreid besproken in deel \ref{sec:videocompressie-h264-SVC}}
}

\newglossaryentry{h265}
{
	name={H.265/HEVC},
	description={H.265/HEVC is een videocodec ontwikkelt als opvolger van H.264/AVC. H.265/HEVC wordt uitgebreid besproken in deel \ref{sec:videocompressie-h265}}
}

\newglossaryentry{av1}
{
	name={AV1},
	description={AV1 is een videocodec ontwikkelt als open standaard. AV1 is een afkorting voor AOMedia Video. AV1 is vooral interessant omdat het royalty free is en dus geen licentiekosten heeft. AV1  wordt uitgebreid besproken in deel \ref{sec:videocompressie-av1}}
}

\newglossaryentry{open-source}
{
	name={open source},
	description={Als een programmeer project open source is wilt dit zeggen dat de broncode raadpleegbaar is. Dit wil echter niet gegarandeerd zeggen dat het software programma gratis is in gebruik of de code zomaar aangepast mag worden. Dit hangt af van de licentie}
}

\newglossaryentry{lossless}
{
	name={lossless},
	description={Binnen datacompressie slaat lossless compressie op het comprimeren van een bestand zonder kwaliteitsverlies. In het geval van video's en afbeeldingen wilt dit zeggen dat een bestand gecomprimeerd met een lossless algoritme identiek is aan het origineel}
}

\newglossaryentry{lossy}
{
	name={lossy},
	description={Binnen datacompressie slaat lossy compressie op het comprimeren van een bestand met kwaliteitsverlies voor het besparen van data. In het geval van video's en afbeeldingen wilt dit zeggen dat een bestand gecomprimeerd met een lossy algoritme een significante hoeveelheid aan data en kwaliteit kan verliezen in vergelijking met het originele bestand}
}

\newglossaryentry{afbeeldingsevaluatietool}
{
	name={afbeeldingsevaluatietool},
	description={Een applicatie gemaakt voor het voeren van een objectief of subjectief onderzoek naar afbeeldingskwaliteit. De mogelijke soorten tools worden verder besproken in hoofdstuk \ref{ch:kwaliteit}. Een subjectieve afbeeldingsevaluatietool werd gebouwd voor deze paper en wordt verder besproken in hoofdstuk \ref{ch:onderzoek}}
}

\newglossaryentry{binaire-voorstelling-bestandsgrootte}
{
	name={Binaire voorstelling},
	description={De binaire voorstelling voor bestandsgroottes maakt gebruik van 1024 als basis wat overeen komt met $ 2^{10} $. Dit wordt verder besproken in deel \ref{sec:bestandsgrootte-dataopslag-voorvoegsels-binair}}
}

\newglossaryentry{si-voorstelling-bestandsgrootte}
{
	name={SI voorstelling},
	description={De SI voorstelling voor bestandsgroottes maakt gebruik van 1000 als basis wat overeen komt met $ 10^{3} $. Dit wordt verder besproken in deel \ref{sec:bestandsgrootte-dataopslag-voorvoegsels-si}}
}

\newglossaryentry{ieee}
{
	name={IEEE},
	description={Institute of Electrical and Electronics Engineers. Het is een non-profit instituut dat zich inzet voor technologische vooruitgang}
}

\newglossaryentry{clustergrootte}
{
	name={clustergrootte},
	description={De grootte van een cluster. Dit is verder besproken in deel \ref{sec:bestandsgrootte-dataopslag-clustergrootte}}
}

\newglossaryentry{cluster}
{
	name={cluster},
	description={Het kleinste deel van een oplsagmedium waar data in voorzien kan worden. Dit is verder besproken in deel \ref{sec:bestandsgrootte-dataopslag-clustergrootte}},
	plural={clusters}
}

\newglossaryentry{byte}
{
	name={byte},
	description={Een term voor acht bits},
	plural={bytes}
}

\newglossaryentry{leestijd}
{
	name={leestijd},
	description={De tijd die verstrekt tussen het aanvragen van een bestand op een opslagmedium tot het effectief verkrijgen van dat bestand},
	plural={leestijden}
}

\newglossaryentry{bandbreedte}
{
	name={bandbreedte},
	description={Een term waarmee de maximale beschikbare hoeveelheid data dat over een netwerkverbinding verstuurd kan worden bedoeld wordt}
}

\newglossaryentry{prefix-code}
{
	name={prefix code},
	description={Een korte waarde die een langere waarde voorstelt binnen prefix coding. Dit is verder beschreven in deel \ref{sec:ontstaan-datacompressie-primitieve-technieken-binnen-it}},
	plural={prefix codes}
}

\newglossaryentry{huffman-coding}
{
	name={Huffman coding},
	description={Een compressie-algoritme gebasseerd op prefix code. Huffman coding wordt verder toegelicht aan de hand van een voorbeeld in deel \ref{sec:primitieve-technieken-voorbeeld-huffman-encoding}. Een implementatie wordt voorzien in hoofdstuk \ref{ch:compressietool}}
}

\newglossaryentry{compressie-algoritme}
{
	name={compressie-algoritme},
	description={Een set instructies dat het mogelijk maakt om bepaalde informatie met minder resources voor te stellen. Binnen deze bachelorproef wilt dit code voorstellen die een een bestand zijn bestandsgrootte kan verkleinen. Dit kan zowel lossless als lossy},
	plural={compressie-algoritmen}
}

\newglossaryentry{lookup-table}
{
	name={lookup table},
	description={Een tabel waar waardes in opgezocht kunnen worden. Bij prefix code is dit de tabel waar de lange waarde staat die de korte prefix code voorstelt}
}

\newglossaryentry{prefix-coding}
{
	name={prefix coding},
	description={Een vorm van datacompressie waar bij een waarde wordt toegekend aan iets dat verwijst naar een andere, langere waarde. Dit is verder beschreven in deel \ref{sec:ontstaan-datacompressie-primitieve-technieken-binnen-it}}
}

\newglossaryentry{use-case}
{
	name={use case},
	description={Representeert een bepaald doel en de manier waarop dat doel bereken wenst te worden},
	plural={use cases}
}

\newglossaryentry{rle-long}
{
	name={run length encoding},
	description={Een compressietechniek die verder besproken wordt in deel \ref{sec:primitieve-technieken-voorbeeld-rle} en geïmplementeerd wordt in deel \ref{ch:compressietool}}
}

\newglossaryentry{rle-short}
{
	name={RLE},
	description={Kort voor run length encoding}
}

\newglossaryentry{ascii}
{
	name={ASCII},
	description={Een tekenset die voorgesteld wordt door acht bits}
}

\newglossaryentry{encoding}
{
	name={encoding},
	description={Het proces dat een encoder uitvoert}
}

\newglossaryentry{decoding}
{
	name={decoding},
	description={het process dat een decoder uitvoert}
}

\newglossaryentry{meta-data}
{
	name={metadata},
	description={Data over data}
}

\newglossaryentry{webp}
{
	name={WebP},
	description={Een afbeeldingsformaat dat uitgebreid besproken wordt in deel \ref{sec:afbeeldingscompressie-webp}}
}

\newglossaryentry{heif}
{
	name={HEIF},
	description={Een codec dat uitgebreid besproken wordt in deel \ref{sec:afbeeldingscompressie-heif}}
}

\newglossaryentry{ps}
{
	name={Adobe Photoshop},
	description={Computersoftware voor het aanmaken en bewerken van rasterafbeeldingen}
}

\newglossaryentry{css}
{
	name={CSS},
	description={Programmeertaal die verantwoordelijk is voor het opmaken van een webpagina}
}

\newglossaryentry{minifyen}
{
	name={minifyen},
	description={Een proces waarbij code zodanig herschreven wordt dat dit de minste hoveelheid ruimte in beslag neemt}
}

\newglossaryentry{render}
{
	name={render},
	description={het genereren van een digitale afbeelding of video met behulp van computersoftware}
}

\newglossaryentry{raw}
{
	name={RAW},
	description={Een benaming voor afbeeldingsformaten dat in geen enkele vorm gecomprimeerd worden. RAW wordt uitgebreid besproken in deel \ref{sec:afbeeldingscompressie-raw}}
}

\newglossaryentry{compressietool}
{
	name={datacompressietool},
	description={Een kleine applicatie voor het comprimeren van data. In hoofdstuk \ref{ch:compressietool} wordt een proof of concept compressietool geïmplementeerd}
}

\newglossaryentry{github}
{
	name={GitHub},
	description={Een online platform voor versiebeheer van code}
}

\newglossaryentry{hosting}
{
	name={hosting},
	description={Het delen van een bepaalde webpagina met anderen}
}

\newglossaryentry{php}
{
	name={PHP},
	description={Een programmeertaal die voornamelijk de logica bij websites voorziet}
}

\newglossaryentry{sql}
{
	name={SQL},
	description={Een standaard programmertaal voor communicatie met een relationele databank}
}

\newglossaryentry{xampp}
{
	name={XAMPP},
	description={Een software pakket waarmee eenvoudig een hosting omgeving kan opgezet worden}
}

\newglossaryentry{jquery}
{
	name={JQuery},
	description={Een uitbreiding op JavaScript}
}

\newglossaryentry{drift}
{
	name={Drift},
	description={Een JavaScript library dat het eenvoudig maakt om in te zoomen op afbeeldingen zonder manipulaties}
}

\newglossaryentry{bootstrap}
{
	name={Bootstrap},
	description={Een toolkit voor het eenvoudig programmeren met HTML, CSS, JS en jQuery}
}

\newglossaryentry{js}
{
	name={JavaScript},
	description={Een programmeertaal waarmee onder andere de inhoud van een webpagina gemanipuleerd kan worden}
}

\newglossaryentry{yt}
{
	name={YouTube},
	description={Een online website waar gebruikers gratis video's kunnen uploaden. YouTube is een dochterbedrijf van Google en is één van de grootste streaming diensten}
}

\newglossaryentry{plug-in}
{
	name={plug-in},
	description={Een downloadbare collectie code die bepaalde functionaliteit voorziet},
	plural={plug-ins}
}

\newglossaryentry{artefact}
{
	name={artefact},
	description={Een kunstmatig verschijnsel. Binnen datacompressie zijn artefacten zichtbare of hoorbare fouten},
	plural={artefacten}
}

\newglossaryentry{wavelet}
{
	name={wavelet},
	description={Een golfvormige soort data},
	plural={wavelets}
}

\newglossaryentry{ai}
{
	name={artificiële intelligentie},
	description={Wanneer een apparaat kan reageren op binnenkomende data en op basis daarvan een eigen beslissing kan maken zonder dit expliciet geprogrammeerd is spreek men van artificiële intelligentie}
}

\newglossaryentry{nef}
{
	name={NEF},
	description={Het RAW afbeeldingsformaat van Nikon}
}

\newglossaryentry{dng}
{
	name={DNG},
	description={Digital Negative Specification is een RAW formaat dat Adobe heeft uitgevonden met de bedoelding dat dit wereldwijd geadopteerd zou worden als standaard voor RAW afbeeldingsformaten}
}

\newglossaryentry{w3c}
{
	name={W3C},
	description={World Wide Web Consortium is een organisatie die standaarden voor het web voorziet}
}

\newglossaryentry{gif}
{
	name={GIF},
	description={Graphics interchange format is een afbeeldingsformaat dat op heden voornamelijk gebruikt wordt om bewegende delen voor te stellen}
}

\newglossaryentry{rgb}
{
	name={RGB},
	description={Red Green Blue is een kleurcodering systeem voor het voorstellen van kleuren aan de hand van deze drie basiskleuren}
}

\newglossaryentry{cmyk}
{
	name={CMYK},
	description={Cyan, Magenta, Yellow, Key is een kleurcodering systeem voor het voorstellen van kleuren aan de hand van deze drie basiskleuren}
}

\newglossaryentry{jpeg-exif}
{
	name={JPEG/Exif},
	description={Een bestandsformaat verder toegelicht in deel \ref{sec:afbeeldingscompressie-jpeg}}
}

\newglossaryentry{jpeg-jfif}
{
	name={JPEG/JFIF},
	description={Een bestandsformaat verder toegelicht in deel \ref{sec:afbeeldingscompressie-jpeg}}
}

\newglossaryentry{extensie}
{
	name={bestandsextensie},
	description={Een toevoeging op het einde van een bestandsnaam om weer te geven wat voor soort bestand het is},
	plural={bestandsextensies}
}

\newglossaryentry{dct}
{
	name={DCT},
	description={Discrete cosine transform is een wiskundige formule dat binnen datacompressie voornamelijk gebruikt wordt om een pixels te kunnen voorstellen als een makkelijk comprimeerbaar getal}
}

\newglossaryentry{wtcq}
{
	name={WTCQ},
	description={Wavelet/Trellis Coded Quantization is een compressie algoritme dat de start vormde voor JPEG2000. JPEG2000 wordt uigebreid besproken in deel \ref{sec:afbeeldingscompressie-jpeg2000}}
}

\newglossaryentry{iso}
{
	name={ISO},
	description={De International Organization for Standardization is een organisatie die internationale standaarden oplecht voor bijvoorbeeld compressie-algoritmen}
}

\newglossaryentry{jpf}
{
	name={JPF},
	description={JPf is de bestandsexentsie van JPEG2000. JPEG2000 wordt uigebreid besproken in deel \ref{sec:afbeeldingscompressie-jpeg2000}}
}

\newglossaryentry{intra-frame}
{
	name={intra-frame},
	description={intra-frame compressie is een techniek van afbeeldingen comprimeren binnen videocompressie dat uitgebreid besproken wordt in deel \ref{sec:videocompressie-intra-inter}},
	plural={intra-frames}
}

\newglossaryentry{inter-frame}
{
	name={inter-frame},
	description={inter-frame compressie is een techniek van afbeeldingen comprimeren binnen videocompressie dat uitgebreid besproken wordt in deel \ref{sec:videocompressie-intra-inter}},
	plural={inter-frames}
}

\newglossaryentry{decoder}
{
	name={decoder},
	description={Een softwareapplicatie dat een gecomprimeerd bestand terug omzet naar zijn originele vorm, al dan niet met kwaliteitsverlies },
	plural={decoders}
}

\newglossaryentry{encoder}
{
	name={encoder},
	description={Een softwareapplicatie dat een bestand omzet naar een gecomprimeerde vorm, al dan niet met kwaliteitsverlies},
	plural={encoders}
}

\newglossaryentry{heic}
{
	name={HEIC},
	description={Een afbeeldingsformaat dat uitgebreid besproken wordt in \ref{sec:afbeeldingscompressie-heif}},
}

\newglossaryentry{on-premise}
{
	name={on premise},
	description={Software is on-premise wanneer deze lokaal geïnstalleerd en gedraaid wordt op het toestel van de gebruiker},
}

\newglossaryentry{vector}
{
	name={vector},
	description={Een categorie voor afbeeldingsformaten dat verder toegelicht wordt in deel \ref{sec:afbeeldingscompressie-raster-vector}},
}

\newglossaryentry{cms}
{
	name={CMS},
	description={Een content management system is een systeem dat het voor een eindgebruiker eenvoudig moet maken om de inhoud binnen een IT-applicatie eenvoudig te beheren},
	plural={CMS'en}
}

\newglossaryentry{wordpress}
{
	name={WordPress},
	description={Een CMS voor web development}
}

\newglossaryentry{raster}
{
	name={raster},
	description={Een categorie voor afbeeldingsformaten dat verder toegelicht wordt in deel \ref{sec:afbeeldingscompressie-raster-vector}}
}

\newglossaryentry{lbhuffman}
{
	name={lbhuffman},
	description={Een bestandsextensie gebruikt voor binnen de datacompressietool verder besproken in deel \ref{sec:compressietool-opslaan}}
}

\newglossaryentry{lbrlea}
{
	name={lbrlea},
	description={Een bestandsextensie gebruikt voor binnen de datacompressietool verder besproken in deel \ref{sec:compressietool-opslaan}}
}

\newglossaryentry{lbrle}
{
	name={lbrle},
	description={Een bestandsextensie gebruikt voor binnen de datacompressietool verder besproken in deel \ref{sec:compressietool-opslaan}}
}

\newglossaryentry{html}
{
	name={HTML},
	description={Een programmeertaal voor het opbouwen van webpagina's}
}

\newglossaryentry{library}
{
	name={library},
	description={Een verzameling code dat door een programma gebruikt kan worden},
	plural={Libraries}
}

\newglossaryentry{string}
{
	name={string},
	description={Een term binnen programmeren dat slaat op een tekstfragment zonder opmaak},
	plural={strings}
}

\newglossaryentry{regex}
{
	name={regex},
	description={Regular Expressions zijn uitdrukkingen waarmee op zoek kan gegaan worden naar een tekst met een bepaalde structuur}
}

\newglossaryentry{recursieve-functie}
{
	name={recursieve functie},
	description={een recursieve functie is een functie die zichelf (al dan niet meermaals)oproept}
}

\newglossaryentry{array}
{
	name={array},
	description={Een term binnen programmeren dat slaat op een verzameling van variabelen}
}

\newglossaryentry{json}
{
	name={json},
	description={Een techniek voor het opslaan van variabelen in een string}
}

\newglossaryentry{rmse}
{
	name={rmse},
	description={Root mean square error, bespsroken in deel \ref{sec:kwaliteit-rmse}}
}

\newglossaryentry{ssim}
{
	name={SSIM},
	description={Structural similarity index, bespsroken in deel \ref{sec:kwaliteit-ssim}}
}

\newglossaryentry{vdp}
{
	name={VDP},
	description={Visual difference predictor is een term voor functies die visuele verschillen tussen afbeeldingen probeert uit te drukken}
}

\newglossaryentry{dxomark}
{
	name={DxOMark},
	description={Een erkende instelling dat de kwaliteit van, voornamelijk smartphone, camera's beoordeeld. DxOMark wordt verder toegelicht in deel \ref{sec:kwaliteit-dxomark}}
}

\newglossaryentry{maximum-compression}
{
	name={Maximum Compression},
	description={Een erkende instelling dat de compressieratio van lossles compressie-algoritmes beoordeeld. Maximum Compression wordt verder toegelicht in deel \ref{sec:kwaliteit-maximum-compression}}
}

\newglossaryentry{compressieratio}
{
	name={compressieratio},
	description={De verhouding waarmee een bestand gecomprimeerd is ten opzichte van het originele bestand}
}

\newglossaryentry{python}
{
	name={Python},
	description={Een programmeertaal dat voornamelijk gebruikt wordt voor het verwerken van data in de vorm van grote datascheets}
}

\newglossaryentry{pandas}
{
	name={Pandas},
	description={Een uitbreiding voor Python}
}

\newglossaryentry{illustrator}
{
	name={Adobe Illustrator},
	description={Computerssoftware voor het bewerken van vector afbeeldingen}
}

\newglossaryentry{svg}
{
	name={SVG},
	description={Scalable Vector Graphics zijn is een schaalbare vorm van vector afbeeldingen. Vector wordt verder toegelicht in deel \ref{sec:afbeeldingscompressie-raster-vector}}
}

\newglossaryentry{deflate}
{
	name={deflate},
	description={Een datacompressie techniek die voornamelijk binnen videcompressie gebruikt wordt}
}

\newglossaryentry{pixel-prediction}
{
	name={pixel prediction},
	description={Een datacompressie techniek die de pixel van een afbeelding gaat voorspellen aan de hand van een andere pixel}
}

\newglossaryentry{crc}
{
	name={CRC},
	description={Cyclic redundancy check is een foutdetectie code om te kunnen nagana of een bestand goed is toegekomen}
}

\newglossaryentry{bitplane}
{
	name={bitplane},
	description={Een term voor een groepering van bits},
	plural={bitplanes}
}

\newglossaryentry{mp4}
{
	name={MP4},
	description={Een container voor onder andere videocodecs. Videocodecs worden verder toegelicht in hoofdstuk \ref{ch:videocompressie}}
}

\newglossaryentry{avi}
{
	name={AVI},
	description={Een container voor onder andere videocodecs. Videocodecs worden verder toegelicht in hoofdstuk \ref{ch:videocompressie}}
}

\newglossaryentry{mkv}
{
	name={MKV},
	description={Een container voor onder andere videocodecs. Videocodecs worden verder toegelicht in hoofdstuk \ref{ch:videocompressie}}
}

\newglossaryentry{mpeg-4}
{
	name={MPEG-4},
	description={Een videocompressie algoritme dat de voorganger was van H.264/AVC toegelicht in deel \ref{sec:videocompressie-h264-AVC}}
}

\newglossaryentry{hlg}
{
	name={HLG},
	description={Hybrid Log-Gamma is een standaard voor het voorzien van HDR content}
}

\newglossaryentry{hdr}
{
	name={HDR},
	description={High Dynamic Range beelden zijn beelden die een hoog dynamisch bereik hebben}
}

\newglossaryentry{frame}
{
	name={frame},
	description={Een frame is één stilstaande afbeelding binnen een video},
	plural={frames}
}

\newglossaryentry{interlace}
{
	name={interlace},
	description={Een techniek waarbij twee frames tegelijk getoond worden waardoor de kijker een vals beeld krijgt dat de frame rate sneller is dan hij werkelijk is}
}

\newglossaryentry{frame-rate}
{
	name={frame rate},
	description={De snelheid waarmee frames elkaar opvolgen, vaak uitgedruk in frames per seconden (FPS)}
}

\newglossaryentry{pixel}
{
	name={pixel},
	description={Één punt binnen een rasterafbeelding},
	plural={pixels}
}

\newglossaryentry{vp8}
{
	name={VP8},
	description={Een video codec gemaakt door Google. WebP, besproken in deel \ref{sec:afbeeldingscompressie-webp}, is afkomstig van VP8. AV1 besproken in \ref{sec:videocompressie-av1} is een verre opvolger van VP8}
}

%Meer controle over placement van figuren
\usepackage{placeins}

%Kleuren in tabellen
\usepackage{xcolor}

%Cite van online beter
\DeclareCiteCommand{\urlcite}
[\mkbibfootnote]
{}
{\printfield{title} -- \printfield{url}}
{\addcomma\addspace}
{}

%%---------- Documenteigenschappen ----------
% De titel van het rapport/bachelorproef
\title{Je kijkt er naar, maar ziet het niet: datacompressie	principes --- ontstaan en uitdagingen --- implementaties --- \gls{afbeeldingscompressie}: \gls{png} | \gls{jpeg} | \gls{jpeg2000} | \gls{webp} | \gls{heif} --- \gls{videocompressie}:  \gls{h264-avc} | \gls{h264-svc} | \gls{h265} | \gls{av1}}

% Je eigen naam
\author{Bontinck Lennert}

% De naam van je promotor (lector van de opleiding)
\promotor{Wim De Bruyn}

% De naam van je co-promotor. Als je promotor ook je opdrachtgever is en je
% dus ook inhoudelijk begeleidt (en enkel dan!), mag je dit leeg laten.
\copromotor{Tom Paridaens}

% Indien je bachelorproef in opdracht van/in samenwerking met een bedrijf of
% externe organisatie geschreven is, geef je hier de naam. Zoniet laat je dit
% zoals het is.
\instelling{---}

% Academiejaar
\academiejaar{2018-2019}

% Examenperiode
%  - 1e semester = 1e examenperiode => 1
%  - 2e semester = 2e examenperiode => 2
%  - tweede zit  = 3e examenperiode => 3
\examenperiode{2}

%==================
% Inhoud document
%==================

\begin{document}

%---------- Taalselectie ----------
% Als je je bachelorproef in het Engels schrijft, haal dan onderstaande regel
% uit commentaar. Let op: de tekst op de voorkaft blijft in het Nederlands, en
% dat is ook de bedoeling!

%\selectlanguage{english}

%---------- Titelblad ----------
\inserttitlepage

%---------- Samenvatting, voorwoord ----------
\usechapterimagefalse
%==================
%% Voorwoord
%==================

\chapter*{Woord vooraf}
\label{ch:voorwoord}

De zoektocht naar een leuk, leerrijk en bruikbaar bachelorproefonderwerp is niet makkelijk. Wanneer ik echter op zoek was naar een geschikt \gls{afbeeldingsformaat} voor het bewaren van mijn steeds groeiende afbeeldingscollectie was ik stomverbaasd hoe weinig ik over dit onderwerp wist. Ook mijn vrienden, waarvan vele medestudenten Toegepaste Informatica te HoGent, konden mij geen antwoord geven op de vraag welk \gls{afbeeldingsformaat} een goede keuze zou zijn en kenden buiten het feit dat \gls{png} wel een transparante achtergrond kan hebben en \gls{jpeg} niet, geen echte verschillen tussen deze twee bekendste \glspl{afbeeldingsformaat}.

Ook op mijn stageplaats, waar websites en webshops gemaakt en onderhouden worden, waren maar weinig mensen zich echt bewust van de voordelen en nadelen van de verschillende \glspl{afbeeldingsformaat}. Meestal exporteerden ze afbeeldingen vanuit \gls{ps} met een kwaliteitsinstelling zodanig het eindbestand kleiner was dan 100kb om de performance van een website hoog te houden. De keuze voor een \gls{afbeeldingsformaat} anders dan \gls{jpeg} of \gls{png} wordt hier en in het algemeen te weinig overwegen ondanks dat hier een grote kwaliteitswinst gedaan kan worden met dezelfde of kleinere bestandsgrootte. 

Deze onwetendheid deed mij beseffen dat een onderzoek naar de verschillende soorten \gls{datacompressie} interessant kon worden. Zowel voor mede programmeurs die hun \gls{css} \gls{minifyen} om enkele kilobytes te besparen maar niet stilstaan hoeveel data ze kunnen besparen door het kiezen van een gepast \gls{afbeeldingsformaat} met de juiste \gls{render}instellingen en andere \gls{datacompressie}technieken. 

De achterliggende wiskunde en boeiende uitdagingen zoals \gls{dna-compressie} wisten mij ook te overtuigen dat dit onderwerp zeer verreikend zou zijn voor mij. Het vinden van een geweldige co-promotor, vakexpert Tom Paridaens, was dan ook de kers op de taart.

\pagebreak

Graag bedank ik dan ook mijn co-promotor, Tom Paridaens, voor de nauwe samenwerking binnen deze bachelorproef. Ook bedank ik graag Wim De Bruyn voor de leerrijke lessen onderzoekstechnieken die ons de nodige kennis brachten voor het voeren van een gegrond onderzoek. Bovendien is Wim De Bruyn de promotor van deze bachelorproef waarvoor ik hem ook wil danken.

Ook wil ik Mayté Bogaert, van MaytéB fotografie, bedanken voor het aanleveren van meerdere \gls{raw} bestanden die ik heb gebruikt voor het uitvoeren van mijn onderzoek naar een geschikt \gls{afbeeldingsformaat} voor een bepaalde \gls{use-case} besproken in deze bachelorproef.

Bert Van Vreckem en collega's verdienen ook een dankwoord voor het opstellen van een \LaTeX{} sjabloon voor deze bachelorproef. Tot slot bedank ik graag de deelnemers die aan de hand van mijn \gls{afbeeldingsevaluatietool} hebben bijgedragen naar het onderzoek van een geschikt \gls{afbeeldingsformaat} voor de \gls{use-case} besproken in deze bachelorproef.
%==================
%% Samenvatting
%==================

% De "abstract" of samenvatting is een kernachtige (~ 1 blz. voor een - op het einde
% thesis) synthese van het document.
%
% Deze aspecten moeten zeker aan bod komen:
% - Context: waarom is dit werk belangrijk?
% - Nood: waarom moest dit onderzocht worden?
% - Taak: wat heb je precies gedaan?
% - Object: wat staat in dit document geschreven?
% - Resultaat: wat was het resultaat?
% - Conclusie: wat is/zijn de belangrijkste conclusie(s)?
% - Perspectief: blijven er nog vragen open die in de toekomst nog kunnen
%    onderzocht worden? Wat is een mogelijk vervolg voor jouw onderzoek?
%
% LET OP! Een samenvatting is GEEN voorwoord!


\chapter*{Samenvatting}
\label{ch:samenvatting}

\Gls{datacompressie}: een fundamenteel onderdeel van de IT-wereld waar weinig belanghebbenden een basiskennis van hebben. Waarom is dat? Schrikken de complexe en zeer uitgebreide papers reeds geschreven over dit onderwerp geïnteresseerden af? Is het een te complex onderwerp om te voorzien in meer IT-gerelateerde opleidingen? Staat \gls{datacompressie} stil in de tijd dat standaarden als het \glspl{afbeeldingsformaat} \gls{jpeg} al meer dan twintig jaar het bekendste \gls{afbeeldingsformaat} is? Deze bachelorproef tracht een antwoord te geven op die vragen en de tal van andere onderzoeksvragen besproken in deel \ref{sec:onderzoeksvragen}. 

Dit document is gemaakt met een eenvoudig visie: de nodige basiskennis over het ontstaan van \gls{datacompressie}, de werking van enkele \glspl{compressie-algoritme}, tal van \glspl{afbeeldingsformaat} en video \glspl{codec} en de manieren voor het evalueren van compressiekwaliteit toe te lichten. Na het lezen van deze bachelorproef zal de lezer meer stilstaan bij de keuze voor een geschikt \gls{compressie-algoritme}.

Er is zowel een proof of concept \gls{compressietool} als een uitgebreide \gls{afbeeldingsevaluatietool} geschreven voor de bachelorproef die gratis zijn in gebruik en \gls{open-source} toegankelijk zijn op de \gls{github} repository van deze bachelorproef\urlcite{githubbachelorproef}. Er wordt dieper ingegaan op het ontstaan, de werking en de voordelen en nadelen van volgende \glspl{afbeeldingsformaat}: \gls{png} | \gls{jpeg} | \gls{jpeg2000} | \gls{webp} | \gls{heif}. Hetzelfde wordt gedaan voor de volgende video \glspl{codec}: \gls{h264-avc} | \gls{h264-svc} | \gls{h265} | \gls{av1}.

Er wordt een subjectief onderzoek gevoerd aan de hand van de eerder benoemde \gls{afbeeldingsevaluatietool}. Dit geeft samen met de theoretische kennis die zal verkregen worden door de bachelorproef een antwoord op de hoofdonderzoeksvraag van deze bachelorproef: Waarom moet er stilgestaan worden bij het gebruiken van \glspl{compressie-algoritme}, hoe kies je een geschikt \gls{compressie-algoritme} voor een bepaalde \gls{use-case} en hoe implementeer je dit het best?

Het volledige verloop van deze bachelorproef is beschreven in deel \ref{sec:opzet-bachelorproef}.

%---------- Inhoudstafel ----------
\pagestyle{empty} % Geen hoofding
\tableofcontents  % Voeg de inhoudstafel toe
\cleardoublepage  % Zorg dat volgende hoofstuk op een oneven pagina begint
\pagestyle{fancy} % Zet hoofding opnieuw aan

%---------- Kern ----------

%%=============================================================================
%% Inleiding
%%=============================================================================

\chapter{Inleiding}
\label{ch:inleiding}

\Gls{datacompressie} en het achterliggende compressie idee is niets nieuw. Integendeel, het is één van de oudste zaken binnen IT dat tot op heden van fundamenteel belang is voor zowat alle IT-toepassingen. Door de databesparing werkt alles niet alleen veelvouden sneller en goedkoper, maar worden bepaalde zaken die voorheen onmogelijk leken mogelijk. Denk hierbij bijvoorbeeld aan recente doorbraken binnen \gls{dna-compressie} dat het mogelijk maken steeds meer onderzoeken met betrekking tot het menselijk genoom uit te voeren.



\section{Probleemstelling}
\label{sec:probleemstelling}


%TODO: verder 
%De inleiding moet de lezer net genoeg informatie verschaffen om het onderwerp te begrijpen en in te zien waarom de onderzoeksvraag de moeite waard is om te onderzoeken. In de inleiding ga je literatuurverwijzingen beperken, zodat de tekst vlot leesbaar blijft. Je kan de inleiding verder onderverdelen in secties als dit de tekst verduidelijkt. Zaken die aan bod kunnen komen in de inleiding~\autocite{Pollefliet2011}:
%\begin{itemize}
%\item context, achtergrond
%\item afbakenen van het onderwerp
%\item verantwoording van het onderwerp, methodologie
%\item probleemstelling
%\item onderzoeksdoelstelling
%\item onderzoeksvraag
%\item \ldots
%\end{itemize}

\section{Onderzoeksvragen}
\label{sec:onderzoeksvragen}


%TODO: verder 
%Wees zo concreet mogelijk bij het formuleren van je onderzoeksvraag. Een onderzoeksvraag is trouwens iets waar nog niemand op dit moment een antwoord heeft (voor zover je kan nagaan). Het opzoeken van bestaande informatie (bv. ``welke tools bestaan er voor deze toepassing?'') is dus geen onderzoeksvraag. Je kan de onderzoeksvraag verder specifiëren in deelvragen. Bv.~als je onderzoek gaat over performantiemetingen, dan 

\section{Onderzoeksdoelstelling}
\label{sec:onderzoeksdoelstelling}

%TODO: verder 
%Wat is het beoogde resultaat van je bachelorproef? Wat zijn de criteria voor succes? Beschrijf die zo concreet mogelijk. Gaat het bv. om een proof-of-concept, een prototype, een verslag met aanbevelingen, een vergelijkende studie, enz.

\section{Opzet van deze bachelorproef}
\label{sec:opzet-bachelorproef}

% Het is gebruikelijk aan het einde van de inleiding een overzicht te
% geven van de opbouw van de rest van de tekst. Deze sectie bevat al een aanzet
% die je kan aanvullen/aanpassen in functie van je eigen tekst.

\subsection{Deel 1: situering en literatuurstudie}
\label{sec:opzet-bachelorproef-deel-1}

Deze paper zal zich in het eerste deel focussen op het toelichten van de belangrijke termen binnen \gls{datacompressie}. In hoofdstuk~\ref{ch:termen} is een lijst met belangrijker termen te vinden die binnen \gls{datacompressie} en deze paper vaak voorkomen. Doorheen deze paper zullen tal van referenties naar deze termen gelegd worden. 

Hoofdstuk~\ref{ch:methodologie} licht de gebruikte methodologie voor deze paper toe. Hieruit wordt duidelijk dat deze paper zo objectief mogelijk is opgesteld met een focus op duidelijkheid en reproduceerbaarheid.

Hoofdstuk~\ref{ch:literatuurstudie} behoort ook tot het situerende eerste deel en zal het ontstaan van \gls{datacompressie} en enkele basisprincipes toelichten. Een reeks van deze primitieve technieken zullen aan de hand van een voorbeeld toegelicht worden.
%todo: link naar voorbeeld en welke manieren

\subsection{Deel 2: compressie tool ontwikkelen}
\label{sec:opzet-bachelorproef-deel-2}
 
 In het tweede zal een basis \gls{datacompressie} tool programmatisch geïmplementeerd worden om de theorie uit het eerste deel in praktijk te zien.
 
Hoofdstuk~\ref{ch:compressietool} is hierdoor gericht voor technische lezers als programmeurs. Er is echter telkens voldoende randinformatie gegeven zodanig ook de minder technische lezers een blik achter de schermen kunnen verkrijgen.
 %TODO: welke taal tool gemaakt is en wat deze juist doet etc

\subsection{Deel 3: afbeelding- en videocompressie}
\label{sec:opzet-bachelorproef-deel-3}

In het derde deel worden twee sub domeinen van \gls{datacompressie} verder toegelicht; \gls{afbeeldingscompressie} en \gls{videocompressie}. 

In hoofdstuk~\ref{ch:afbeeldingcompressie} zal er dieper ingegaan worden op volgende afbeelding \glspl{codec}: \gls{jpeg}, \gls{jpeg2000} en \gls{png}. 
%TODO: aanpassen indien andere op input copromoter

In hoofdstuk~\ref{ch:videocompressie} zal er verder ingegaan worden op de gekende \gls{videocompressie} standaard: \gls{h264-avc} en \gls{h264-svc}. Ook de opvolger \gls{h265} en open source \gls{av1} zullen besproken worden.
%TODO: aanpassen indien andere op input copromoter

\subsection{Deel 4: onderzoek afbeelding compressie}
\label{sec:opzet-bachelorproef-deel-4}

In het vierde deel wordt besproken hoe compressiemethoden binnen video en afbeelding geëvalueerd worden. Hoofdstuk~\ref{ch:kwaliteit} zal enkele veel gebruikte tools en methoden voor objectieve en subjectieve metingen toelichten.

In hoofdstuk~\ref{ch:onderzoek} wordt een subjectieve test voor het evalueren van afbeeldingskwaliteit opgesteld voor portretfoto's. Hierbij zullen enkele van de besproken  \glspl{codec} uit hoofdstuk~\ref{ch:afbeeldingcompressie} tegen elkaar concurreren. De gebruikte tool is voor deze paper opgesteld en zal \gls{open-source} toegankelijk zijn wat het eenvoudig mogelijk maakt om een gelijkaardig onderzoek uit te voeren.
%TODO: in staat stelllen als bv prog of content zelf kiezen welk gebruiken

\subsection{Deel 5: uitdagingen en conclusie}
\label{sec:opzet-bachelorproef-deel-5}

In het vijfde deel zullen de huidige uitdagingen van \gls{datacompressie} kort toegelicht worden. Zo zal hoofdstuk~\ref{ch:uitdagingen} een beeld geven van de taken die mensen als Tom Paridaens, co-promoter voor deze paper, krijgen.
%TODO: copromoter zijn job en eerbetoon vermelden

In hoofdstuk~\ref{ch:conclusie} wordt kort teruggeblikt op de paper en worden enkele besluiten uit het onderzoek van hoofdstuk~\ref{ch:onderzoek} opgesomd. Daarbij wordt ook een aanzet gegeven om zelf meer na te denken over het gebruik van \gls{datacompressie} en bepaalde  \glspl{codec} in projecten, of nog beter, zelf een onderzoek uit te voeren!
%TODO: lezer aanzetten meer etc
%deel 1
\chapter{Belangrijke termen in datacompressie}
\label{ch:termen}

Om deze paper vlot te kunnen lezen, zijn er enkele termen die de lezer moet kennen. Termen die vaak voorkomen in datacompressie en doorheen deze paper worden hier opgesomd. Er zal een referentie voorzien zijn naar deze lijst bij het gebruik van een term uit de lijst.

\glsaddall
\printglossary[title=Woordenlijst]

%%=============================================================================
%% Methodologie
%%=============================================================================

\chapter{Methodologie}
\label{ch:methodologie}

%% TODO: Hoe ben je te werk gegaan? Verdeel je onderzoek in grote fasen, en
%% licht in elke fase toe welke stappen je gevolgd hebt. Verantwoord waarom je
%% op deze manier te werk gegaan bent. Je moet kunnen aantonen dat je de best
%% mogelijke manier toegepast hebt om een antwoord te vinden op de
%% onderzoeksvraag.

%TODO: zeggen dat je literatuurstudie van betrouwbare bronnen komt 
%TODO: zeggen dat je tool open source is en zelfgeschreven met validatie van co-promoter 
%TODO: zeggen dat je opzoekingswerk afb en vid compressie goed is door co-promoter en zijn collega's
%TODO: zeggen dat eval tool open source zelfgemaakt is met zelfde pc etc en dat daarom betrouwbare resultaten zijn die kunnen nagebootst worden
%TODO: zeggen dat door werkwijze toelichten zelf een test opgezet kan worden zodanig dit besluit gevalideerd kan worden of zelf verder onderzoek
\chapter{Literatuurstudie}
\label{ch:literatuurstudie}

% Tip: Begin elk hoofdstuk met een paragraaf inleiding die beschrijft hoe
% dit hoofdstuk past binnen het geheel van de bachelorproef. Geef in het
% bijzonder aan wat de link is met het vorige en volgende hoofdstuk.

Deze literatuurstudie zal samen met de lijst van termen uit hoofdstuk \ref{ch:termen} de nodige achtergrondinformatie bieden om de volgende delen van de paper te begrijpen. Er zal verwezen worden naar meerdere papers van andere instellingen zodanig dat u zich verder kan inlezen waar gewenst.

% Pas na deze inleidende paragraaf komt de eerste sectiehoofding.


%Dit hoofdstuk bevat je literatuurstudie. De inhoud gaat verder op de inleiding, maar zal het onderwerp van de bachelorproef *diepgaand* uitspitten. De bedoeling is dat de lezer na lezing van dit hoofdstuk helemaal op de hoogte is van de huidige stand van zaken (state-of-the-art) in het onderzoeksdomein. Iemand die niet vertrouwd is met het onderwerp, weet nu voldoende om de rest van het verhaal te kunnen volgen, zonder dat die er nog andere informatie moet over opzoeken \autocite{Pollefliet2011}.

%Je verwijst bij elke bewering die je doet, vakterm die je introduceert, enz. naar je bronnen. In \LaTeX{} kan dat met het commando \texttt{$\backslash${textcite\{\}}} of \texttt{$\backslash${autocite\{\}}}. Als argument van het commando geef je de ``sleutel'' van een ``record'' in een bibliografische databank in het Bib\LaTeX{}-formaat (een tekstbestand). Als je expliciet naar de auteur verwijst in de zin, gebruik je \texttt{$\backslash${}textcite\{\}}.
%Soms wil je de auteur niet expliciet vernoemen, dan gebruik je \texttt{$\backslash${}autocite\{\}}. In de volgende paragraaf een voorbeeld van elk.


%\textcite{Knuth1998} schreef een van de standaardwerken over sorteer- en zoekcompressie-algoritmen. Experten zijn het erover eens dat cloud computing een interessante opportuniteit vormen, zowel voor gebruikers als voor dienstverleners op vlak van informatietechnologie~\autocite{Creeger2009}.

\section{Bestandsgrootte en dataopslag}
\label{sec:bestandsgrootte-dataopslag}
Zoals in de definitie van \gls{datacompressie} besproken is, bestaat \gls{datacompressie} uit het digitaal opslaan van een bestand met zo weinig mogelijk \glspl{bit}.

\Gls{bit} staat kort voor binary digit. Een bit wordt beschouwd als de kleinste data eenheid voor dataopslag. Een bit kan twee waarden aannemen, deze worden voorgesteld door 1 of 0 (binair talstelsel) maar kunnen ook geïnterpreteerd worden als aan of uit, ja of nee…

\subsection{Voorvoegsels voor het uitdrukken van bestandsgrootte}
\label{sec:bestandsgrootte-dataopslag-voorvoegsels}

Bestandsgroottes worden meestal uitgedrukt in bytes (8 bits), al dan niet met voorvoegsel dat een veelvoud voorstelt. Deze voorvoegsels en het door elkaar gebruik van de \gls{si-voorstelling-bestandsgrootte} en \gls{binaire-voorstelling-bestandsgrootte} kan voor enige verwarring zorgen. Denk hierbij aan het fenomeen dat harde schijven die geadverteerd zijn als 1TB (\gls{si-voorstelling-bestandsgrootte}) overeen komt met 931GiB (\gls{binaire-voorstelling-bestandsgrootte}) op de meeste besturingssystemen. Doorheen deze paper zal de \gls{si-voorstelling-bestandsgrootte} gebruikt worden.

\subsubsection{SI voorstelling}
\label{sec:bestandsgrootte-dataopslag-voorvoegsels-si}

De \gls{si-voorstelling-bestandsgrootte} gebruikt als basis 1000 wat overeen komt met $ 10^{3} $. SI staat voor International System of Units en beschrijft. Het wordt beschouwd als een moderne vorm op het metrisch stelsel. SI beschrijft onder IEC 60027 het gebruik van bepaalde voorvoegsels voor het uitdrukken van machten op 10. (\cite{iec60027}) Een conversietabel is hieronder raadpleegbaar. 

\FloatBarrier
\begin{table}[h]
	\begin{tabular}{lll}
		Voorvoegsel & symbool & waarde \\
		kilo & Ki & $ 10^{3} = 1000^{1} $  = 1 000 \\
		mega & Mi & $ 10^{6} = 1000^{2} $  = 1 000 000 \\
		giga & Gi & $ 10^{9} = 1000^{3} $  = 1 000 000 000
	\end{tabular}
\end{table}
\FloatBarrier

\subsubsection{binaire voorstelling}
\label{sec:bestandsgrootte-dataopslag-voorvoegsels-binair}

De \gls{binaire-voorstelling-bestandsgrootte} gebruikt als basis 1024 wat overeen komt met $ 2^{10} $. Deze voorstelling is een standaardisatie opgelegd door \gls{ieee} 1541-2002 (\cite{ieee15412002}). Een conversietabel is hieronder raadpleegbaar.

\FloatBarrier
\begin{table}[h]
	\begin{tabular}{lll}
		Voorvoegsel & symbool & waarde \\
		kibi & Ki & $ 2^{10} = 1024^{1} $  = 1 024 \\
		mebi & Mi & $ 2^{20} = 1024^{2} $ = 1048 576 \\
		gibi & Gi & $ 2^{30} = 1024^{3} $ = 1 073 741 824
	\end{tabular}
\end{table}
\FloatBarrier

\subsubsection{clustergrootte}
\label{sec:bestandsgrootte-dataopslag-clustergrootte}

Een andere belangrijke term bij dataopslag is de \gls{clustergrootte}. Data moet namelijk bijgehouden worden in één of meerdere \glspl{cluster} op een opslagmedium zodat naar deze \glspl{cluster} kan verwezen worden voor het lezen van de data. 

Aangezien alle data steeds minstens in één \gls{cluster} staat en twee verschillende databestanden nooit een \gls{cluster} kunnen delen kan dit voor opslagruimteverlies zorgen. 

Neem bijvoorbeeld een \gls{clustergrootte} van 4096 \glspl{byte}, een vaak voorkomende \gls{clustergrootte}. Als in deze situatie een bestand van 2000 \glspl{byte} groot zou opgeslagen worden, zijn de overige 2096 \glspl{byte} aan opslagcapaciteit op die schijf verloren. Een bestand van 4097 \glspl{byte} zou twee \glspl{cluster} in beslag nemen waardoor 4095 \glspl{byte} verloren gaan. 

Dit speelt vooral een rol wanneer \gls{datacompressie} gebruikt wordt voor het besparen van opslagruimte op een opslagmedium. Een gecomprimeerd bestand met een kleinere bestandsgrootte dat dezelfde hoeveelheid \glspl{clustergrootte} nodig heeft op het medium zal dus niet voor plaats besparing zorgen op dat opslagmedium. 

Theoretisch gezien zal er wel een verbetering te zien zijn in \glspl{leestijd} en de gebruikte \gls{bandbreedte} bij een bestandsoverdracht omdat de effectieve bestandsgrootte kleiner is. 

Er zijn tal van reden waarom een andere \gls{clustergrootte} aangeraden is, een recente discussie is terug te vinden in een blogpost van Microsoft\urlcite{microsoftblogcluster}

\section{Ontstaan datacompressie en primitieve technieken}
\label{sec:ontstaan-datacompressie-primitieve-technieken}
\subsection{Eerste vorm van datacompressie}
\label{sec:ontstaan-datacompressie-primitieve-technieken-eerste-vorm}
Vele onderzoekers zijn het erover eens dat \gls{datacompressie} dateert van voor de uitvinding van de computer. Vele onderzoekers zijn het er over eens dat morsecode de eerste vorm was van \gls{datacompressie}. Morsecode is uitgevonden in 1832 door Samuel F.B. Morse en wordt aanzien als eerste vorm van datacompressie doordat veel voorkomende letters een kortere audiotoon kregen dan minder gebruikte letters. (\cite{morsecode})

\subsection{Ontwikkeling datacompressie binnen IT}
\label{sec:ontstaan-datacompressie-primitieve-technieken-binnen-it}
Bij de prille opkomst van mainframe eind de jaren 40 en begin de jaren 50 zijn twee belangrijke doorbraken binnen \gls{datacompressie} gemaakt. Beiden maken gebruik van \gls{prefix-code}. De originele uitvinder van dit soort compressie was Shannon Claude dat Shannon coding uitvond, een proof of concept voor zijn artikel \citetitle{shannon1948} (\cite{shannon1948}). In diezelfde periode werd ook Shannon-Fano coding voorgesteld, een project samen met Robert Fano dat verschillende \glspl{use-case} had. Geen van beide technieken waren echter optimaal aangezien de \glspl{compressie-algoritme} niet gegarandeerd de korst mogelijke prefix codes gaven. 

\Gls{huffman-coding}, voorgesteld in  \citetitle{huffman} (\cite{huffman}) was een optimale variant op deze techniek. Dit is een \gls{compressie-algoritme} door David Huffman gemaakt als academie opdracht dat bijna 70 jaar na publicatie nog steeds de basis legt voor vele \gls{lossless} \gls{datacompressie} \glspl{compressie-algoritme}. Deze soort \glspl{compressie-algoritme} worden \gls{frequency-based} \glspl{compressie-algoritme} genoemd. Het exacte verschil tussen \gls{huffman-coding} en Shannon-Fano coding en meer informatie over deze \glspl{compressie-algoritme} zijn beschreven in \citetitle{lelewer87datacompression} (\cite{lelewer87datacompression}). In deel \ref{sec:primitieve-technieken-voorbeeld-huffman-encoding} wordt een praktisch voorbeeld van \gls{huffman-coding} uitgewerkt. In hoofdstuk  \ref{ch:compressietool} zal onder andere \gls{huffman-coding} gebruikt worden voor het maken van de compressietool.

De jaren 70 en 80 zorgden voor tal van belangrijke doorbraken binnen \gls{datacompressie}. Dit was te weiden aan de opkomst van het internet en de steeds groter wordende bestanden. Ook werd hardware matige compressie (zoals \gls{prefix-code} met vaste \gls{lookup-table} voor tekstbestanden) steeds meer vervangen door dynamische compressie (codegewijs). 

Deze eerste softwareoplossingen waren veelal implementatie van \gls{huffman-coding}, eventueel met kleine aanpassingen. Eind de jaren 70 waren de eerste Lempel-Ziv \gls{compressie-algoritme} uitgevonden: LZ77 en LZ78. Dit zijn de grondleggers van \gls{dictionary-coding}. Een veelgebruikte variant van LZ798 is LZW (1984). Net zoals \gls{huffman-coding} de basis legde voor vele van de eerste softwareoplossingen zorgde de grondleggers van \gls{dictionary-coding} voor vele nieuwe softwareoplossingen. De doorgroei van deze \glspl{compressie-algoritme} is zichtbaar in figuur \ref{fig:lossles-datacompressie-overzicht}.

\begin{figure}
	\includegraphics{img/literatuurstudie/lossles_datacompressie_overzicht.png}
	\caption{Lossless datacompressie overzicht (\cite{ethwcompressionhistory})}
	\label{fig:lossles-datacompressie-overzicht}
\end{figure}

Het grootste verschil tussen \gls{prefix-coding} en \gls{dictionary-coding} zit in de naam zelf. Bij \gls{prefix-coding} wordt elk karakter vervangen door een \gls{prefix-code} terwijl bij \gls{dictionary-coding} een reeks van karakters vervangen kunnen worden door één enkele \gls{prefix-code}.  

Eind de jaren 80 en begin de jaren 90, door de digitalisering van afbeeldingen en muziek, begonnen \gls{lossy} \glspl{compressie-algoritme} steeds meer op te komen. Het verschil tussen \gls{lossless} en \gls{lossy} \glspl{compressie-algoritme} wordt in deel \ref{sec:ontstaan-datacompressie-lossless-lossy} verder besproken.


\subsection{Primitieve technieken: een voorbeeld}
\label{sec:primitieve-technieken-voorbeeld}
TODO

\subsubsection{Situering}
\label{sec:primitieve-technieken-voorbeeld-situering}
TODO

\subsubsection{ASCII encoding en decoding}
\label{sec:primitieve-technieken-voorbeeld-ascii}
TODO

\paragraph{ASCII Probleemstelling 1: 8 bits per karakter}
\label{par:primitieve-technieken-voorbeeld-ascii-probleem-1}
TODO

\subsubsection{RLE: run length encoding en decoding}
\label{sec:primitieve-technieken-voorbeeld-rle}
TODO

\paragraph{RLE Probleemstelling 1: gecomprimeerd bestand groter dan bron}
\label{par:primitieve-technieken-voorbeeld-rle-probleem-1}
TODO

\subsubsection{Huffman coding}
\label{sec:primitieve-technieken-voorbeeld-huffman-encoding}
TODO

\paragraph{Huffman encoding stap 1: meerdere bomen}
\label{par:primitieve-technieken-voorbeeld-huffman-encoding-1}
TODO

\paragraph{Huffman encoding stap 2: bomen samenvoegen}
\label{par:primitieve-technieken-voorbeeld-huffman-encoding-2}
TODO

\paragraph{stap 3: prefix tabel (optioneel)}
\label{par:primitieve-technieken-voorbeeld-huffman-encoding-3}
TODO

\paragraph{stap 4: encoding}
\label{par:primitieve-technieken-voorbeeld-huffman-encoding-4}
TODO

\paragraph{Huffman decoding}
\label{par:primitieve-technieken-voorbeeld-huffman-decoding}
TODO

\paragraph{Huffman coding Probleemstelling 1: binaire boom niet opgeslagen}
\label{par:primitieve-technieken-voorbeeld-huffman-probleem-1}
TODO

\paragraph{Huffman coding Probleemstelling 2: gecomprimeerd bestand groter dan bron}
\label{par:primitieve-technieken-voorbeeld-huffman-probleem-2}
TODO

\paragraph{Huffman coding Probleemstelling 3: overlappende prefix codes}
\label{par:primitieve-technieken-voorbeeld-huffman-probleem-3}
TODO

\section{Lossless vs lossy datacompressie}
\label{sec:ontstaan-datacompressie-lossless-lossy}

%deel 2
\chapter{Proof of concept compressietool}
\label{ch:compressietool}

In deel \ref{sec:primitieve-technieken-voorbeeld-rle} en \ref{sec:primitieve-technieken-voorbeeld-huffman-encoding} wordt uitgelegd hoe \gls{rle-long} en \gls{huffman-coding} werken. Dit hoofdstuk focust zich op de implementatie van deze \gls{compressie-algoritme}. Er zal een proof of concept \gls{compressietool} gebouwd worden in \gls{php} en de werking zal toegelicht worden. Deze is in staat om input bestanden onder de vorm van simpele tekst in een txt bestand om te zetten naar hen gecomprimeerde vorm.
 
\section{Gebruikte technologie}
\label{sec:compressietool-gebruikte-technologie}

Deze \gls{compressietool} is geschreven in \gls{php}. Dit maakt het mogelijk te tool eenvoudig lokaal te runnen door het gebruik van een webserver omgeving als \gls{xampp} of hem online te zetten op een \gls{hosting} platform. Deze tool is online raadpleegbaar op de website van Lennert Bontinck\urlcite{compressietool}.

\section{Run length encoding}
\label{sec:compressietool-rle}

TODO
%TODO: compressietool

\section{Huffman Coding}
\label{sec:compressietool-huffman}

TODO
%TODO: compressietool

\section{Patronen}
\label{sec:compressietool-patronen}

TODO
%TODO: compressietool

\section{Resultaten}
\label{sec:compressietool-resultaten}

TODO
%TODO: compressietool
%deel 3
\chapter{Afbeeldingscompressie}
\label{ch:afbeeldingscompressie}

\Gls{afbeeldingscompressie} is een subdomein van \gls{datacompressie}. \Gls{afbeeldingscompressie} bestaat er uit een afbeelding in zo weinig mogelijk aantal \glspl{bit} op te slaan terwijl een aanvaardbare kwaliteit behouden blijft. Dit kan zowel via \gls{lossless} als \gls{lossy} \glspl{compressie-algoritme}.

\Gls{afbeeldingscompressie} is zeer belangrijk voor de snelheid en schaalbaarheid van IT-projecten alsook voor de gebruikerservaring. Bijna alle IT-projecten bevatten afbeeldingen, dit is zeker bij websites het geval.

In een officiële blog post van Google, \citetitle{googleinternetspeed} (\cite{googleinternetspeed}), wordt besproken dat een efficiënte webpagina steeds onder de twee seconden zou moeten geladen kunnen worden. In diezelfde post wordt zelfs aangehaald dat binnen een halve second de gebruiker reeds inhoud van de webpagina zou moeten zien.

Uit een nog recenter onderzoek door Akamai, \citetitle{akamaiinternetspeed} (\cite{akamaiinternetspeed}), blijkt dat bij een laadtijd van meer dan drie seconden op een mobiele webpagina meer dan de helft van de bezoekers de webpagina verlaat.

Afbeeldingen zijn meestal de grootste bestanden die bij het laden van een webpagina gedownload moeten worden. Het is dan ook met een juiste keuze aan \gls{afbeeldingscompressie} dat het meeste tijd voor de bezoeker kan gespaard worden. 

Dit hoofdstuk licht toe welke soorten \gls{afbeeldingscompressie} er bestaan. Ook enkele \glspl{afbeeldingsformaat} zullen besproken worden samen met hun voordelen en nadelen. De mogelijke problemen en oplossingen bij de implementatie van nieuwe generatie \glspl{afbeeldingsformaat} zal alsook besproken worden.

Hoofdstuk \ref{ch:kwaliteit} bespreekt hoe de kwaliteit van een \gls{afbeeldingsformaat} objectief en subjectief beoordeeld kan worden. In hoofdstuk \ref{ch:onderzoek} wordt voor een bepaalde \gls{use-case} een geschikt \gls{afbeeldingsformaat} gezocht aan de hand van een subjectief onderzoek met een voor deze bachelorproef geschreven \gls{afbeeldingsevaluatietool}.

\section{Raster vs vector afbeeldingsformaten}
\label{sec:afbeeldingscompressie-raster-vector}

Binnen \glspl{afbeeldingsformaat} kan men een onderscheid maken tussen twee soorten \glspl{afbeeldingsformaat}: \gls{raster} en \gls{vector}. Net zoals bij de keuze tussen \gls{lossless} en \gls{lossy} is er geen eenduidig antwoord welke beter is. De keuze is \gls{use-case} gebonden. Het grote verschil tussen de twee is de manier waarop ze de data bijhouden voor het weergeven van de afbeelding.

\Gls{raster} afbeeldingen bestaan uit een grid van kleine punten, meestal vierkantjes, dat pixels genoemd worden. Voor elke pixel wordt een bepaalde kleur bijgehouden. Bij bepaalde \glspl{afbeeldingsformaat} kan ook de doorzichtigheid van een pixel bijgehouden worden. Het is door het naast elkaar weergeven van al deze pixels dat we een afbeelding verkrijgen. Hoe dichter deze pixels bij elkaar staan hoe scherper de afbeelding oogt. Wanneer je voldoende inzoomt op een afbeeldingen met een \gls{raster} \gls{afbeeldingsformaat} kan je deze pixels visueel zien.

\Gls{vector} afbeeldingen werken niet met pixels maar kunnen aanzien worden als een soort tekening. Een afbeelding bestaat dan uit allerlei vormen, denk hierbij onder andere aan cirkels, rechthoeken en (gebogen) lijnen. Voor elk van deze vormen wordt dan de kleur bijgehouden, de start en eindpunten, de boog in graden, ... Het is door het samenvoegen van al deze vormen dat we de finale afbeelding verkrijgen. Het grote voordeel van vector \glspl{afbeeldingsformaat} is dat je kan blijven inzoomen zonder dat je daarbij kwaliteit verliest. Er is namelijk geen sprake van een gelimiteerd aantal pixels maar van bepaalde vormen die zo groot of klein als gewenst getekend kunnen worden. Figuur \ref{fig:raster-vs-vector} geeft dit duidelijk weer.

\begin{figure}
	\centering
	\fbox{\includegraphics[width=0.5\linewidth]{img/afbeeldingscompressie/raster-vector.jpg}}
	\caption{Illustratie dat het verschil tussen \gls{raster} en \gls{vector} weergeeft (\cite{rastervsvector})}
	\label{fig:raster-vs-vector}
\end{figure}

\Gls{raster} \glspl{afbeeldingsformaat} zijn ideaal voor het opslaan van foto's aangezien de beeldsensor van een camera ook bestaat uit meerdere punten (pixels). Een foto met natuurlijke objecten is veelal ook (veel) te complex om op een efficiënte en realistische manier te kunnen voorstellen met de vormen dat \gls{raster} \glspl{afbeeldingsformaat}  ondersteunen. Grafisch werk als logo's en iconen kunnen vaak wel opgeslagen worden door deze vormen waardoor een \gls{vector} \gls{afbeeldingsformaat} aangeraden is.

Gekende software voor het maken en bewerken van \gls{raster} afbeeldingen is \gls{ps}, voor het maken en bewerken van \gls{vector} afbeeldingen is dit \gls{illustrator}. De \glspl{afbeeldingsformaat} die in dit hoofdstuk besproken zullen worden zijn allen voorbeelden van \gls{raster} \glspl{afbeeldingsformaat}. Een gekend voorbeeld van \gls{vector} \glspl{afbeeldingsformaat} is het \gls{svg} \gls{afbeeldingsformaat}.

\section{Afbeeldingsformaten}
\label{sec:afbeeldingscompressie-afbeeldingsformaten}

Het nemen van een foto met een digitale camera komt overeen met het openstellen van de beeldsensor aan licht voor een bepaalde duur (sluitertijd). De gegevens die gedurende die tijd waargenomen worden, kunnen direct verwerkt worden en gecomprimeerd opgeslagen worden als bijvoorbeeld een \gls{jpeg}. Dit is hoe de meeste smartphone camera’s te werk gaan. Bij vele, voornamelijk professionele, toestellen kan ingesteld worden dat er geen gecomprimeerd \gls{afbeeldingsformaat} gebruikt moet worden maar maar een \gls{raw} \gls{afbeeldingsformaat}. 

Hieronder worden enkele \glspl{afbeeldingsformaat} verder toegelicht. Het is belangrijk om te weten dat dit niet de enige \glspl{afbeeldingsformaat} zijn die bestaan. De lijst van \glspl{afbeeldingsformaat} blijft groeien en bestaande \glspl{afbeeldingsformaat} kunnen extensies krijgen om gekende problemen als bepaalde \glspl{artefact} tegen te gaan. 

Een mogelijke manier om deze artefacten tegen te gaan is het gebruik van een andere \gls{wavelet} zoals besproken in \citetitle{inproceedings} (\cite{inproceedings}). Recente doorbraken binnen \gls{ai} maken het zelfs mogelijk op een nog meer dynamische manier \glspl{artefact} in \glspl{afbeeldingsformaat} tegen te gaan zoals besproken in \citetitle{jpegartefactereductionai} (\cite{jpegartefactereductionai}).


\subsection{RAW}
\label{sec:afbeeldingscompressie-raw}

Een \gls{raw} \gls{afbeeldingsformaat} bevat alle ruwe, onbewerkte en ongecomprimeerde gegevens die de beeldsensor heeft vastgelegd. In een \gls{raw} \gls{afbeeldingsformaat} wordt ook tal van \gls{meta-data}  bijgehouden zoals de gebruikte camera en lens, hun instellingen… De bestandsgrootte van een \gls{raw} bestand is hierdoor aanzienlijk. 

\Gls{raw} is geen afkorting noch een echt \gls{afbeeldingsformaat} zoals \gls{jpeg} of \gls{png} maar een benaming voor een groep van \glspl{afbeeldingsformaat} die voldoen aan de benoemde eigenschappen. Het effectieve \gls{afbeeldingsformaat} kan verschillen van merk tot merk en zelfs van toestel tot toestel. Zo zijn de \gls{raw} bestanden gebruikt voor het onderzoek in hoofdstuk \ref{ch:onderzoek} afkomstig van een Nikon toestel en zijn ze opgeslagen in het \gls{nef} \gls{afbeeldingsformaat}.

Hoewel er reeds voorstellen zijn gedaan voor een open \gls{raw} standaard om bewerkingen makkelijk te maken zoals \gls{dng} van Adobe is er tot op heden een grote diversiteit aan \gls{raw} \glspl{afbeeldingsformaat} te vinden. Dit vormt binnen \gls{afbeeldingscompressie} en de evaluatie ervan enkele nadelen. Doordat er zo veel verschillende \gls{raw} \glspl{afbeeldingsformaat} zijn en dus veel uiteenlopende licenties en rechten, is het een uitdaging een \gls{compressie-algoritme} voor een \gls{afbeeldingsformaat} te maken dat alle \gls{raw} \glspl{afbeeldingsformaat} ondersteund als input. 

Starten van een \gls{raw} \gls{afbeeldingsformaat} voor het evalueren van \gls{afbeeldingscompressie} is echter wel aangeraden aangezien zelfs het gebruik van een \gls{lossless} \gls{afbeeldingsformaat} voor verlies van \gls{meta-data} kan zorgen zoals eerder besproken. Het weglaten van deze \gls{meta-data} kan onderdeel zijn van het gekozen \gls{afbeeldingsformaat} en bijhorend \gls{compressie-algoritme}. Als deze \gls{meta-data} niet inbegrepen is in het inputbestand wordt het weglaten ervan niet gerepresenteerd in de eindscore wat voor een vals beeld kan zorgen.

\subsection{PNG}
\label{sec:afbeeldingscompressie-png}

Portable Network Graphics is een \gls{lossless} \gls{afbeeldingsformaat}. Het is ontwikkeld door de Portable Network Graphics Development Group met een eerste beta in 1995, een draft versie voor \gls{w3c} eind 1995, een officiële \gls{w3c} voorstelling op 1 juli 1996 en goedgekeurd als \gls{w3c} aanbeveling op 1 oktober 1996. Datums uit \citetitle{pnghistory} (\cite{pnghistory}).

\Gls{png} was gemaakt als vervanger van het toen veelgebruikte \gls{gif} \gls{afbeeldingsformaat} dat net zoals vele andere \glspl{compressie-algoritme} een nachtmerrie van licenties en patenten aan het worden was. 

\Gls{png} had als doel een \gls{afbeeldingsformaat} te worden dat zeer flexibel is, gemakkelijk te gebruiken is op het internet en allerlei soorten afbeeldingen ondersteund. Meer dan 20 jaar later slaagt het daar nog altijd in.

\subsubsection{PNG: werking}
\label{sec:afbeeldingscompressie-png-werking}

De werking van \gls{png} wordt niet diepgaand uitgelegd omdat hierover gehele papers geschreven kunnen worden. De informatie over de werking van \gls{png} is uit het uitstekende boek over \gls{datacompressie}: \citetitle{Salomon2006} (\cite{Salomon2006}) gehaald waar nog dieper op de werking van \gls{png} wordt ingegaan.

Een \gls{png} bestand is opgebouwd uit verschillende delen dat 'chunks' genoemd worden. Een chunk bestaat uit:

\begin{itemize}
	
	\item Grootte van het dataveld in deze chunk.
	
	\item Naam van deze chunk. Vier letters lang.
	
	\item Dataveld.
	
	\item Een \gls{crc} voor het valideren van de chunk. Deze is steeds 32 \glspl{bit} groot.
	
\end{itemize}

Chunks kunnen verplicht (critical chunks) of optioneel (ancillary chunks) zijn. Optionele chunks kunnen door een \gls{decoder} genegeerd worden en kunnen zaken als \gls{meta-data} bevatten. Critical chunks moeten door de \gls{decoder} gelezen kunnen worden en de \gls{crc} controle moet kloppen anders wordt er een error weergegeven. Een chunk is verplicht wanneer de eerste letter een hoofdletter is. Indien de tweede letter een hoofdletter is duid het op een standaard door \gls{png} voorziene chunk, anders is het een uitbreiding dat door een extensie op \gls{png} is toegevoegd. De derde letter is steeds een hoofdletter en de vierde letter is een hoofdletter wanneer de chunk niet gekopieerd mag worden. 

Een \gls{png} bestand kan één van de vijf volgende kleursoorten hebben:

\begin{itemize}
	\item \Gls{rgb} met acht of zestien \glspl{bitplane}.
	
	\item \Gls{rgb} met acht of zestien \glspl{bitplane} en een alpha kanaal voor transparantie.
	
	\item Palette met één, twee vier of acht \glspl{bitplane}.
	
	\item Grijsschaal met één, twee vier, acht of zestien \glspl{bitplane}.
	
	\item Grijsschaal met acht of zestien \glspl{bitplane} en een alpha kanaal voor transparantie.
	
\end{itemize}
 

De \gls{decoder} kan een \gls{png} progressief inladen wanneer er gebruik gemaakt wordt van de Adam zeven interlacing functionaliteit. Deze interlacing verdeeld de afbeelding in gelijke blokken van 64 pixels (8x8) dat in zeven stappen ingeladen wordt. Eerst wordt de pixel links vanboven ingeladen een weergegeven over alle 64 pixels in die blok. Wanneer dit voor alle blokken gedaan is worden in totaal twee pixels ingeladen per blok dat weer gekopieerd worden overheen de gehele blok. Het aantal in te laden pixels blijft verdubbelen tot alle 64 de pixels van dat blok ingeladen zijn en dus het gehele \gls{png} bestand ingeladen is.

De effectieve compressie van \gls{png} gebeurd door \gls{pixel-prediction} en \gls{deflate}. Eerst wordt een waarde voor een pixel berekend door \gls{pixel-prediction}, de 'predicted value', waarna het verschil tussen de pixel en de predicted value opgeslaan wordt door het te encoden met \gls{deflate}. De pixel omzetten naar het verschil van de predicted value is niet hetgeen dat voor de \gls{datacompressie} zorgt. Dit zorgt er voor dat de pixel gerepresenteerd wordt door een volgorde van \glspl{bit} dat met een hoger compressieratio kan gecomprimeerd worden door het \gls{lossless} \gls{compressie-algoritme} \gls{deflate}.

\subsubsection{PNG: voordelen}
\label{sec:afbeeldingscompressie-png-voordelen}

Het \gls{png} \gls{afbeeldingsformaat} bied tal van voordelen ten opzichte van zijn voorgangers en \gls{lossy} tegenstanders. Enkele van deze voordelen zijn:

\begin{itemize}
	\item Beschikt over een alpha kanaal waardoor doorzichtigheid meegegeven kan worden als een getal tussen 0 (volledig doorzichtig) en 100 (geen doorzichtigheid).
	
	\item Aanzien als een van de standaarden voor \gls{lossless} \glspl{afbeeldingsformaat} waardoor het een goede support heeft overheen verschillende hardware en software.
	
	\item \Gls{lossless} \gls{afbeeldingsformaat} waardoor er geen kwaliteit verloren gaat.
	
	\item De  \gls{decoder} kan progressief de afbeelding inladen, startend met een weergave tegen lage resolutie tot deze uiteindelijk volledig is ingeladen.
	
	\item Goede uitbreidbaarheid waardoor \gls{meta-data} en andere randvariabelen aan een \gls{png} bestand kunnen toegevoegd worden terwijl het bestand compatibel blijft met oudere versies.
	
	\item Een keuze uit meer dan 16 miljoen kleuren dankzij het \gls{rgb} kleurenprofiel met alpha kanaal. Een groot contrast tegenover \gls{gif} dat maar 256 kleuren ondersteund. 
\end{itemize}

\subsubsection{PNG: nadelen}
\label{sec:afbeeldingscompressie-png-nadelen}

\Gls{png} heeft echter ook enkele minpunten, voornamelijk te weiden aan het feit dat \gls{png} ontwikkeld is om te gebruiken op het internet.

\begin{itemize}
	\item Geen standaard ondersteuning voor geanimeerde beelden.
	
	\item Grote bestandsgrootte door zijn \gls{lossless} eigenschap.
\end{itemize}

\subsection{JPEG}
\label{sec:afbeeldingscompressie-jpeg}

Joint Photographic Experts Group is technisch gezien geen \gls{afbeeldingsformaat} maar een \gls{codec}. Het is een \gls{compressie-algoritme} dat oorspronkelijk op een \gls{lossless} of \gls{lossy} manier te werk kan gaan. De \gls{lossless} variant wordt echter niet grootschalig gebruikt voor \glspl{afbeeldingsformaat} en wordt daarom niet verder toegelicht in deze bachelorproef. Wanneer gesproken wordt over het \gls{jpeg} als \gls{afbeeldingsformaat} verwijst dit meestal naar  \gls{jpeg-jfif} of \gls{jpeg-exif} welke wel een \gls{afbeeldingsformaat} zijn.

\gls{jpeg} wordt doorgaans als \gls{lossy} \gls{compressie-algoritme} gebruikt voor het opslaan van afbeeldingen waarbij een controle over bestandsgrootte en kwaliteit gewenst is. Dit is mogelijk doordat \gls{jpeg} verschillende parameters ondersteund om de werking van het \gls{compressie-algoritme} te beïnvloeden en dus ook het uiteindelijk bestand zijn kwaliteit en bestandsgrootte.

De ontwikkeling van het \gls{jpeg} \gls{compressie-algoritme} is begonnen in 1986 en de \gls{jpeg} standaard is gemaakt in 1992. Deze bestaat uit 7 delen met de laatste officiële revisie in 1994. Uiteraard zijn er tal van uitbreidingen (of extensies zoals deel 3 van de \gls{iso}/IEC 10918 standaard ze benoemd) gemaakt tot op heden. Datums overgenomen van de officiële \gls{jpeg} website (\cite{jpegorg}). 

\gls{jpeg} is ook gekend onder de kortere vorm JPG omdat dit de extensie is die het meest gebruikt wordt voor de \gls{jpeg} \gls{codec}. Dit was omdat in oudere versies van het Windows besturingssysteem, zoals bijvoorbeeld Windows 98, een \gls{extensie} maximaal drie karakters lang mocht zijn. In de huidige versies van Windows is deze beperking echter niet meer actief waardoor de JPG en \gls{jpeg} \glspl{extensie} door elkaar gebruikt kunnen worden.

\subsubsection{JPEG: werking}
\label{sec:afbeeldingscompressie-jpeg-werking}

De werking van \gls{jpeg} en \gls{jpeg-jfif} worden niet diepgaand uitgelegd omdat hierover gehele papers geschreven kunnen worden. De informatie over de werking van  \gls{jpeg} en \gls{jpeg-jfif} is uit het uitstekende boek over \gls{datacompressie}: \citetitle{Salomon2006} (\cite{Salomon2006}) gehaald waar nog dieper op de werking van \gls{jpeg} en \gls{jpeg-jfif} wordt ingegaan.

\Gls{jpeg} is een \gls{compressie-algoritme} en geen volledig \gls{afbeeldingsformaat}. Daarom zijn belangrijke zaken als de aspect ratio en het kleurprofiel niet opgenomen in het \gls{jpeg} \gls{compressie-algoritme} zelf maar in de bijhorende \glspl{afbeeldingsformaat} zoals \gls{jpeg-jfif} en \gls{jpeg-exif}. 

Kenmerkend aan \gls{jpeg} is dat de \gls{lossy} variant enorm veel input mogelijkheden bevat dat controle geven over wat juist verloren mag gaan door het \gls{compressie-algoritme} en in welke mate. Deze input variabelen aanpassen past dus het \gls{compressieratio} van \gls{jpeg} aan.

\Gls{jpeg} is een symmetrisch \gls{compressie-algoritme} wat wilt zeggen dat de \gls{decoder} dezelfde stappen van de \gls{encoder} uitvoert in omgekeerde volgorde. Deze stappen zijn:

\begin{enumerate}
	
	\item Indien de te encoderen afbeelding in kleur is wordt het kleurprofiel aangepast naar een luminantie/ chrominantie kleurprofiel. Dit kleurprofiel past beter bij de perceptie van het menselijke oog. Deze ziet kleine variaties in luminantie echter zeer sterk terwijl variaties in chrominantie veel minder worden waargenomen. \Gls{jpeg} maakt hier gebruik van door veel chrominantie informatie te verkleinen en aan te passen zodat het beter gecomprimeerd kan worden. Deze stap is optioneel maar aangeraden om het beste \gls{compressieratio} met \gls{jpeg} te bereiken.
	
	\item Indien de encoderen afbeelding in kleur is en in het luminantie/ chrominantie kleurprofiel word de chrominantie waarde over meerdere aaneensluitende pixels gedeeld. Overheen hoeveel pixels de chrominantie gedeeld moet worden is een instelbare input variabele.
	
	\item De afbeelding wordt ingedeeld over blokken van 64 pixels (8x8) dat elk apart gecomprimeerd zullen worden. Dit worden data units genoemd. Het is door deze groepering en onafhankelijke \gls{datacompressie} dat bekende \glspl{artefact} als blok\glspl{artefact} voorkomen bij \gls{jpeg}. Een voorbeeld waar deze blok\glspl{artefact} goed zichtbaar zijn is weergegeven in figuur \ref{fig:block-artefact}.
	
	\begin{figure}
		\centering
		\fbox{\includegraphics[width=0.5\linewidth]{img/afbeeldingscompressie/block-artefact.png}}
		\caption{Zwaar gecomprimeerde afbeelding met blok\glspl{artefact}. Afbeelding overgenomen uit \citetitle{blokartefact} (\cite{blokartefact})}
		\label{fig:block-artefact}
	\end{figure}
	
	\item Door het gebruik van \gls{dct} wordt een map van 64 (8x8) frequentie componenten gemaakt. Dit zijn de pixels nu voorgesteld als een getal. In deze stap is reeds enige informatie verloren gegaan door het gebruik van \gls{dct}, een wiskundig algoritme, maar dit verlies aan informatie is niet het gene wat voor het grootte \gls{compressieratio} van \gls{jpeg} zorgt.
	
	\item De 64 frequentie componenten per data unit worden nu gedeeld door 64 te voorziene input variabele: de kwantisatiecoëfficiënten (QCs). Het bekomen resultaat wordt afgerond naar een natuurlijk getal. Het is deze stap waar het \gls{lossy} informatie data gaat snoeien, hoe groter de QCs hoe meer kwaliteitsverlies.
	
	\item De bekomen resultaten worden door een combinatie van \gls{rle-long} en \gls{huffman-coding} gecomprimeerd bijgehouden. Er kan ook gekozen worden om QM coder te gebruiken in plaats van \gls{huffman-coding}.
	
	\item De laatste stap voorziet de door \gls{rle-long} en \gls{huffman-coding} gecomprimeerde data van de nodige \gls{meta-data} en geeft dit finaal pakket terug als output.

\end{enumerate}

Het effectieve afbeeldingsformaat, bijvoorbeeld \gls{jpeg-jfif}, zorgt dan voor de nodige \gls{meta-data} met informatie over de afbeelding zelf. Het is ook die \gls{afbeeldingsformaat} dat de mogelijkheid voor progressief \gls{decoding} voorziet.

\subsubsection{JPEG: voordelen}
\label{sec:afbeeldingscompressie-jpeg-voordelen}

\Gls{jpeg} is één van de meest gebruikte \gls{lossy} \gls{compressie-algoritme} voor afbeeldingen en heeft onder andere daarom enkele belangrijke voordelen zoals:

\begin{itemize}
	\item Uitstekende support overheen verschillende hardware en software.
	
	\item De  \gls{decoder} kan progressief de afbeelding inladen, startend met een weergave tegen lage resolutie tot deze uiteindelijk volledig is ingeladen.
	
	\item Door de mogelijkheid om als \gls{lossy} \gls{compressie-algoritme} te werken kan \gls{jpeg} een enorm kleine bestandsgrootte aannemen afhankelijk van de instellingen.
	
	\item De kleinere bestandsgrootte creëert de mogelijkheid om meer foto's per second te verwerken. Op deze manier kan een digitale camera meer opnames maken in burst wanneer er voor \gls{jpeg-exif} is gekozen als \gls{afbeeldingsformaat} in plaats van een \gls{raw} \gls{afbeeldingsformaat}. 
\end{itemize}

\subsubsection{JPEG: nadelen}
\label{sec:afbeeldingscompressie-jpeg-nadelen}

\Gls{jpeg} heeft uiteraard ook enkele nadelen. De voornaamste zijn:

\begin{itemize}
	\item Door de \gls{lossy} eigenschap aan de hand van clustering kunnen allerlei vormen van \glspl{artefact} voorkomen.
	
	\item Geen mogelijkheid voor doorzichtigheid.
	
	\item Onnatuurlijke afbeeldingen zoals logo's zijn zeer gevoelig aan het verlies van scherpe lijnen en het ontstaan van \glspl{artefact} door scherp contrast.
\end{itemize}

\subsection{JPEG2000}
\label{sec:afbeeldingscompressie-jpeg2000}

\Gls{jpeg2000} is, zoals de naam doet vermoeden, een opvolger van \gls{jpeg} gemaakt door de Joint Photographic Experts Group. Deze waren van mening dat door de opkomst van het internet een betere variant van \gls{jpeg} nodig was.

In maart 1997 kondigde de Joint Photographic Experts Group aan dat ze een nieuwe standaard voor afbeeldingscompressie willen ontwikkelen en open staan voor bijdrages. Dit wekte de interesse van vele scholen en bedrijven. Zo had enkele maanden na de aankondiging de Universiteit van Arizona samen met SAIC reeds een proof of concept voorgesteld op basis van een \gls{wtcq} \gls{compressie-algoritme} in plaats van het \gls{dct} \gls{compressie-algoritme} gebruikt bij \gls{jpeg}.

In augustus 2000 besloot het Joint Photographic Experts Group dat de toen huidige draft versie klaar was om voor te stellen als nieuwe standaard aan de \gls{iso}. In December van 2000 werd dit voorstel goedgekeurd en sinds heden is  \gls{jpeg2000} te vinden onder \gls{iso}/IEC 15444. 

Die \gls{iso} standaard bestaat op het moment van schrijven uit 14 delen, het laatste deel is gepubliceerd in 2013. De meest gebruikte \glspl{afbeeldingsformaat} voor \gls{jpeg2000} zijn die beschreven in deel één en twee. De gebruikte variant van \gls{jpeg2000} voor het onderzoek van hoofdstuk \ref{ch:onderzoek}, \gls{jpf}, is beschreven in het tweede deel 2 (\gls{iso}/IEC 15444-2).

Hoewel \gls{jpeg2000} veel voorkomend is in \gls{videocompressie} en \gls{afbeeldingscompressie} daar waar kwaliteit belangrijk is zoals cinema en medische afbeeldingen, is het nooit de opvolger geworden van \gls{jpeg} die de Joint Photographic Experts Group voor ogen had.


\subsubsection{JPEG2000: werking}
\label{sec:afbeeldingscompressie-jpeg2000-werking}

Doordat \gls{jpeg2000} meerdere revisies kent, elk met hen eigen \glspl{encoder} en \glspl{decoder}, wordt de effectieve werking van het \gls{compressie-algoritme} niet uitgelegd in deze bachelorproef. \Gls{jpeg2000} is een zeer uitgebreid en complex \gls{compressie-algoritme} dat zowel \gls{lossless} als \gls{lossy} te werk kan gaan. De standaard is gedefinieerd in \gls{iso}/IEC 15444.

\Gls{jpeg2000} wordt een schaalbaar \gls{compressie-algoritme} genoemd doordat de \gls{decoding} op verschillende manieren kan gebeuren. Zo kunnen de \glspl{bit} op een andere manier gerangschikt en deels weggelaten worden om een lagere resolutie variant te maken.

Een belangrijk verschil met \gls{jpeg} is dat het geen gebruik meer maakt van de besproken 8x8 data units. Dit zorgt er voor dat de storende blok\glspl{artefact} niet meer voorkomen bij \gls{jpeg2000}. \Gls{jpeg2000} kan wel ring\glspl{artefact} bevatten, deze zijn weergegeven in figuur \ref{fig:ring-artefact}.

\begin{figure}
	\centering
	\fbox{\includegraphics[width=0.5\linewidth]{img/afbeeldingscompressie/ring-artefact.png}}
	\caption{Gecomprimeerde afbeelding op linkerkant met ring\glspl{artefact}, oorspronkelijke afbeelding op de rechterkant. Afbeeldingen van \cite{ringartefact}}
	\label{fig:ring-artefact}
\end{figure}

\gls{jpeg2000} maakt gebruik van \glspl{wavelet} en het quantization \gls{compressie-algoritme}.

\subsubsection{JPEG2000: voordelen}
\label{sec:afbeeldingscompressie-jpeg2000-voordelen}

Aangezien \gls{jpeg2000} door de Joint Photographic Experts Group zelf als opvolger van \gls{jpeg} is benoemd zijn sommige van de voordelen van \gls{jpeg2000} hetzelfde als die van \gls{jpeg}.

De voornaamste voordelen van \gls{jpeg2000} zijn:

\begin{itemize}
	\item In vergelijking met \gls{jpeg} kan \gls{jpeg2000} zowel als \gls{lossless} en \gls{lossy} \gls{compressie-algoritme} gebruikt worden.
	
	\item \gls{jpeg2000} heeft voor de meeste \glspl{use-case} betere kwaliteit dan \gls{jpeg} voor een afbeelding met dezelfde bestandsgrootte. Naarmate de compressieratio stijgt wordt dit voordeel groter. (Zoals aangetoond in studies als \citetitle{jpegvsjpeg2000quality} - \cite{jpegvsjpeg2000quality})
	
	\item Kent meerdere varianten wat voor zeer veel flexibiliteit zorgt.
	
	\item Minder \glspl{artefact} dan \gls{jpeg2000}.
	
	\item De  \gls{decoder} kan progressief de afbeelding inladen, startend met een weergave tegen lage resolutie tot deze uiteindelijk volledig is ingeladen.
	
	\item Ondersteuning voor transparantie. In latere versies niet besproken in deze bachelorproef is ook support voor animatie aanwezig.
	
	\item Schaalbaar en tal van andere voordelen bij het gebruik van \gls{jpeg2000} als \gls{intra-frame} \gls{datacompressie} schema in \gls{videocompressie}. Deze paper gaat niet verder in op het gebruik van \gls{jpeg2000} binnen \gls{videocompressie}.
\end{itemize}

\subsubsection{JPEG2000: nadelen}
\label{sec:afbeeldingscompressie-jpeg2000-nadelen}

Hoewel het plan van \gls{jpeg2000} om de nieuwe standaard te worden enigszins gelukt is binnen \gls{videocompressie} is dit in \gls{afbeeldingscompressie} niet het geval. Dit en nog enkele limitaties van \gls{jpeg2000} zorgen voor onder meer volgende nadelen:

\begin{itemize}
	\item Slechte internetbrowser support: momenteel enkel ondersteund op Safari voor macOS en iOS. 
	
	\item Doordat de verschillende \gls{iso} revisies steeds andere \glspl{extensie} toelichten is er geen achterwaartse comptabiliteit met oude \glspl{decoder}.
	
	\item \Gls{encoding} duurt met de standaard \glspl{encoder} doorgaans langer dan bij \gls{jpeg}. Het \gls{encoding} en \gls{decoding} proces vereist ook meer systeemresources dan \gls{jpeg}.
\end{itemize}

\subsection{WEBP}
\label{sec:afbeeldingscompressie-webp}

\Gls{webp} is een \gls{afbeeldingsformaat} dat door Google is uitgebracht in 2010 dat tot op heden uitgebreid wordt. 
 
\Gls{webp} is een zeer belovend \gls{afbeeldingsformaat} voor het internet. Het ondersteunt een zeer grote variatie van \glspl{use-case}. Het kan \gls{lossless} en \gls{lossy} gebruikt worden. Het presteert in zij \gls{lossless} variant gemiddeld gezien beter dan \gls{png}. De \gls{lossy} variant presteert gemiddeld gezien dan weer beter als \gls{jpeg}.

De prestatie van het \gls{webp} \gls{afbeeldingsformaat} is reeds objectief en subjectief getest door zowel Google zelf als door onafhankelijke partijen. De resultaten zijn daar gelijklopend en steeds in het voordeel \gls{webp}. De objectieve evaluatie tussen \gls{jpeg} en \gls{webp} (\cite{jpegwebp}) alsook dat tussen \gls{png} en \gls{webp} (\cite{pngwebp}) van KeyCDN preekt zowel op vlak van kwaliteit als snelheid in het voordeel van \gls{webp}.

Dit alles is niet alleen te danken aan het budget dat Google heeft om verdere ontwikkeling voor dit \gls{afbeeldingsformaat} te voorzien maar ook door de macht die het bedrijf heeft. Ze laten hun eigen projecten zoals Android, ChromeOS, YouTube, Gmail, de Google Play Store en meer zo veel mogelijk \gls{webp} gebruiken, door dat dit een immens groot deel is van de markt worden concurrenten onrechtstreeks verplicht dit nieuw \gls{afbeeldingsformaat} ook te gebruiken.

\subsubsection{WEBP: werking}
\label{sec:afbeeldingscompressie-webp-werking}

De werking van de \gls{webp} wordt in dit deel beknopt uitgelegd aan de hand van de officiële Google documentatie\urlcite{webpdocs}. 

De \gls{lossy} variant van \gls{webp} is een \gls{afbeeldingsformaat} dat zoals vele nieuwe generatie \glspl{afbeeldingsformaat} ontstaan is uit een \gls{intra-frame} \gls{compressie-algoritme} voor \gls{videocompressie}in dit geval \gls{vp8}, een \gls{compressie-algoritme} dat in 2008 door Google ontwikkeld is. Het maakt is gebaseerd op het block prediction \gls{compressie-algoritme}.

Het \gls{compressie-algoritme} voor de \gls{lossy} variant deel de afbeelding dus op in verschillende segmenten die macroblocks genoemd worden. Een macroblock word bijgehouden door de verschillen met de voorgaande macroblock bij te houden, zoals \gls{png} dat doet voor pixels. Dit wordt predictive coding genoemd. De resulterende data is dan net zoals \gls{jpeg} omgezet via \gls{dct} wat resulteert in een resultaat met veel nul waardes. Tot deze stap werkt het \gls{compressie-algoritme} volledig \gls{lossless}. Na deze stap wordt het resultaat \gls{lossy} comprimeerd aan de hand van Quantization. Dat resultaat wordt vervolgens entropy-coded.

De \gls{lossless} \gls{webp} variant gebruikt een \gls{compressie-algoritme} dat volledig door Google geschreven is. Ze gebruiken hier onder andere een variatie van \gls{huffman-coding} voor.

\subsubsection{WEBP: voordelen}
\label{sec:afbeeldingscompressie-webp-voordelen}

Het gebruik van \gls{webp} bied vele voordelen, enkele van de voornaamste zijn:

\begin{itemize}
	\item In vergelijking met andere 'nieuwe generatie' \glspl{afbeeldingsformaat} heeft \gls{webp} een goede internetbrowser support. Volgens caniuse.com is \gls{webp} ondersteund in elke grote internetbrowser buiten Safari.
	
	\item Zowel de \gls{lossless} als \gls{lossy} variant presteren gemiddeld gezien beter dan respectievelijk \gls{png} en \gls{jpeg}. (Zie \ref{sec:afbeeldingscompressie-webp})
	
	\item Ondersteund transparantie en animatie.
\end{itemize}

\subsubsection{WEBP: nadelen}
\label{sec:afbeeldingscompressie-webp-nadelen}

\Gls{webp} heeft vooral ondersteuning gerelateerde problemen. De voornaamste zijnde:

\begin{itemize}
	\item Geen support in Safari en enkele kleinere of gedateerde internetbrowsers: Internet Explorer, KaiOS en Blackbarry Browser.
	
	\item Geen progressieve \gls{decoding} mogelijk.
	
	\item Geen standaard support in \gls{ps} maar gratis \glspl{plug-in} beschikbaar. In het onderzoek uit hoofdstuk \ref{ch:onderzoek} wordt de voor het onderzoek gebruikte \gls{plug-in} toegelicht.
\end{itemize}

\subsection{HEIF/HEIC}
\label{sec:afbeeldingscompressie-heif}

High Efficiency Image Format is in een \gls{codec} ontwikkeld door de Moving Picture Experts Group dat het zelf benoemd als \gls{afbeeldingsformaat}. Het is herkend als standaard in het twaalfde deel van \gls{iso} 23008.

Het voornaamste gebruik van \gls{heif} als \gls{afbeeldingsformaat} is bij \glspl{intra-frame} bij \gls{h265} \gls{videocompressie}.

Je kan het aanzien als een heel geavanceerde vorm van \gls{jpeg}. Er is ondersteuning voor blokken van 64 x 64 \glspl{pixel} voor compressie in plaats van de 8 x 8 bij \gls{jpeg}. Het maakt gebruikt van arithmetische codering (CABAC) en geavanceerde voorspellingen... Door de complexiteit van \gls{heif} kunnen alleenstaande papers geschreven worden omtrent de werking. Deze werking beschrijven valt dus niet in de scope van deze bachelorproef.

Met de opkomst iOS 11 in 2017 was Apple echter het eerste bedrijf dat \gls{heif} grootschalige gebruikte voor opslag van afbeeldingen. Alle videobestanden waren sindsdien intern opgeslagen met de \gls{h265} \gls{codec} en alle afbeeldingen als \gls{heif} met de de \gls{heic} \gls{extensie}. 

De overstap van \gls{jpeg} naar het nieuwe generatie \gls{afbeeldingsformaat} \gls{heif} was voor Apple interessant op verschillende vlakken. De voornaamste waren kwaliteits- en snelheidswinst. Dit resulteerde dan ook in een kleinere bestandsgrootte waardoor er meer afbeeldingen opgeslagen kunnen worden op een iOS toestel.

Ook de ondersteuning voor burst foto's en live foto's kwam deze keuze ten voordele, het is namelijk mogelijk meerdere afbeeldingen op te slaan onder één \gls{heif} bestand.

\subsubsection{HEIF: voordelen}
\label{sec:afbeeldingscompressie-heif-voordelen}

\Gls{heif} en het door Apple gebruikte \gls{heic} bied als nieuwe generatie \gls{afbeeldingsformaat} tal van voordelen. De voornaamste zijn: 

\begin{itemize}
	\item Zowel \gls{lossless} als \gls{lossy} operatie mogelijk.
	
	\item Photoshop CC ondersteuning sinds het einde van 2018. Deze bestanden worden aanzien als \gls{raw}.
	
	\item Een uitgebreid assortiment aan functionaliteit zoals het opslaan van meerdere afbeeldingen in één bestand. Deze functionaliteiten zijn interessant voor burst foto's, HDR foto's,...
	
	\item Ondersteuning voor transparantie en animatie.
	
	\item \Gls{heif} bied ook tal van voordelen bij het gebruik in \gls{videocompressie} die niet verder in deze bachelorproef besproken worden.
\end{itemize}

\subsubsection{HEIF: nadelen}
\label{sec:afbeeldingscompressie-heif-nadelen}

Net zoals \gls{webp} heeft \gls{heif} één groot nadeel: ondersteuning. Apple lost dit op door bij het delen van een foto de afbeelding te converteren naar \gls{jpeg} maar daarmee gaan ook alle voordelen van \gls{heif} verloren... 

Desondanks Apple gebruiker is van \gls{heif}/\gls{heic} is er nog geen ondersteuning voor in de Safari internetbrowser. Ook andere gekende internetbrowsers bieden geen ondersteuning voor dit \gls{afbeeldingsformaat}.

Ook is er geen progressieve \gls{decoding} mogelijk.

\section{De juiste keuze}
\label{sec:afbeeldingscompressie-keuze}

De juiste keuze van \gls{afbeeldingsformaat} maken is geen eenvoudige taak en zeer \gls{use-case} gebonden. Hoewel nieuwe \glspl{afbeeldingsformaat} tal van interessante voordelen bieden is vooral compatibiliteit een wederkerend probleem. Hier bestaan relatief eenvoudige oplossingen voor in webomgevingen waarvan enkele besproken zijn in deel \ref{sec:afbeeldingscompressie-implementatie-on-premise}. De keuze voor een nieuw \gls{afbeeldingsformaat} bij \gls{on-premise} applicaties is iets lastiger om te implementeren en word kort besproken in deel \ref{sec:afbeeldingscompressie-implementatie-on-premise}. Indien de \gls{use-case} professionele drukwerk omvat is een keuze voor een \gls{raster} \gls{afbeeldingsformaat} ten sterkste aangeraden. Deze \glspl{afbeeldingsformaat} zijn niet verder besproken in deze bachelorproef.

\subsection{Functievereisten}
\label{sec:afbeeldingscompressie-functievereisten}

Het is belangrijk om voor elke \gls{use-case} grondig na te denken welke \glspl{afbeeldingsformaat} de beste keuzes zijn. De selectie verfijnen kan je reeds eenvoudig doen door naar enkele functievereisten als ondersteuning voor transparantie te kijken. 

Een korte overzichtstabel van enkele kernfunctionaliteiten per \gls{afbeeldingsformaat} is te vinden in figuur \ref{fig:overzichtstabel-afbeeldingsformaten-functies}. Hier zijn enkele opmerkingen bij:

\begin{itemize}
	\item \gls{jpeg2000} ondersteunt pas sinds \gls{iso} 15444 deel drie animatie onder de vorm van Motion \gls{jpeg2000} (.mj2). In deze bachelorproef is echter de meest voorkomende versie .jpf uit deel twee besproken. Deze ondersteund geen animatie.
	
	\item \Gls{webp} ondersteunt geen progressieve \gls{decoding} maar wel incrementele \gls{decoding}. Dit zorgt er voor dat het wel mogelijk is reeds 'iets' weer te geven terwijl \gls{decoding} (en zelfs download) nog gaande is.
\end{itemize}

\begin{table}[]
	\begin{tabular}{|l|l|l|l|l|l|}
		\hline
		& \textbf{PNG}                   & \textbf{JPEG}                  & \textbf{JPEG200}               & \textbf{WebP}                  & \textbf{HEIF}               \\ \hline
		\textbf{Lossless}            & \cellcolor[HTML]{32CB00}Ja     & \cellcolor[HTML]{9B9B9B}Deels* & \cellcolor[HTML]{32CB00}Ja     & \cellcolor[HTML]{32CB00}Ja     & \cellcolor[HTML]{32CB00}Ja  \\ \hline
		\textbf{lossy}               & \cellcolor[HTML]{CB0000}Nee    & \cellcolor[HTML]{32CB00}Ja     & \cellcolor[HTML]{32CB00}Ja     & \cellcolor[HTML]{32CB00}Ja     & \cellcolor[HTML]{32CB00}Ja  \\ \hline
		\textbf{Transparantie}       & \cellcolor[HTML]{32CB00}Ja     & \cellcolor[HTML]{CB0000}Nee    & \cellcolor[HTML]{32CB00}Ja     & \cellcolor[HTML]{32CB00}Ja     & \cellcolor[HTML]{32CB00}Ja  \\ \hline
		\textbf{Animatie}            & \cellcolor[HTML]{CB0000}Nee  & \cellcolor[HTML]{CB0000}Nee    & \cellcolor[HTML]{9B9B9B}Deels* & \cellcolor[HTML]{32CB00}Ja     & \cellcolor[HTML]{32CB00}Ja  \\ \hline
		\textbf{Progressief decoden} & \cellcolor[HTML]{32CB00}Ja     & \cellcolor[HTML]{32CB00}Ja     & \cellcolor[HTML]{32CB00}Ja     & \cellcolor[HTML]{9B9B9B}Deels* & \cellcolor[HTML]{CB0000}Nee \\ \hline
	\end{tabular}
	\caption{Overzichtstabel van enkele kernfunctionaliteiten per \gls{afbeeldingsformaat}. Dit wordt verder besproken in deel \ref{sec:afbeeldingscompressie-functievereisten}.}
	\label{fig:overzichtstabel-afbeeldingsformaten-functies}
\end{table}

\subsection{Ondersteuning}
\label{sec:afbeeldingscompressie-ondersteuning}

De algemene ondersteuning voor \gls{jpeg} en \gls{png} is zeer uitgebreid. Alle recente internetbrowsers en besturingssystemen kunnen er perfect met om. Dit is één van de grootste redenen waarom deze oudere \gls{afbeeldingsformaat} nog zo dominant aanwezig zijn. Nieuwe \glspl{afbeeldingsformaat} falen namelijk vaak doordat er een slechte ondersteuning is en niet door een slecht \gls{compressie-algoritme}.

Zo heeft \gls{heic} geen browserondersteuning tot op heden. \Gls{jpeg2000} heeft een hele slechte internetbrowser ondersteuning met enkel ondersteuning in Safari onder de gekende browsers. \Gls{webp} heeft een aanvaardbare internetbrowser ondersteuning met momenteel enkel geen ondersteuning in Safari onder de gekende internetbrowser.

Een compleet overzicht is beschikbaar in figuur \ref{fig:overzichtstabel-afbeeldingsformaten-support}

\begin{table}[]
	\begin{tabular}{|l|l|l|l|l|l|}
		\hline
		& \textbf{PNG}                                      & \textbf{JPEG}                                     & \textbf{JPEG200}                                  & \textbf{WebP}                                     & \textbf{HEIF}               \\ \hline
		\textbf{Chrome}             & \cellcolor[HTML]{32CB00}{\color[HTML]{333333} Ja} & \cellcolor[HTML]{32CB00}{\color[HTML]{333333} Ja} & \cellcolor[HTML]{CB0000}Nee                       & \cellcolor[HTML]{32CB00}{\color[HTML]{333333} Ja} & \cellcolor[HTML]{CB0000}Nee \\ \hline
		\textbf{Edge}               & \cellcolor[HTML]{32CB00}{\color[HTML]{333333} Ja} & \cellcolor[HTML]{32CB00}{\color[HTML]{333333} Ja} & \cellcolor[HTML]{CB0000}Nee                       & \cellcolor[HTML]{32CB00}{\color[HTML]{333333} Ja} & \cellcolor[HTML]{CB0000}Nee \\ \hline
		\textbf{Firefox}            & \cellcolor[HTML]{32CB00}{\color[HTML]{333333} Ja} & \cellcolor[HTML]{32CB00}{\color[HTML]{333333} Ja} & \cellcolor[HTML]{CB0000}Nee                       & \cellcolor[HTML]{32CB00}{\color[HTML]{333333} Ja} & \cellcolor[HTML]{CB0000}Nee \\ \hline
		\textbf{Safari}             & \cellcolor[HTML]{32CB00}{\color[HTML]{333333} Ja} & \cellcolor[HTML]{32CB00}{\color[HTML]{333333} Ja} & \cellcolor[HTML]{32CB00}{\color[HTML]{333333} Ja} & \cellcolor[HTML]{CB0000}Nee                       & \cellcolor[HTML]{CB0000}Nee \\ \hline
		\textbf{Internet explorer}  & \cellcolor[HTML]{32CB00}{\color[HTML]{333333} Ja} & \cellcolor[HTML]{32CB00}{\color[HTML]{333333} Ja} & \cellcolor[HTML]{CB0000}Nee                       & \cellcolor[HTML]{CB0000}Nee                       & \cellcolor[HTML]{CB0000}Nee \\ \hline
		\textbf{Opera}              & \cellcolor[HTML]{32CB00}{\color[HTML]{333333} Ja} & \cellcolor[HTML]{32CB00}{\color[HTML]{333333} Ja} & \cellcolor[HTML]{CB0000}Nee                       & \cellcolor[HTML]{32CB00}{\color[HTML]{333333} Ja} & \cellcolor[HTML]{CB0000}Nee \\ \hline
		\textbf{Blackberry browser} & \cellcolor[HTML]{32CB00}{\color[HTML]{333333} Ja} & \cellcolor[HTML]{32CB00}{\color[HTML]{333333} Ja} & \cellcolor[HTML]{CB0000}Nee                       & \cellcolor[HTML]{CB0000}Nee                       & \cellcolor[HTML]{CB0000}Nee \\ \hline
		\textbf{KaiOS browser}      & \cellcolor[HTML]{32CB00}{\color[HTML]{333333} Ja} & \cellcolor[HTML]{32CB00}{\color[HTML]{333333} Ja} & \cellcolor[HTML]{CB0000}Nee                       & \cellcolor[HTML]{CB0000}Nee                       & \cellcolor[HTML]{CB0000}Nee \\ \hline
	\end{tabular}
	\caption{Overzichtstabel van internetbrowser ondersteuning per afbeeldingsformaat. Data verkregen van caniuse.com in mei 2019.}
	\label{fig:overzichtstabel-afbeeldingsformaten-support}
\end{table}

\section{Implementatiemogelijkheden voor nieuwe afbeeldingsformaten}
\label{sec:afbeeldingscompressie-implementatie}

Het gebruik van nieuwe \glspl{afbeeldingsformaat} kan afgeschrikt worden wanneer je leest dat de ondersteuning nog niet optimaal is. Er zijn echter tal van manieren om toch de juiste afbeelding weer te geven wanneer er geen ondersteuning beschikbaar is. Dit wordt in onderstaande delen kort besproken voor zowel implementatie in webomgevingen als \gls{on-premise} omgevingen.

\subsection{Webomgeving}
\label{sec:afbeeldingscompressie-implementatie-web}

Binnen webomgevingen zijn tal van manieren om een alternatieve afbeelding op te geven als terugval afbeelding moest het laden van een afbeelding mislukken. Dit zorgt er voor dat een internetbrowser die het gekozen \gls{afbeeldingsformaat} niet ondersteund de terugval afbeelding weergeeft. Deze is doorgaans dan een \gls{png} of \gls{jpeg}. Dit kan op verschillende manieren bereken worden.

\subsection{Manueel}
\label{sec:afbeeldingscompressie-implementatie-web-manueel}

Een éénvoudige en universeel werkende manieren om een terugval afbeelding op te geven is door gebruik te maken van een picture tag in HTML. Deze ziet er als volgt uit:

\begin{lstlisting}[style=htmlcssjs]
<picture>
	<source srcset="pad/naar/afbeelding.webp" type="image/webp">
	<source srcset="pad/naar/afbeelding.heic" type="image/heic">
	<source srcset="pad/naar/afbeelding.jpf" type="image/jpx">
	<source srcset="pad/naar/afbeelding.png" type="image/png"> 
	<source srcset="pad/naar/afbeelding.jpg" type="image/jpeg"> 
	<img src="pad/naar/afbeelding.jpg">
</picture>
\end{lstlisting}

De volgorde is hier enorm van belang. Internetbrowsers die de picture tag niet ondersteunen negeren de source tags ook en herkennen enkel de img tag en geven alsvolgt de afbeelding binnen die tag ingesteld weer. Indien een internetbrowser de picture tag wel ondersteund zal hij het type uit de eerste sourceset die hij ondersteund nemen als bron voor de afbeelding. De internetbrowser werkt hier van boven naar onder en stopt bij een match.

\subsection{geautomatiseerd}
\label{sec:afbeeldingscompressie-implementatie-web-automated}

Er zijn tal van mogelijkheden om op een geautomatiseerde manier gebruik te maken van nieuwe \glspl{afbeeldingsformaat}. Vele caching diensten, zoals Cloudflare, bieden de mogelijkheid automatisch elke afbeelding te converteren naar verschillende afbeeldingsformaten en diegene weer te geven met de kleinst mogelijke bestandsgrootte. 

Voor \gls{wordpress} en tal van andere \glspl{cms} zijn er ook \glspl{plug-in} voorzien die op een zelfde manier te werk gaan. Voor \gls{wordpress} is er bijvoorbeeld de WebP Express \urlcite{webpwordpress} \gls{plug-in}.

Sommige automatisaties kiezen er voor de afbeelding direct op te slaan in alle \glspl{afbeeldingsformaat} wat de opslagruimte te min gaat maar response time te goeie doet. Andere houden enkele het bronbestand bij en genereren het gevraagde formaat bij aanvraag. Dit gaat dan weer te min van de response time en het cpu gebruik van de \gls{hosting} server. Dit is uiteraard ook manueel te implementeren.

Voor de meeste \glspl{afbeeldingsformaat} zijn ook \gls{js} \glspl{decoder} beschikbaar. Dit is \gls{js} code dat afbeeldingen van een bepaald \gls{afbeeldingsformaat} omzet naar een ander \gls{afbeeldingsformaat}. Deze conversie gebeurd op het toestel van de bezoeker en vereist dus geen systeemresources van de \gls{hosting} server. De conversie gebeurt meestal naar een \gls{lossless} \gls{afbeeldingsformaat} zoals \gls{png} omdat op die manier geen kwaliteit verloren gaat en \gls{png} een goede internetbrowser ondersteuning heeft. Deze conversie kan altijd gebeuren of wanneer de bezoeker zijn internetbrowser het normale \gls{afbeeldingsformaat} niet ondersteund. Een voorbeeld van een \gls{js} \glspl{decoder} voor het \gls{webp} \gls{afbeeldingsformaat} naar \gls{png} om te zetten is WebPJS van Dominik Homberger\urlcite{webpjsdecoder}.

\subsection{On-premise omgeving}
\label{sec:afbeeldingscompressie-implementatie-on-premise}

Bij een \gls{on-premise} omgeving zijn er ook verschillende mogelijkheden om een terugval afbeelding in te stellen. In een situatie als die van Apple (besproken in \ref{sec:afbeeldingscompressie-heif-nadelen}) kan gekozen worden om een converter in te bouwen die het \gls{afbeeldingsformaat} omzet naar een wel ondersteund formaat bij het bekijken of delen van een afbeelding. Dit kan echter heel intensief voor de CPU worden.

Een andere oplossing is een encoder en of decoder te voorzien in de installatie die de ondersteuning voor het \gls{afbeeldingsformaat} levert. Dit bestaat in het geval van \gls{webp} voor zowel macOS, Windows als bepaalde Linux distributies.

Er kan uiteraard ook gekozen worden om de terugval afbeelding effectief mee op te slaan op het toestel. Er kan dan gebruik gemaakt worden van een simpele universeel aanspreekbare controle dat het juiste afbeeldingsformaat selecteert. Deze controle kan ook tijdens de installatie gedaan worden en zo enkel de afbeeldingen in het nodige \gls{afbeeldingsformaat} over te zetten naar het toestel maar dit kan voor problemen zorgen wanneer de software op het systeem update en de ondersteuningen verander.
\chapter{Videocompressie}
\label{ch:videocompressie}

TODO

%TODO: videocompressie
%deel 4
\chapter{Kwaliteit beoordelen}
\label{ch:kwaliteit}

TODO

%TODO: kwaliteit
\chapter{Onderzoek}
\label{ch:onderzoek}

TODO

%TODO: onderzoek

\section{Waarom een subjectief onderzoek}
\label{sec:onderzoek-waarom-subjectief}

TODO
%TODO: onderzoek

\section{Use case}
\label{sec:onderzoek-use-case}

TODO
%TODO: onderzoek

TODO
%TODO: onderzoek

\section{Uitvoering}
\label{sec:onderzoek-uitvoering}

TODO
%TODO: onderzoek

\subsection{Geëvalueerde afbeeldingsformaten}
\label{sec:onderzoek-afbeeldingsformaten}

TODO
%TODO: onderzoek

\subsection{Evaluatietool}
\label{sec:onderzoek-evaluatietool}

TODO
%TODO: onderzoek

\subsection{Deelnemers}
\label{sec:onderzoek-deelnemers}

TODO
%TODO: onderzoek

\section{Resultaten}
\label{sec:onderzoek-resultaten}

TODO
%TODO: onderzoek

\section{Besluit}
\label{sec:onderzoek-besluit}

TODO
%TODO: onderzoek
%deel 5
\chapter{Huidige en toekomstige uitdagingen}
\label{ch:uitdagingen}

%TODO: uitdagingen
%%=============================================================================
%% Conclusie
%%=============================================================================

\chapter{Conclusie}
\label{ch:conclusie}

\Gls{datacompressie}, en compressie in het algemeen is niets nieuw. Integendeel, het is één van de oudste concepten binnen IT en tot op heden van fundamenteel belang voor zowat alle IT-toepassingen. Een basiskennis over \gls{datacompressie} en de belangrijkste \glspl{afbeeldingsformaat} en video \gls{codec} is dan ook geen luxe binnen de IT-wereld. Veel vakgerelateerde opleidingen, zoals de opleiding Toegepaste Informatica te HoGent, voorzien echter geen lessen rond \gls{datacompressie}, waardoor deze basiskennis voor velen onbestaande is.

Deze bachelorproef biedt een oplossing voor dat probleem. Het vormt een gegronde basiskennis over \gls{datacompressie} zonder onnodig complex te zijn, wat het geschikt maakt voor de grote variatie van belanghebbenden. Vanaf het voorstel waren de doelstellingen van deze bachelorproef, een antwoord bieden op zeven onderzoeksvragen en één hoofdonderzoeksvraag. Deze onderzoeksvragen worden hieronder nog eens kort aangegaan met een terugblik naar de kennis verworven in deze bachelorproef.

\subsection*{Hoe is datacompressie binnen IT ontstaan?}
\label{sec:conclussie-onderzoeksvraag-1}

Vele onderzoekers zijn het erover eens dat \gls{datacompressie} dateert van voor de uitvinding van de computer. Zo kan morsecode gezien worden als een vorm van \gls{datacompressie}. Morsecode is uitgevonden voor het computertijdperk, in 1832, door Samuel F.B. Morse. Het kan gezien worden als een vorm van datacompressie doordat veelvoorkomende letters een kortere audiotoon kregen dan minder gebruikte letters (\cite{morsecode}).

\subsection*{Wat waren enkele van de eerste compressie-algoritmen?}
\label{sec:conclussie-onderzoeksvraag-2}

Enkele van de eerste \glspl{compressie-algoritme} komen aan bod in deel \ref{sec:ontstaan-datacompressie-primitieve-technieken-binnen-it} van deze bachelorproef. Twee belangrijke \glspl{compressie-algoritme} die al meer dan vijftig jaar bestaan, maar tot heden de basis vormen voor vele toepassingen binnen \gls{datacompressie} zijn \gls{rle-long} en \gls{huffman-coding}. De werking van deze \glspl{compressie-algoritme} is dan ook uitgebreid aan bod gekomen in deze bachelorproef. Een theoretische uitleg met een eenvoudig voorbeeld is voorzien in deel \ref{sec:primitieve-technieken-voorbeeld}. In de proof of concept \gls{compressietool} gemaakt voor deze bachelorproef zijn het ook deze twee \glspl{compressie-algoritme} die gebruikt worden. Deze \gls{compressietool} is verder toegelicht in hoofdstuk \ref{ch:compressietool}.

\subsection*{Waar zitten de verschillen tussen de afbeeldingsformaten en video codecs?}
\label{sec:conclussie-onderzoeksvraag-3}

\Glspl{afbeeldingsformaat} en video \glspl{codec} hebben meer gemeen dan oorspronkelijk gedacht zou worden. Vele \glspl{afbeeldingsformaat} vormen de basis voor een goed presterende video \glspl{codec} en de besproken \glspl{afbeeldingsformaat} \gls{webp} en \gls{heic} vinden juist hun ontstaan in \gls{videocompressie}. De onderlinge verschillen tussen de verschillende besproken \glspl{afbeeldingsformaat} en video \gls{codec} is af te leiden uit de voordelen en nadelen te vinden in hoofdstuk \ref{ch:afbeeldingscompressie} en \ref{ch:videocompressie}. De delen over het maken van een juiste keuze van \gls{afbeeldingsformaat} (deel \ref{sec:afbeeldingscompressie-keuze}) en video \gls{codec} (deel \ref{sec:videocompressie-keuze}) bieden aan de hand van enkele overzichten ook een duidelijk antwoord op deze vraag.

\subsection*{Hoe kan datacompressie correct geïmplementeerd worden?}
\label{sec:conclussie-onderzoeksvraag-4}

Hoe \gls{datacompressie} correct geïmplementeerd kan worden, is terug te vinden in verschillende delen van deze bachelorproef. De \gls{compressietool} en achterliggende code wordt uitgebreid besproken in hoofdstuk \ref{ch:compressietool}. Deze is \gls{open-source} ter beschikking gesteld op \gls{github} en kan zonder enige licenties gebruikt en aangepast worden. Er worden ook enkele beperkingen met deze \gls{compressietool} toegelicht en mogelijke oplossingen wat een geïnteresseerde lezer kan aanzetten deze beperkingen zelf weg te werken. Er wordt ook toegelicht hoe nieuwe generatie \glspl{afbeeldingsformaat} geïmplementeerd kunnen worden met ondersteuning voor alle internetbrowsers in gedachten. Dit is verder toegelicht in deel \ref{sec:afbeeldingscompressie-implementatie}.

\subsection*{Wat is het verschil tussen de afbeeldingsformaten: PNG, JPEG, JPEG2000, WebP en HEIF?}
\label{sec:onderzoeksvraag-5}

Elk \gls{afbeeldingsformaat} wordt toegelicht in hoofdstuk \ref{ch:afbeeldingscompressie}. Hier wordt zowel het ontstaan, de werking en voordelen en nadelen van de verschillende \glspl{afbeeldingsformaat} uitgelegd. Dit biedt samen met de resultaten van het onderzoek besproken in deel \ref{sec:onderzoek-resultaten} en \ref{sec:onderzoek-besluit} een uitgebreid inzicht in de verschillen tussen deze \glspl{afbeeldingsformaat}.

\subsection*{Wat is het verschil tussen de video codecs: H.264/AVC, H.264/SVC, H.265/HEVC en AV1?}
\label{sec:conclussie-onderzoeksvraag-6}

Elke video \glspl{codec} wordt toegelicht in hoofdstuk \ref{ch:videocompressie}. Hier wordt zowel het ontstaan als de voordelen en nadelen van de verschillende video \glspl{codec} aangekaart. Zoals in de overzichten van deel \ref{sec:videocompressie-keuze} duidelijk is weergegeven is er binnen \gls{videocompressie} een enorm probleem van complexe licenties. Het is ook daarom dat Google samenwerkt met tal van andere grote bedrijven als Mozilla en Microsoft en zo \gls{av1} op de markt heeft gebracht. Deze veelbelovende video \gls{codec} wordt ook in hoofdstuk \ref{ch:videocompressie} uitgebreid besproken.

\subsection*{Wat is DNA compressie en wat zijn andere uitdagingen binnen datacompressie?}
\label{sec:onderzoeksvraag-7}

Als afsluitend hoofdstuk (\ref{ch:uitdagingen}) is een korte vermelding van enkele huidige uitdagingen binnen \gls{datacompressie} toegelicht. Dit is bewust zeer beknopt gehouden zodat de lezer warm gemaakt wordt verder opzoekingswerk naar de interessante wereld van \gls{datacompressie} te verrichten!

\subsection{Hoofdonderzoeksvraag}
\label{sec:conclussie-hoofdonderzoeksvraag}

Door het beantwoorden van alle sub onderzoeksvragen kan de hoofdonderzoeksvraag, 'waarom moet er stilgestaan worden bij het gebruiken van \glspl{compressie-algoritme}, hoe kies je een geschikt \gls{compressie-algoritme} voor een bepaalde \gls{use-case} en hoe implementeer je dit best', door de lezer zelf beantwoord worden. Deze bachelorproef bevat namelijk alle nodige informatie om op een gegronde manier op zoek te gaan naar een \gls{compressie-algoritme} voor een bepaalde \gls{use-case}.

\subsection{Mogelijke uitbreidingen}
\label{sec:conclussie-uitbreidingen}

Desondanks deze bachelorproef reeds uit meer dan honderd pagina's bestaat, is er nog altijd ruimte voor uitbreidingen. Zo kan het onderzoek herwerkt worden zodat de verschillende afbeeldingen een gelijke bestandsgrootte hebben, wat een conclusie trekken makkelijker zal maken. Deze uitbreiding is relatief eenvoudig te voorzien doordat \gls{afbeeldingsevaluatietool}, die gemaakt is voor deze bachelorproef, \gls{open-source} is en gratis gebruikt mag worden.

Maar ook uitbreidingen op de gemaakte proof of concept \gls{compressietool} zijn mogelijk. Denk hierbij aan het combineren van \gls{rle-long} en \gls{huffman-coding} of het implementeren van een compleet nieuw \gls{compressie-algoritme}.




%==================
%% Bijlagen
%==================

\chapter{Bijlages}
\label{ch:bijlages}

Om de leesbaarheid van de bachelorproef te behouden zijn sommige niet essentiële afbeeldingen, tabellen en overige documenten in dit hoofdstuk opgenomen. 

\section{Figuren literatuurstudie}
\label{sec:bijlages-literatuurstudie}

\FloatBarrier
\begin{figure}[h!]
	\centering
	\fbox{\includegraphics[width=0.45\linewidth]{img/literatuurstudie/lossles_datacompressie_overzicht.png}}
	\caption{Lossless datacompressie overzicht (\cite{ethwcompressionhistory})}
	\label{fig:lossles-datacompressie-overzicht}
\end{figure}
\FloatBarrier

\section{Screenshots datacompressietool}
\label{sec:bijlages-screenshot-datacompressietool}

De \gls{compressietool} is raadpleegbaar via de website van Lennert Bontinck\urlcite{compressietool}. De code is te vinden op de \gls{github} repository van deze bachelorproef\urlcite{githubbachelorproef}.

\FloatBarrier
\begin{figure}[h!]
	\fbox{\includegraphics[width=\linewidth]{img/bijlages/compressietool/index.png}}
	\caption{Verwelkomingsscherm van de \gls{afbeeldingsevaluatietool}.}
	\label{fig:bijlages-screenshot-datacompressietool-index}
\end{figure}
\FloatBarrier

\FloatBarrier
\begin{figure}[h!]
	\fbox{\includegraphics[width=\linewidth]{img/bijlages/compressietool/encoded.png}}
	\caption{Scherm verkregen door een bestand te encoderen \gls{afbeeldingsevaluatietool}.}
	\label{fig:bijlages-screenshot-datacompressietool-encoded}
\end{figure}
\FloatBarrier

\FloatBarrier
\begin{figure}[h!]
	\fbox{\includegraphics[width=\linewidth]{img/bijlages/compressietool/decoded.png}}
	\caption{Scherm verkregen door een bestand te decoderen \gls{afbeeldingsevaluatietool}.}
	\label{fig:bijlages-screenshot-datacompressietool-decoded}
\end{figure}
\FloatBarrier


\section{Screenshots afbeeldingsevaluatietool}
\label{sec:bijlages-screenshot-afbeeldingsevaluatietool}

De \gls{afbeeldingsevaluatietool} is raadpleegbaar via de website van Lennert Bontinck\urlcite{evaluatietool}. De code is te vinden op de \gls{github} repository van deze bachelorproef\urlcite{githubbachelorproef}.

\FloatBarrier
\begin{figure}[h!]
	\fbox{\includegraphics[width=\linewidth]{img/bijlages/afbeeldingsevaluatietool/setup.png}}
	\caption{Setup scherm van de \gls{afbeeldingsevaluatietool}.}
	\label{fig:bijlages-screenshot-afbeeldingsevaluatietool-setup}
\end{figure}
\FloatBarrier

\FloatBarrier
\begin{figure}[h!]
	\fbox{\includegraphics[width=\linewidth]{img/bijlages/afbeeldingsevaluatietool/welkom.png}}
	\caption{Verwelkomingsscherm van de \gls{afbeeldingsevaluatietool}.}
	\label{fig:bijlages-screenshot-afbeeldingsevaluatietool-welkom}
\end{figure}
\FloatBarrier

\FloatBarrier
\begin{figure}[h!]
	\fbox{\includegraphics[width=\linewidth]{img/bijlages/afbeeldingsevaluatietool/video.png}}
	\caption{Introductievideo met uitleg van de \gls{afbeeldingsevaluatietool}.}
	\label{fig:bijlages-screenshot-afbeeldingsevaluatietool-video}
\end{figure}
\FloatBarrier

\FloatBarrier
\begin{figure}[h!]
	\fbox{\includegraphics[width=\linewidth]{img/bijlages/afbeeldingsevaluatietool/over-u.png}}
	\caption{Informatie over de deelnemer in de \gls{afbeeldingsevaluatietool}.}
	\label{fig:bijlages-screenshot-afbeeldingsevaluatietool-over-u}
\end{figure}
\FloatBarrier

\FloatBarrier
\begin{figure}[h!]
	\fbox{\includegraphics[width=\linewidth]{img/bijlages/afbeeldingsevaluatietool/evaluatie.png}}
	\caption{Afbeelding beoordelen in de \gls{afbeeldingsevaluatietool} met 5x zoom mogelijkheid.}
	\label{fig:bijlages-screenshot-afbeeldingsevaluatietool-evalutie}
\end{figure}
\FloatBarrier

\FloatBarrier
\begin{figure}[h!]
	\fbox{\includegraphics[width=\linewidth]{img/bijlages/afbeeldingsevaluatietool/export.png}}
	\caption{Het scherm waar de resultaat gedownload kunnen worden.}
	\label{fig:bijlages-screenshot-afbeeldingsevaluatietool-export}
\end{figure}
\FloatBarrier

\section{Extra documenten onderzoek}
\label{sec:bijlages-onderzoek}

\FloatBarrier
\begin{figure}[h!]
	\centering
	\fbox{\includegraphics[width=0.5\linewidth]{img/bijlages/onderzoek/renderopties.png}}
	\caption{De renderopties ingesteld in \gls{ps} voor alle vijftien afbeeldingen en vier \glspl{afbeeldingsformaat}.}
	\label{fig:bijlages-onderzoek-render}
\end{figure}
\FloatBarrier

\FloatBarrier
\begin{figure}[h!]
	\centering
	\fbox{\includegraphics[width=0.5\linewidth]{img/bijlages/onderzoek/resultaat/lossless/png-gem-med.png}}
	\caption{Het gemiddelde en de mediaan voor het kenmerk 'algemene indruk' van afbeeldingen met het \gls{lossless} \gls{png} afbeeldingsformaat.}
	\label{fig:bijlages-onderzoek-resultaten-png-gem-med}
\end{figure}
\FloatBarrier

\FloatBarrier
\begin{figure}[h!]
	\centering
	\fbox{\includegraphics[width=0.5\linewidth]{img/bijlages/onderzoek/resultaat/lossless/lossless-webp-gem-med.png}}
	\caption{Het gemiddelde en de mediaan voor het kenmerk 'algemene indruk' van afbeeldingen met het \gls{lossless} \gls{webp} afbeeldingsformaat.}
	\label{fig:bijlages-onderzoek-resultaten-lossless-webp-gem-med}
\end{figure}
\FloatBarrier

\FloatBarrier
\begin{figure}[h!]
	\centering
	\fbox{\includegraphics[width=0.5\linewidth]{img/bijlages/onderzoek/resultaat/lossless/lossless-jpeg2000-gem-med.png}}
	\caption{Het gemiddelde en de mediaan voor het kenmerk 'algemene indruk' van afbeeldingen met het \gls{lossless} \gls{jpeg2000} afbeeldingsformaat.}
	\label{fig:bijlages-onderzoek-resultaten-lossless-jpeg2000-gem-med}
\end{figure}
\FloatBarrier

\FloatBarrier
\begin{figure}[h!]
	\centering
	\fbox{\includegraphics[width=0.5\linewidth]{img/bijlages/onderzoek/resultaat/lossy/lossy_sizes.png}}
	\caption{De bestandsgrootte in \glspl{bit} voor de afbeeldingen die \gls{lossy} gecomprimeerd zijn door zowel \gls{jpeg}, \gls{jpeg2000} en \gls{webp}.}
	\label{fig:onderzoek-resultaten-lossy-sizes}
\end{figure}
\FloatBarrier

\FloatBarrier
\begin{figure}[h!]
	\centering
	\fbox{\includegraphics[width=0.5\linewidth]{img/bijlages/onderzoek/resultaat/lossy/lossy_rating_ratio_jpf.png}}
	\caption{Verhouding van de gemiddelde score ten opzichte van de bestandsgrootte voor \gls{jpf}.}
	\label{fig:onderzoek-resultaten-lossy-ratio-jpf}
\end{figure}
\FloatBarrier

\FloatBarrier
\begin{figure}[h!]
	\centering
	\fbox{\includegraphics[width=0.5\linewidth]{img/bijlages/onderzoek/resultaat/lossy/lossy_rating_ratio_jpg.png}}
	\caption{Verhouding van de gemiddelde score ten opzichte van de bestandsgrootte voor \gls{jpeg}.}
	\label{fig:onderzoek-resultaten-lossy-ratio-jpg}
\end{figure}
\FloatBarrier

\FloatBarrier
\begin{figure}[h!]
	\centering
	\fbox{\includegraphics[width=0.5\linewidth]{img/bijlages/onderzoek/resultaat/lossy/lossy_rating_ratio_webp.png}}
	\caption{Verhouding van de gemiddelde score ten opzichte van de bestandsgrootte voor \gls{webp}.}
	\label{fig:onderzoek-resultaten-lossy-ratio-webp}
\end{figure}
\FloatBarrier
\appendix
\renewcommand{\chaptername}{Appendix}

%%---------- Onderzoeksvoorstel ----------

\chapter{Onderzoeksvoorstel}

Het onderwerp van deze bachelorproef is gebaseerd op een onderzoeksvoorstel dat vooraf werd beoordeeld door de promotor. Dat voorstel is opgenomen in deze bijlage.

% Verwijzing naar het bestand met de inhoud van het onderzoeksvoorstel
%---------- Inleiding ---------------------------------------------------------

\section{Introductie} % The \section*{} command stops section numbering
\label{sec:introductie}

Compressie is overal, bij katten video’s op YouTube, vakantiefoto’s op Instagram, zelfs bij het digitaal raadplegen van iemands DNA. Een wereld zonder compressie is ondenkbaar, er zouden enorme veelvouden van de huidige data opslag, brandbreedte en hardware capaciteit nodig moeten zijn om dezelfde data van vandaag te verwerken.

Bij DNA compressie is een verkleining van bestandsgrootte van meer dan 99 \% niet ongewoonlijk \autocite{Afify2011}. Bij afbeelding- en videocompressie kan een andere codec die een visueel gelijkaardig resultaat geeft een bestandsgrootte van factor tien hebben. Dit wilt zeggen dat compressie één van de belangrijkste factoren is, zeker vanuit het perspectief van de eindgebruiker, voor het optimaliseren van snelheid en kost bij applicatieontwikkeling en meer.

Bij een kleine bevraging van een tiental toegepaste informatica studenten te HoGent, één digital content team, twee mobile app developers en drie web developers bleek echter dat geen enkel van deze intensief bezig was met het bepalen van welke codec ze zullen gebruiken voor de afbeeldingen en video’s binnen hen project. Vrijwel iedereen wist wel dat het belangrijk was afbeeldingen en video’s te uploaden tegen een lagere resolutie maar de gebruikte codec verdedigen ging voor velen niet verder dan “het is voorgesteld door dit tooltje” of “bij JPEG heb je geen doorzichtige achtergrond”. 

Deze vaststelling was de triggerende factor voor deze bachelorproef. Een onderzoek naar waarom standaarden als JPEG en PNG nog niet vervangen zijn, welke interessanter is voor welk gebruik en hoeveel tijd en geld bespaard kan worden door minimale inspanning van de juiste codec keuze. 

Concreet zal het ontstaan van compressie en de fundamentele wiskunde achter de primitieve vormen van compressie besproken worden. De werking van JPEG en PNG compressie toegelicht worden om zo te kunnen bepalen welke beter is voor welke doeleinden. Videocompressie uitgelegd worden a.d.h.v. een vergelijkende studie tussen H.264/AVC en H.264/SVC. Ook zal er een blik geworpen worden op huidige en toekomstige uitdagingen voor compressie zoals DNA-compressie.

%---------- Stand van zaken ---------------------------------------------------

\section{Stand van zaken}
\label{sec:stand-van-zaken}




Hier beschrijf je de \emph{state-of-the-art} rondom je gekozen onderzoeksdomein. Dit kan bijvoorbeeld een literatuurstudie zijn. Je mag de titel van deze sectie ook aanpassen (literatuurstudie, stand van zaken, enz.). Zijn er al gelijkaardige onderzoeken gevoerd? Wat concluderen ze? Wat is het verschil met jouw onderzoek? Wat is de relevantie met jouw onderzoek?

Verwijs bij elke introductie van een term of bewering over het domein naar de vakliteratuur, bijvoorbeeld~\textcite{Salomon2006}! Denk zeker goed na welke werken je refereert en waarom.

% Voor literatuurverwijzingen zijn er twee belangrijke commando's:
% \autocite{KEY} => (Auteur, jaartal) Gebruik dit als de naam van de auteur
%   geen onderdeel is van de zin.
% \textcite{KEY} => Auteur (jaartal)  Gebruik dit als de auteursnaam wel een
%   functie heeft in de zin (bv. ``Uit onderzoek door Doll & Hill (1954) bleek
%   ...'')

Je mag gerust gebruik maken van subsecties in dit onderdeel.

%---------- Methodologie ------------------------------------------------------
\section{Methodologie}
\label{sec:methodologie}

Hier beschrijf je hoe je van plan bent het onderzoek te voeren. Welke onderzoekstechniek ga je toepassen om elk van je onderzoeksvragen te beantwoorden? Gebruik je hiervoor experimenten, vragenlijsten, simulaties? Je beschrijft ook al welke tools je denkt hiervoor te gebruiken of te ontwikkelen.

%---------- Verwachte resultaten ----------------------------------------------
\section{Verwachte resultaten}
\label{sec:verwachte_resultaten}

Hier beschrijf je welke resultaten je verwacht. Als je metingen en simulaties uitvoert, kan je hier al mock-ups maken van de grafieken samen met de verwachte conclusies. Benoem zeker al je assen en de stukken van de grafiek die je gaat gebruiken. Dit zorgt ervoor dat je concreet weet hoe je je data gaat moeten structureren.

%---------- Verwachte conclusies ----------------------------------------------
\section{Verwachte conclusies}
\label{sec:verwachte_conclusies}

Hier beschrijf je wat je verwacht uit je onderzoek, met de motivatie waarom. Het is \textbf{niet} erg indien uit je onderzoek andere resultaten en conclusies vloeien dan dat je hier beschrijft: het is dan juist interessant om te onderzoeken waarom jouw hypothesen niet overeenkomen met de resultaten.



%%---------- Andere bijlagen ----------
% TODO: Voeg hier eventuele andere bijlagen toe
%\input{...}

%%---------- Referentielijst ----------

%voorzie alle bronnen uit de db in referenties
\nocite{*}
%\phantomsection
\printbibliography[heading=bibintoc]

\end{document}
