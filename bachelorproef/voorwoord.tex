%==================
%% Voorwoord
%==================

\chapter*{Woord vooraf}
\label{ch:voorwoord}

De zoektocht naar een leuk, leerrijk en bruikbaar bachelorproefonderwerp is niet makkelijk. Wanneer ik echter op zoek was naar een geschikt \gls{afbeeldingsformaat} voor het bewaren van mijn steeds groeiende afbeeldingscollectie was ik stomverbaasd hoe weinig ik over dit onderwerp wist. Ook mijn vrienden, waarvan vele medestudenten Toegepaste Informatica te HoGent, konden mij geen antwoord geven op de vraag welk \gls{afbeeldingsformaat} een goede keuze zou zijn en kenden buiten het feit dat \gls{png} wel een transparante achtergrond kan hebben en \gls{jpeg} niet, geen echte verschillen tussen deze twee bekendste \glspl{afbeeldingsformaat}.

Ook op mijn stageplaats, waar websites en webshops gemaakt en onderhouden worden, waren maar weinig mensen zich echt bewust van de voordelen en nadelen van de verschillende \glspl{afbeeldingsformaat}. Meestal exporteerden ze afbeeldingen vanuit \gls{ps} met een kwaliteitsinstelling zodanig het eindbestand kleiner was dan 100kb om de performance van een website hoog te houden. De keuze voor een \gls{afbeeldingsformaat} anders dan \gls{jpeg} of \gls{png} wordt hier en in het algemeen te weinig overwogen ondanks dat hier een grote kwaliteitswinst gedaan kan worden met dezelfde of kleinere bestandsgrootte. 

Deze onwetendheid deed mij beseffen dat een onderzoek naar de verschillende soorten \gls{datacompressie} interessant kon worden. Zowel voor mede programmeurs, die hun \gls{css} \gls{minifyen} om enkele kilobytes te besparen maar niet stilstaan hoeveel data ze kunnen besparen door het kiezen van een gepast \gls{afbeeldingsformaat}, als content creators en tal van andere personen in de IT-wereld. 

De achterliggende wiskunde en boeiende uitdagingen zoals \gls{dna-compressie} wisten mij ook te overtuigen dat dit onderwerp zeer verreikend zou zijn voor mij. Het vinden van een geweldige co-promotor, vakexpert Tom Paridaens, was dan ook de kers op de taart.

\pagebreak

Graag bedank ik dan ook mijn co-promotor, Tom Paridaens, voor de nauwe samenwerking binnen deze bachelorproef. Ook bedank ik graag Wim De Bruyn voor de leerrijke lessen onderzoekstechnieken die ons de nodige kennis brachten voor het voeren van een gegrond onderzoek. Bovendien is Wim De Bruyn de promotor van deze bachelorproef waarvoor ik hem ook wil danken.

Ook wil ik Mayté Bogaert, van MaytéB fotografie, bedanken voor het aanleveren van meerdere \gls{raw} bestanden die ik heb gebruikt voor het uitvoeren van mijn onderzoek.

Bert Van Vreckem en collega's verdienen ook een dankwoord voor het opstellen van een \LaTeX{} sjabloon voor deze bachelorproef. Tot slot bedank ik graag de deelnemers die aan de hand van mijn \gls{afbeeldingsevaluatietool} hebben bijgedragen naar het onderzoek van een geschikt \gls{afbeeldingsformaat} voor de \gls{use-case} besproken in deze bachelorproef.