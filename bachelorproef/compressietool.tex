\chapter{Proof of concept compressietool}
\label{ch:compressietool}

In deel \ref{sec:primitieve-technieken-voorbeeld-rle} en \ref{sec:primitieve-technieken-voorbeeld-huffman-encoding} wordt uitgelegd hoe \gls{rle-long} en \gls{huffman-coding} werken. Dit hoofdstuk focust zich op de implementatie van deze \gls{compressie-algoritme}. Er zal een proof of concept \gls{compressietool} gebouwd worden in \gls{php} en de werking zal toegelicht worden. Deze is in staat om input bestanden onder de vorm van simpele tekst in een txt bestand om te zetten naar hen gecomprimeerde vorm.
 
\section{Gebruikte technologie}
\label{sec:compressietool-gebruikte-technologie}

Deze \gls{compressietool} is geschreven in \gls{php}. Dit maakt het mogelijk te tool eenvoudig lokaal te runnen door het gebruik van een webserver omgeving als \gls{xampp} of hem online te zetten op een \gls{hosting} platform. Deze tool is online raadpleegbaar op de website van Lennert Bontinck\urlcite{compressietool}.

\section{Run length encoding}
\label{sec:compressietool-rle}

TODO
%TODO: compressietool

\section{Huffman Coding}
\label{sec:compressietool-huffman}

TODO
%TODO: compressietool

\section{Patronen}
\label{sec:compressietool-patronen}

TODO
%TODO: compressietool

\section{Resultaten}
\label{sec:compressietool-resultaten}

TODO
%TODO: compressietool