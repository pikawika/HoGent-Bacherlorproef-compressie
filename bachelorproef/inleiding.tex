%%=============================================================================
%% Inleiding
%%=============================================================================

\chapter{Inleiding}
\label{ch:inleiding}

\Gls{datacompressie}, en compressie in het algemeen is niets nieuw. Integendeel, het is één van de oudste concepten binnen IT en tot op heden van fundamenteel belang voor zowat alle IT-toepassingen. Door de databesparing kan veelvouden sneller en goedkoper gewerkt worden. Datacompressie en de altijd groeiende waaier aan \glspl{compressie-algoritme} maken het ook mogelijk met data te verwerken waar voorheen geen beginnen aan was. Denk hierbij bijvoorbeeld aan recente doorbraken binnen \gls{dna-compressie} die het mogelijk maken steeds meer onderzoeken met betrekking tot het menselijk genoom uit te voeren.

Er zijn reeds tal van uitgebreide en professionele papers geschreven rond zowel \gls{datacompressie} als \gls{afbeeldingscompressie} en \gls{videocompressie} in het bijzonder. Dit is logisch, want het zijn onderwerpen die van fundamenteel belang zijn binnen IT. De focus van deze paper ligt dan ook niet op het herschrijven van papers die reeds bestaan maar aan het maken van een nuttig document voor iedereen binnen de IT-wereld. 

Door de verworven kennis zal je als programmeur, content creator of eender welk andere belanghebber inzicht krijgen hoe \gls{datacompressie} ontstaan is, hoe het werkt en waarom het zo belangrijk is. Er zal duidelijkheid komen rond de verschillende mogelijkheden voor \gls{afbeeldingscompressie} en \gls{videocompressie} en de bijkomende voordelen en nadelen. Dit zal inzicht geven tot waarom het zo belangrijk is voor de juiste \glspl{codec} en \glspl{afbeeldingsformaat} te kiezen.

Er zal een onderzoek uitgevoerd worden welk \gls{afbeeldingsformaat} het best is voor een bepaalde \gls{use-case}: portret foto's op de portfolio van een fotografe. De hiervoor ontwikkelde en gratis te gebruiken open source  \gls{afbeeldingsevaluatietool} zal je dan ook de mogelijkheid bieden zelf een subjectief onderzoek te voeren binnen je doelpubliek.

\section{Probleemstelling}
\label{sec:probleemstelling}
\Gls{datacompressie}, \glspl{afbeeldingsformaat} en video \glspl{codec} spelen een belangrijke rol voor vele werknemers binnen de IT-sector. De kennis van deze onderwerpen is echter ondermaats. 

Dit is deels te weiden aan vele opleidingsinstituten die niet dieper ingaan op \gls{datacompressie} binnen opleidingen als Toegepaste Informatica, Communicatie Management en Digital Media Manager.

Anderzijds is de beschikbare informatie veelal te complex uitgelegd dat een doorsnee lezer al snel afhaakt. Ook focussen veel van deze documenten zich op één specifiek onderdeel binnen \gls{datacompressie} zoals het gedetailleerd uitleggen van één \gls{afbeeldingsformaat} waardoor een globaal beeld moeilijk te scheppen valt.

Het is juist hier waar deze paper beter wilt doen. Dit wordt bereikt door het gebruik van een brede waaier van besproken \glspl{afbeeldingsformaat} en video \glspl{codec}, een woordenlijst (hoofdstuk \ref{ch:termen}), en de nodige theoretische als praktische uitleg gevalideerd door vakexpert Tom Paridaens.

\section{Onderzoeksvragen}
\label{sec:onderzoeksvragen}
Deze bachelorproef tracht een antwoord te geven op de volgende vragen: 
\begin{itemize}
	\item Hoe is \gls{datacompressie} binnen IT ontstaan?
	\item Wat waren enkele van de eerste primitieve \glspl{compressie-algoritme}?
	\item Waar zitten de verschillen tussen bij grote waaier aan \glspl{afbeeldingsformaat} en video \glspl{codec}?
	\item Hoe kan \gls{datacompressie} correct geïmplementeerd worden?
	\item Wat is het verschil tussen de \glspl{afbeeldingsformaat}:\gls{png}, \gls{jpeg}, \gls{jpeg2000}, \gls{webp} en \gls{heic}
	\item Wat is het verschil tussen de video \glspl{codec}: \gls{h264-avc}, \gls{h264-svc}, \gls{h265} en \gls{av1}?
	\item Wat is \gls{dna-compressie} en wat zijn andere uitdagingen binnen \gls{datacompressie}?
\end{itemize}

Hierdoor zou de hoofdonderzoeksvraag moeten kunnen beantwoord worden, zijnde:
\begin{itemize}
	\item Waarom moet er stilgestaan worden bij het gebruiken van \glspl{compressie-algoritme}, hoe kies je een juist \gls{compressie-algoritme} voor een bepaalde \gls{use-case} en hoe implementeer je die het best.
\end{itemize}

\section{Onderzoeksdoelstelling}
\label{sec:onderzoeksdoelstelling}

De doelstelling van deze bachelorproef is het vormen van een nuttig document voor het brede scala belanghebbende naar de onderzoeksvragen (\ref{sec:onderzoeksvragen}). Enerzijds is dit document dus een vergelijkende studie tussen verschillende video \glspl{codec} en \glspl{afbeeldingsformaat} waardoor het ook een verslag vormt met tal van aanbevelingen. Anderzijds zorgt de technische uitleg en basic proof-of-concept \gls{compressietool} samen met de open source \gls{afbeeldingsevaluatietool} voor de nodige kennis en middelen om verder onderzoek te richten.

%TODO: verder 
%Wat is het beoogde resultaat van je bachelorproef? Wat zijn de criteria voor succes? Beschrijf die zo concreet mogelijk. Gaat het bv. om een proof-of-concept, een prototype, een verslag met aanbevelingen, een vergelijkende studie, enz.

\section{Opzet van deze bachelorproef}
\label{sec:opzet-bachelorproef}
\subsection{Deel 1: situering en literatuurstudie}
\label{sec:opzet-bachelorproef-deel-1}

Deze paper zal zich in het eerste deel focussen op het toelichten van de belangrijke termen binnen \gls{datacompressie}. In hoofdstuk \ref{ch:termen} is een lijst met belangrijker termen te vinden die binnen \gls{datacompressie} en deze paper vaak voorkomen. Doorheen deze paper zullen tal van referenties naar deze termen gelegd worden. 

Hoofdstuk \ref{ch:methodologie} licht de gebruikte methodologie voor deze paper toe. Hieruit wordt duidelijk dat deze paper zo objectief mogelijk is opgesteld met een focus op duidelijkheid en reproduceerbaarheid.

Hoofdstuk \ref{ch:literatuurstudie} behoort ook tot het situerende eerste deel en zal het ontstaan van \gls{datacompressie} en enkele basisprincipes toelichten. Een reeks van deze primitieve technieken zullen aan de hand van een voorbeeld toegelicht worden.
%todo: link naar voorbeeld en welke manieren

\subsection{Deel 2: compressie tool ontwikkelen}
\label{sec:opzet-bachelorproef-deel-2}
 
 In het tweede zal een basis \gls{compressietool} programmatisch geïmplementeerd worden om de theorie uit het eerste deel in praktijk te brengen.
 
Hoofdstuk \ref{ch:compressietool} is hierdoor gericht voor technische lezers als programmeurs. Er is echter telkens voldoende randinformatie gegeven zodanig ook de minder technische lezers een blik achter de schermen kunnen verkrijgen.
 %TODO: welke taal tool gemaakt is en wat deze juist doet etc

\subsection{Deel 3: afbeelding- en videocompressie}
\label{sec:opzet-bachelorproef-deel-3}

In het derde deel worden twee subdomeinen van \gls{datacompressie} verder toegelicht: \gls{afbeeldingscompressie} en \gls{videocompressie}. 

In hoofdstuk \ref{ch:afbeeldingscompressie} zal er dieper ingegaan worden op volgende \glspl{afbeeldingsformaat} voor \gls{afbeeldingscompressie}: \gls{jpeg}, \gls{jpeg2000} en \gls{png}. 
%TODO: aanpassen indien andere op input copromoter

In hoofdstuk \ref{ch:videocompressie} zal er verder ingegaan worden op de gekende \gls{videocompressie} standaarden: \gls{h264-avc} en \gls{h264-svc}. Ook de opvolger \gls{h265} en het open source alternatief \gls{av1} zullen besproken worden.
%TODO: aanpassen indien andere op input copromoter

\subsection{Deel 4: onderzoek afbeelding compressie}
\label{sec:opzet-bachelorproef-deel-4}

In het vierde deel wordt besproken hoe compressiemethoden voor video's en afbeeldingen geëvalueerd worden. Hoofdstuk \ref{ch:kwaliteit} zal enkele veel gebruikte tools en methoden voor objectieve en subjectieve vergelijkingen toelichten.

In hoofdstuk \ref{ch:onderzoek} werd een subjectieve test voor het evalueren van afbeeldingskwaliteit uitgevoerd. Deze test focust zich op portretfoto's. Hierbij zullen enkele van de besproken \glspl{afbeeldingsformaat} uit hoofdstuk \ref{ch:afbeeldingscompressie} tegen elkaar concurreren. De gebruikte \gls{afbeeldingsevaluatietool} is voor deze paper opgesteld en is gratis \gls{open-source} toegankelijk gesteld wat het eenvoudig mogelijk maakt om een gelijkaardig onderzoek uit te voeren.
%TODO: in staat stelllen als bv prog of content zelf kiezen welk gebruiken

\subsection{Deel 5: uitdagingen en conclusie}
\label{sec:opzet-bachelorproef-deel-5}

In het vijfde deel zullen de huidige uitdagingen van \gls{datacompressie} kort toegelicht worden. Zo zal hoofdstuk \ref{ch:uitdagingen} een beeld geven van de taken die mensen als Tom Paridaens, co-promoter voor deze paper, krijgen.
%TODO: copromoter zijn job en eerbetoon vermelden

In hoofdstuk \ref{ch:conclusie} wordt kort teruggeblikt op de paper en worden enkele besluiten uit het onderzoek van hoofdstuk \ref{ch:onderzoek} opgesomd. Daarbij wordt ook een aanzet gegeven om zelf meer na te denken over het gebruik van \gls{datacompressie} en bepaalde  \glspl{codec} in projecten, of nog beter, zelf een onderzoek uit te voeren!
%TODO: lezer aanzetten meer etc