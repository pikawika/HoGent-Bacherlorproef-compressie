%%=============================================================================
%% Conclusie
%%=============================================================================

\chapter{Conclusie}
\label{ch:conclusie}

\Gls{datacompressie}, en compressie in het algemeen is niets nieuw. Integendeel, het is één van de oudste concepten binnen IT en tot op heden van fundamenteel belang voor zowat alle IT-toepassingen. Een basiskennis van \gls{datacompressie} en de belangrijkste \glspl{afbeeldingsformaat} en video \gls{codec} is dan ook geen luxe binnen de IT-wereld. Veel vakgerelateerde opleidingen, zoals de opleiding Toegepaste Informatica te HoGent, voorzien echter geen lessen rond \gls{datacompressie} waardoor deze basiskennis voor velen onbestaande is.

Deze bachelorproef bied een oplossing voor dat probleem. Het vormt een gegronde basiskennis van \gls{datacompressie} zonder onnodig complex te zijn wat het geschikt maakt voor de grote variatie van belanghebbende. Vanaf het voorstel waren de doelstellingen van deze bachelorproef, een antwoord bieden op zeven onderzoeksvragen en één hoofdonderzoeksvraag. Deze onderzoeksvragen worden hieronder nog eens kort aangegaan met een terugblik naar de kennis verworven in deze bachelorproef.

\subsection*{Hoe is datacompressie binnen IT ontstaan?}
\label{sec:conclussie-onderzoeksvraag-1}

Vele onderzoekers zijn het erover eens dat \gls{datacompressie} dateert van voor de uitvinding van de computer. Zo kan morsecode gezien worden als een vorm van \gls{datacompressie}. Morsecode is uitgevonden voor het computertijdperk, in 1832, door Samuel F.B. Morse. Het kan aanzien worden als een vorm van datacompressie doordat veel voorkomende letters een kortere audiotoon kregen dan minder gebruikte letters (\cite{morsecode}).

\subsection*{Wat waren enkele van de eerste compressie-algoritmen?}
\label{sec:conclussie-onderzoeksvraag-2}

Enkele van de eerste \glspl{compressie-algoritme} komen aan bod in deel \ref{sec:ontstaan-datacompressie-primitieve-technieken-binnen-it} van deze bachelorproef. Twee belangrijke \glspl{compressie-algoritme} dat al meer dan vijftig jaar bestaan maar tot heden de basis vormen voor vele toepassingen binnen \gls{datacompressie} zijn \gls{rle-long} en \gls{huffman-coding}. De werking van deze \glspl{compressie-algoritme} is dan ook uitgebreid aan bod gekomen in deze bachelorproef. Een theoretische uitleg met een eenvoudig voorbeeld is voorzien in deel \ref{sec:primitieve-technieken-voorbeeld}. In de proof of concept \gls{compressietool} gemaakt voor deze bachelorproef zijn het ook deze twee \glspl{compressie-algoritme} dat gebruikt worden. Deze \gls{compressietool} is verder toegelicht in hoofdstuk \ref{ch:compressietool}.

\subsection*{Waar zitten de verschillen tussen de afbeeldingsformaten en video codecs?}
\label{sec:conclussie-onderzoeksvraag-3}

\Glspl{afbeeldingsformaat} en video \glspl{codec} hebben meer gemeen dan oorspronkelijk gedacht zo worden. Vele \glspl{afbeeldingsformaat} vormen de basis voor een goed presterende video \glspl{codec} en de besproken \glspl{afbeeldingsformaat} \gls{webp} en \gls{heic} vinden juist hun ontstaan in \gls{videocompressie}. De onderlinge verschillen tussen de verschillende besproken \glspl{afbeeldingsformaat} en video \glspl{codec} is af te leiden uit de voordelen en nadelen te vinden in hoofdstuk \ref{ch:afbeeldingscompressie} en \ref{ch:videocompressie}. De delen over het maken van een juiste keuze van \gls{afbeeldingsformaat} (deel \ref{sec:afbeeldingscompressie-keuze}) en video \gls{codec} (deel \ref{sec:videocompressie-keuze}) bieden aan de hand van enkele overzichten ook een duidelijk antwoord op deze vraag.

\subsection*{Hoe kan datacompressie correct geïmplementeerd worden?}
\label{sec:conclussie-onderzoeksvraag-4}

Hoe \gls{datacompressie} correct geïmplementeerd kan worden is terug te vinden in verschillende porties van deze bachelorproef. De \gls{compressietool} en achterliggende code wordt uitgebreid besproken in hoofdstuk \ref{ch:compressietool}. Deze is \gls{open-source} ter beschikking gesteld op \gls{github} en kan zonder enige licenties gebruikt en aangepast worden. Er worden ook enkele beperkingen met deze \gls{compressietool} toegelicht en mogelijke oplossingen wat een geïnteresseerde lezer kan aanzetten deze beperkingen zelf weg te werken. Er wordt ook toegelicht hoe nieuwe generatie \glspl{afbeeldingsformaat} geïmplementeerd kunnen worden met ondersteuning voor alle internetbrowsers in gedachten. Dit is verder toegelicht in deel \ref{sec:afbeeldingscompressie-implementatie}.

\subsection*{Wat is het verschil tussen de afbeeldingsformaten: PNG, JPEG, JPEG2000, WebP en HEIF?}
\label{sec:onderzoeksvraag-5}

Elk \gls{afbeeldingsformaat} wordt toegelicht in hoofdstuk \ref{ch:afbeeldingscompressie}. Hier wordt zowel het ontstaan, de werking en voordelen en nadelen van de verschillende \glspl{afbeeldingsformaat} uitgelegd. Dit bied samen met de resultaten van het onderzoek besproken in deel \ref{sec:onderzoek-resultaten} en \ref{sec:onderzoek-besluit} een uitgebreid inzicht van de verschillen tussen deze \glspl{afbeeldingsformaat}.

\subsection*{Wat is het verschil tussen de video codecs: H.264/AVC, H.264/SVC, H.265/HEVC en AV1?}
\label{sec:conclussie-onderzoeksvraag-6}

Elke video \glspl{codec} wordt toegelicht in hoofdstuk \ref{ch:videocompressie}. Hier wordt zowel het ontstaan als de voordelen en nadelen van de verschillende video \glspl{codec} aangekaart. Zoals in de overzichten van deel \ref{sec:videocompressie-keuze} duidelijk is weergegeven is er binnen \gls{videocompressie} een enorm probleem van complexe licenties. Het is ook daarom dat Google samenwerkt met tal van andere grote bedrijven als Mozilla en Microsoft \gls{av1} op de markt heeft gebracht. Deze veelbelovende video \gls{codec} wordt ook in hoofdstuk \ref{ch:videocompressie} uitgebreid besproken.

\subsection*{Wat is DNA compressie en wat zijn andere uitdagingen binnen datacompressie?}
\label{sec:onderzoeksvraag-7}

Als afsluitend hoofdstuk (\ref{ch:uitdagingen}) is een korte vermelding van enkele huidige uitdagingen binnen \gls{datacompressie} toegelicht. Dit is bewust zeer beknopt gehouden zodanig de lezer warm gemaakt wordt verder opzoekingswerk naar de interessante wereld van \gls{datacompressie} te verrichten!

\subsection{Hoofdonderzoeksvraag}
\label{sec:conclussie-hoofdonderzoeksvraag}

Door het beantwoorden van alle sub onderzoeksvragen kan de hoofdonderzoeksvraag, Waarom moet er stilgestaan worden bij het gebruiken van \glspl{compressie-algoritme}, hoe kies je een geschikt \gls{compressie-algoritme} voor een bepaalde \gls{use-case} en hoe implementeer je dit het best, door de lezer zelf beantwoord worden. Deze bachelorproef bevat namelijk alle nodige informatie om op een gegronde manier op zoek te gaan naar een \gls{compressie-algoritme} voor een bepaalde \gls{use-case}.

\subsection{Mogelijke uitbreidingen}
\label{sec:conclussie-uitbreidingen}

Desondanks deze bachelorproef reeds uit meer dan honderd pagina's bestaat is er nog altijd ruimte voor uitbreidingen. Zo kan het onderzoek herwerkt worden zodanig de verschillende afbeeldingen een gelijke bestandsgrootte hebben wat een conclusie trekken makkelijker zal maken. Deze uitbreiding is relatief simpel te voorzien door de \gls{open-source} en gratis in gebruik \gls{afbeeldingsevaluatietool} dat gemaakt is voor deze bachelorproef.

Maar ook uitbreidingen op de gemaakte proof of concept \gls{compressietool} zijn mogelijk. Denk hierbij aan het combineren van \gls{rle-long} en \gls{huffman-coding} of het implementeren van een compleet nieuw \gls{compressie-algoritme}.

