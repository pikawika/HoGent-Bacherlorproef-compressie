%run generate.sh en run hier
\newglossaryentry{datacompressie}
{
	name={datacompressie},
	description={Datacompressie bestaat uit het digitaal opslaan van een bestand met zo weinig mogelijk bits. Enkele belangrijke factoren voor het bepalen van de juiste datacompressie zijn gewenste kwaliteit, bestandsgrootte en snelheid}
}

\newglossaryentry{bit}
{
	name={bit},
	description={Zoals de naam bit, kort voor binary digit, suggereert kan een bit beschouwd worden als een binair signaal. Een bit wordt beschouwd als de kleinste eenheid voor dataopslag. Meer informatie rond bestandsgrootte en dataopslag wordt besproken in hoofdstuk \ref{ch:literatuurstudie}, deel \ref{sec:bestandsgrootte-dataopslag}},
	plural={bits}
}

\newglossaryentry{dna-compressie}
{
	name={DNA compressie},
	description={Het menselijke DNA kan digitaal voorgesteld worden door een lange lijst van 5 verschillende karakters, gekend als basen. Deze digitale voorstelling bestaat uit meer dan 3 miljard van deze basen (\cite{dodanaugent2011}). DNA compressie bestaat eruit deze reeks van basen zo efficiënt mogelijk op te slaan zodat performante bewerkingen mogelijk zijn met een zo klein mogelijke bestandsgrootte}
}

\newglossaryentry{afbeeldingscompressie}
{
	name={afbeeldingscompressie},
	description={Afbeeldingscompressie bestaat eruit een afbeelding in zo weinig mogelijk aantal bits op te slaan terwijl een aanvaardbare kwaliteit behouden blijft. Dit kan zowel via lossless als lossy algoritmes. Afbeeldingscompressie wordt uitgebreid besproken in hoofdstuk \ref{ch:afbeeldingscompressie}}
}

\newglossaryentry{videocompressie}
{
	name={videocompressie},
	description={Videocompressie bestaat eruit een videobestand in zo weinig mogelijk aantal bits op te slaan terwijl een accepteerbare kwaliteit behouden blijft. Dit kan zowel via lossless als lossy algoritmes. Videcompressie wordt uitgebreid besproken in hoofdstuk \ref{ch:videocompressie}}
}

\newglossaryentry{codec}
{
	name={codec},
	description={De coder-decoder. Binnen datacompressie betekent codec de gebruikte techniek om een bestand te comprimeren. De codec is dus de technologie verantwoordelijk voor het encoden en decoden van een bestand volgens een bapaald compressiealgoritme. Bv: H.265/HEVC},
	plural={codecs}
}

\newglossaryentry{afbeeldingsformaat}
{
	name={afbeeldingsformaat},
	description={Een afbeeldingsformaat bevat alle gegevens voor het digitaal opslaan van een afbeelding. De vier bekendste categorieën van afbeeldingsformaten zijn: raster, vector, compound en stereo formaten},
	plural={afbeeldingsformaten}
}

\newglossaryentry{container}
{
	name={container},
	description={Binnen datacompressie kan een container vrijwel letterlijk vertaald worden. Het is een verpakking voor alle data die men opslaat en metadata. Onder andere de codec plaatst data in deze container. Wanneer er gesproken wordt over bestandsextensies wordt vaak de container bedoeld. Bv: MP4}
}

\newglossaryentry{jpeg}
{
	name={JPEG},
	description={Ook gekend als JPG. JPEG is een afkorting voor Joint Photographic Experts Group. JPEG is een bestandsformaat voor het opslaan van digitale afbeeldingen via lossy compressie. Afbeeldingscompressie en JPEG worden uitgebreid besproken in hoofdstuk \ref{ch:afbeeldingscompressie}}
}

\newglossaryentry{jpeg2000}
{
	name={JPEG2000},
	description={Ook gekend als JPEG2K. JPEG2000 is een bestandsformaat voor het opslaan van digitale afbeeldingen gemaakt als opvolger van JPEG. Net zoals JPEG maakt het gebruik van lossy compressie. Afbeeldingscompressie en JPEG2000 worden uitgebreid besproken in hoofdstuk \ref{ch:afbeeldingscompressie}}
}

\newglossaryentry{png}
{
	name={PNG},
	description={PNG is een afkorting voor Portable Network Graphics. PNG is een bestandsformaat voor het opslaan van digitale afbeeldingen. PNG maakt gebruik van lossless compressie. Afbeeldingscompressie en PNG worden uitgebreid besproken in hoofdstuk \ref{ch:afbeeldingscompressie}}
}

\newglossaryentry{h264-avc}
{
	name={H.264-AVC},
	description={H.264-AVC is één van de bekendstes videocodecs die grootschalig gebruikt wordt. AVC is een afkorting van Advanced Video Coding. H.264-AVC wordt uitgebreid besproken in deel \ref{sec:videocompressie-h264-AVC}}
}

\newglossaryentry{h264-svc}
{
	name={H.264-SVC},
	description={H.264-SVC is een videocodec ontwikkelt als extensie op H264-AVC. SVC is een afkorting voor Scalable Video Coding. De nadruk bij deze extensie ligt zoals de naam suggereert op schaalbaarheid. H.264-SVC wordt uitgebreid besproken in deel \ref{sec:videocompressie-h264-SVC}}
}

\newglossaryentry{h265}
{
	name={H.265/HEVC},
	description={H.265/HEVC is een videocodec ontwikkelt als opvolger van H.264/AVC. H.265/HEVC wordt uitgebreid besproken in deel \ref{sec:videocompressie-h265}}
}

\newglossaryentry{av1}
{
	name={AV1},
	description={AV1 is een videocodec ontwikkelt als open standaard. AV1 is een afkorting voor AOMedia Video. AV1 is vooral interessant omdat het royalty free is en dus geen licentiekosten heeft. AV1  wordt uitgebreid besproken in deel \ref{sec:videocompressie-av1}}
}

\newglossaryentry{open-source}
{
	name={open source},
	description={Als een programmeer project open source is, wilt dit zeggen dat de broncode raadpleegbaar is. Dit wil echter niet gegarandeerd zeggen dat het software programma gratis is in gebruik of de code zomaar aangepast mag worden. Dit hangt af van de licentie}
}

\newglossaryentry{lossless}
{
	name={lossless},
	description={Binnen datacompressie slaat lossless compressie op het comprimeren van een bestand zonder kwaliteitsverlies. In het geval van video's en afbeeldingen wilt dit zeggen dat een bestand gecomprimeerd met een lossless algoritme identiek is aan het origineel}
}

\newglossaryentry{lossy}
{
	name={lossy},
	description={Binnen datacompressie slaat lossy compressie op het comprimeren van een bestand met kwaliteitsverlies voor het besparen van data. In het geval van video's en afbeeldingen wilt dit zeggen dat een bestand gecomprimeerd met een lossy algoritme een significante hoeveelheid aan data en kwaliteit kan verliezen in vergelijking met het originele bestand}
}

\newglossaryentry{afbeeldingsevaluatietool}
{
	name={afbeeldingsevaluatietool},
	description={Een applicatie gemaakt voor het voeren van een objectief of subjectief onderzoek naar afbeeldingskwaliteit. De mogelijke soorten tools worden verder besproken in hoofdstuk \ref{ch:kwaliteit}. Een subjectieve afbeeldingsevaluatietool werd gebouwd voor deze paper en wordt verder besproken in hoofdstuk \ref{ch:onderzoek}}
}

\newglossaryentry{binaire-voorstelling-bestandsgrootte}
{
	name={Binaire voorstelling},
	description={De binaire voorstelling voor bestandsgroottes maakt gebruik van 1024 als basis, wat overeen komt met $ 2^{10} $. Dit wordt verder besproken in deel \ref{sec:bestandsgrootte-dataopslag-voorvoegsels-binair}}
}

\newglossaryentry{si-voorstelling-bestandsgrootte}
{
	name={SI voorstelling},
	description={De SI voorstelling voor bestandsgroottes maakt gebruik van 1000 als basis wat overeen komt met $ 10^{3} $. Dit wordt verder besproken in deel \ref{sec:bestandsgrootte-dataopslag-voorvoegsels-si}}
}

\newglossaryentry{ieee}
{
	name={IEEE},
	description={Institute of Electrical and Electronics Engineers. Het is een non-profit instituut dat zich inzet voor technologische vooruitgang}
}

\newglossaryentry{clustergrootte}
{
	name={clustergrootte},
	description={De grootte van een cluster. Dit is verder besproken in deel \ref{sec:bestandsgrootte-dataopslag-clustergrootte}}
}

\newglossaryentry{cluster}
{
	name={cluster},
	description={Het kleinste deel van een oplsagmedium waar data in voorzien kan worden. Dit is verder besproken in deel \ref{sec:bestandsgrootte-dataopslag-clustergrootte}},
	plural={clusters}
}

\newglossaryentry{byte}
{
	name={byte},
	description={Een term voor acht bits},
	plural={bytes}
}

\newglossaryentry{leestijd}
{
	name={leestijd},
	description={De tijd die verstrijkt tussen het aanvragen van een bestand op een opslagmedium tot het effectief verkrijgen van dat bestand},
	plural={leestijden}
}

\newglossaryentry{bandbreedte}
{
	name={bandbreedte},
	description={Een term waarmee de maximale beschikbare hoeveelheid data dat over een netwerkverbinding verstuurd kan worden bedoeld wordt}
}

\newglossaryentry{prefix-code}
{
	name={prefix code},
	description={Een korte waarde die een langere waarde voorstelt binnen prefix coding. Dit is verder beschreven in deel \ref{sec:ontstaan-datacompressie-primitieve-technieken-binnen-it}},
	plural={prefix codes}
}

\newglossaryentry{huffman-coding}
{
	name={Huffman coding},
	description={Een compressie-algoritme gebasseerd op prefix code. Huffman coding wordt verder toegelicht aan de hand van een voorbeeld in deel \ref{sec:primitieve-technieken-voorbeeld-huffman-encoding}. Een implementatie wordt voorzien in hoofdstuk \ref{ch:compressietool}}
}

\newglossaryentry{compressie-algoritme}
{
	name={compressie-algoritme},
	description={Een set instructies die het mogelijk maken om bepaalde informatie met minder resources voor te stellen. Binnen deze bachelorproef wilt dit code voorstellen die een bestand zijn bestandsgrootte kan verkleinen. Dit kan zowel lossless als lossy},
	plural={compressie-algoritmen}
}

\newglossaryentry{lookup-table}
{
	name={lookup table},
	description={Een tabel waar waardes in opgezocht kunnen worden. Bij prefix code is dit de tabel waar de lange waarde staat die de korte prefix code voorstelt}
}

\newglossaryentry{prefix-coding}
{
	name={prefix coding},
	description={Een vorm van datacompressie waarbij een waarde wordt toegekend aan iets dat verwijst naar een andere, langere waarde. Dit is verder beschreven in deel \ref{sec:ontstaan-datacompressie-primitieve-technieken-binnen-it}}
}

\newglossaryentry{use-case}
{
	name={use case},
	description={De beschrijving van het gewenst gedrag van een systeem of applicatie},
	plural={use cases}
}

\newglossaryentry{rle-long}
{
	name={run length encoding},
	description={Een compressietechniek die verder besproken wordt in deel \ref{sec:primitieve-technieken-voorbeeld-rle} en geïmplementeerd wordt in deel \ref{ch:compressietool}}
}

\newglossaryentry{rle-short}
{
	name={RLE},
	description={Kort voor run length encoding}
}

\newglossaryentry{ascii}
{
	name={ASCII},
	description={Een tekenset die voorgesteld wordt door acht bits}
}

\newglossaryentry{encoding}
{
	name={encoding},
	description={Het proces dat een encoder uitvoert}
}

\newglossaryentry{decoding}
{
	name={decoding},
	description={Het process dat een decoder uitvoert}
}

\newglossaryentry{meta-data}
{
	name={metadata},
	description={Data over data}
}

\newglossaryentry{webp}
{
	name={WebP},
	description={Een afbeeldingsformaat dat uitgebreid besproken wordt in deel \ref{sec:afbeeldingscompressie-webp}}
}

\newglossaryentry{heif}
{
	name={HEIF},
	description={Een codec dat uitgebreid besproken wordt in deel \ref{sec:afbeeldingscompressie-heif}}
}

\newglossaryentry{ps}
{
	name={Adobe Photoshop},
	description={Computersoftware voor het aanmaken en bewerken van rasterafbeeldingen}
}

\newglossaryentry{css}
{
	name={CSS},
	description={Cascading Style Sheets. Een programmeertaal die verantwoordelijk is voor het opmaken van een webpagina}
}

\newglossaryentry{minifyen}
{
	name={minifyen},
	description={Een proces waarbij code zodanig herschreven wordt dat dit de minste hoveelheid ruimte in beslag neemt}
}

\newglossaryentry{render}
{
	name={render},
	description={Het genereren van een digitale afbeelding of video met behulp van computersoftware}
}

\newglossaryentry{raw}
{
	name={RAW},
	description={Een benaming voor afbeeldingsformaten die in geen enkele vorm gecomprimeerd worden. RAW wordt uitgebreid besproken in deel \ref{sec:afbeeldingscompressie-raw}}
}

\newglossaryentry{compressietool}
{
	name={datacompressietool},
	description={Een kleine applicatie voor het comprimeren van data. In hoofdstuk \ref{ch:compressietool} wordt een proof of concept compressietool geïmplementeerd}
}

\newglossaryentry{github}
{
	name={GitHub},
	description={Een online platform voor versiebeheer van code}
}

\newglossaryentry{hosting}
{
	name={hosting},
	description={Het delen van een bepaalde webpagina met anderen}
}

\newglossaryentry{php}
{
	name={PHP},
	description={Een programmeertaal die voornamelijk de logica bij websites voorziet}
}

\newglossaryentry{sql}
{
	name={SQL},
	description={Een standaard programmertaal voor communicatie met een relationele databank}
}

\newglossaryentry{xampp}
{
	name={XAMPP},
	description={Een software pakket waarmee eenvoudig een hosting omgeving kan opgezet worden}
}

\newglossaryentry{jquery}
{
	name={JQuery},
	description={Een uitbreiding op JavaScript}
}

\newglossaryentry{drift}
{
	name={Drift},
	description={Een JavaScript library die het eenvoudig maakt om in te zoomen op afbeeldingen zonder manipulaties}
}

\newglossaryentry{bootstrap}
{
	name={Bootstrap},
	description={Een toolkit voor het eenvoudig programmeren met HTML, CSS, JS en jQuery}
}

\newglossaryentry{js}
{
	name={JavaScript},
	description={Een programmeertaal waarmee onder andere de inhoud van een webpagina gemanipuleerd kan worden}
}

\newglossaryentry{yt}
{
	name={YouTube},
	description={Een online website waar gebruikers gratis video's kunnen uploaden. YouTube is een dochterbedrijf van Google en is één van de grootste streaming diensten}
}

\newglossaryentry{plug-in}
{
	name={plug-in},
	description={Een downloadbare collectie code die bepaalde functionaliteit voorziet},
	plural={plug-ins}
}

\newglossaryentry{artefact}
{
	name={artefact},
	description={Een kunstmatig verschijnsel. Binnen datacompressie zijn artefacten zichtbare of hoorbare fouten},
	plural={artefacten}
}

\newglossaryentry{wavelet}
{
	name={wavelet},
	description={Een golfvormige soort data},
	plural={wavelets}
}

\newglossaryentry{ai}
{
	name={artificiële intelligentie},
	description={Wanneer een apparaat kan reageren op binnenkomende data en op basis daarvan een eigen beslissing kan maken zonder dat dit expliciet geprogrammeerd is, spreekt men van artificiële intelligentie}
}

\newglossaryentry{nef}
{
	name={NEF},
	description={Het RAW afbeeldingsformaat van Nikon}
}

\newglossaryentry{dng}
{
	name={DNG},
	description={Digital Negative Specification is een RAW formaat dat Adobe heeft uitgevonden met de bedoelding dat dit wereldwijd geadopteerd zou worden als standaard voor RAW afbeeldingsformaten}
}

\newglossaryentry{w3c}
{
	name={W3C},
	description={World Wide Web Consortium is een organisatie die standaarden voor het web voorziet}
}

\newglossaryentry{gif}
{
	name={GIF},
	description={Graphics interchange format is een afbeeldingsformaat dat op heden voornamelijk gebruikt wordt om bewegende delen voor te stellen}
}

\newglossaryentry{rgb}
{
	name={RGB},
	description={Red Green Blue is een kleurcodering systeem voor het voorstellen van kleuren aan de hand van deze drie basiskleuren}
}

\newglossaryentry{cmyk}
{
	name={CMYK},
	description={Cyan, Magenta, Yellow, Key is een kleurcodering systeem voor het voorstellen van kleuren aan de hand van deze drie basiskleuren}
}

\newglossaryentry{jpeg-exif}
{
	name={JPEG/Exif},
	description={Een bestandsformaat verder toegelicht in deel \ref{sec:afbeeldingscompressie-jpeg}}
}

\newglossaryentry{jpeg-jfif}
{
	name={JPEG/JFIF},
	description={Een bestandsformaat verder toegelicht in deel \ref{sec:afbeeldingscompressie-jpeg}}
}

\newglossaryentry{extensie}
{
	name={bestandsextensie},
	description={Een toevoeging op het einde van een bestandsnaam om weer te geven wat voor soort bestand het is},
	plural={bestandsextensies}
}

\newglossaryentry{dct}
{
	name={DCT},
	description={Discrete cosine transform is een wiskundige formule die binnen datacompressie voornamelijk gebruikt wordt om een pixel te kunnen voorstellen als een makkelijk comprimeerbaar getal}
}

\newglossaryentry{wtcq}
{
	name={WTCQ},
	description={Wavelet/Trellis Coded Quantization is een compressie algoritme dat de start vormde voor JPEG2000. JPEG2000 wordt uigebreid besproken in deel \ref{sec:afbeeldingscompressie-jpeg2000}}
}

\newglossaryentry{iso}
{
	name={ISO},
	description={De International Organization for Standardization is een organisatie die internationale standaarden oplegt voor bijvoorbeeld compressie-algoritmen}
}

\newglossaryentry{jpf}
{
	name={JPF},
	description={JPF is de bestandsexentsie van JPEG2000. JPEG2000 wordt uigebreid besproken in deel \ref{sec:afbeeldingscompressie-jpeg2000}}
}

\newglossaryentry{intra-frame}
{
	name={intra-frame},
	description={Intra-frame compressie is een techniek van afbeeldingen comprimeren binnen videocompressie die uitgebreid besproken wordt in deel \ref{sec:videocompressie-intra-inter}},
	plural={intra-frames}
}

\newglossaryentry{inter-frame}
{
	name={inter-frame},
	description={Inter-frame compressie is een techniek van afbeeldingen comprimeren binnen videocompressie die uitgebreid besproken wordt in deel \ref{sec:videocompressie-intra-inter}},
	plural={inter-frames}
}

\newglossaryentry{decoder}
{
	name={decoder},
	description={Een softwareapplicatie die een gecomprimeerd bestand terug omzet naar zijn originele vorm, al dan niet met kwaliteitsverlies },
	plural={decoders}
}

\newglossaryentry{encoder}
{
	name={encoder},
	description={Een softwareapplicatie dat een bestand omzet naar een gecomprimeerde vorm, al dan niet met kwaliteitsverlies},
	plural={encoders}
}

\newglossaryentry{heic}
{
	name={HEIC},
	description={Een afbeeldingsformaat dat uitgebreid besproken wordt in \ref{sec:afbeeldingscompressie-heif}},
}

\newglossaryentry{on-premise}
{
	name={on premise},
	description={Software is on-premise wanneer deze lokaal geïnstalleerd en gedraaid wordt op het toestel van de gebruiker},
}

\newglossaryentry{vector}
{
	name={vector},
	description={Een categorie voor afbeeldingsformaten die verder toegelicht wordt in deel \ref{sec:afbeeldingscompressie-raster-vector}},
}

\newglossaryentry{cms}
{
	name={CMS},
	description={Een content management system is een systeem dat het voor een eindgebruiker eenvoudig moet maken om de inhoud binnen een IT-applicatie eenvoudig te beheren},
	plural={CMS'en}
}

\newglossaryentry{wordpress}
{
	name={WordPress},
	description={Een CMS voor web development}
}

\newglossaryentry{raster}
{
	name={raster},
	description={Een categorie voor afbeeldingsformaten die verder toegelicht wordt in deel \ref{sec:afbeeldingscompressie-raster-vector}}
}

\newglossaryentry{lbhuffman}
{
	name={lbhuffman},
	description={Een bestandsextensie die gebruikt wordt binnen de datacompressietool. Wordt verder besproken in deel \ref{sec:compressietool-opslaan}}
}

\newglossaryentry{lbrlea}
{
	name={lbrlea},
	description={Een bestandsextensie die gebruikt wordt binnen de datacompressietool. Wordt verder besproken in deel \ref{sec:compressietool-opslaan}}
}

\newglossaryentry{lbrle}
{
	name={lbrle},
	description={Een bestandsextensie die gebruikt wordt binnen de datacompressietool. Wordt verder besproken in deel \ref{sec:compressietool-opslaan}}
}

\newglossaryentry{html}
{
	name={HTML},
	description={Een programmeertaal voor het opbouwen van webpagina's}
}

\newglossaryentry{library}
{
	name={library},
	description={Een verzameling codes dat door een programma gebruikt kunnen worden},
	plural={Libraries}
}

\newglossaryentry{string}
{
	name={string},
	description={Een term binnen programmeren die slaat op een tekstfragment zonder opmaak},
	plural={strings}
}

\newglossaryentry{regex}
{
	name={regex},
	description={Regular Expressions zijn uitdrukkingen waarmee op zoek gegaan kan worden naar een tekst met een bepaalde structuur}
}

\newglossaryentry{recursieve-functie}
{
	name={recursieve functie},
	description={een recursieve functie is een functie die zichelf (al dan niet meermaals) oproept}
}

\newglossaryentry{array}
{
	name={array},
	description={Een term binnen programmeren die slaat op een verzameling van variabelen}
}

\newglossaryentry{json}
{
	name={json},
	description={Een techniek voor het opslaan van variabelen in een string}
}

\newglossaryentry{rmse}
{
	name={rmse},
	description={Root mean square error, besproken in deel \ref{sec:kwaliteit-rmse}}
}

\newglossaryentry{ssim}
{
	name={SSIM},
	description={Structural similarity index, besproken in deel \ref{sec:kwaliteit-ssim}}
}

\newglossaryentry{vdp}
{
	name={VDP},
	description={Visual difference predictor is een term voor functies die visuele verschillen tussen afbeeldingen probeert uit te drukken}
}

\newglossaryentry{dxomark}
{
	name={DxOMark},
	description={Een erkende instelling die de kwaliteit van, voornamelijk smartphone, camera's beoordeeld. DxOMark wordt verder toegelicht in deel \ref{sec:kwaliteit-dxomark}}
}

\newglossaryentry{maximum-compression}
{
	name={Maximum Compression},
	description={Een erkende instelling die de compressieratio van lossles compressie-algoritmes beoordeelt. Maximum Compression wordt verder toegelicht in deel \ref{sec:kwaliteit-maximum-compression}}
}

\newglossaryentry{compressieratio}
{
	name={compressieratio},
	description={De verhouding waarmee een bestand gecomprimeerd is ten opzichte van het originele bestand}
}

\newglossaryentry{python}
{
	name={Python},
	description={Een programmeertaal die voornamelijk gebruikt wordt voor het verwerken van data in de vorm van grote datascheets}
}

\newglossaryentry{pandas}
{
	name={Pandas},
	description={Een uitbreiding voor Python}
}

\newglossaryentry{illustrator}
{
	name={Adobe Illustrator},
	description={Computerssoftware voor het bewerken van vector afbeeldingen}
}

\newglossaryentry{svg}
{
	name={SVG},
	description={Scalable Vector Graphics zijn een schaalbare vorm van vector afbeeldingen. Vector wordt verder toegelicht in deel \ref{sec:afbeeldingscompressie-raster-vector}}
}

\newglossaryentry{deflate}
{
	name={deflate},
	description={Een datacompressie techniek die voornamelijk binnen videcompressie gebruikt wordt}
}

\newglossaryentry{pixel-prediction}
{
	name={pixel prediction},
	description={Een datacompressie techniek die de pixel van een afbeelding gaat voorspellen aan de hand van een andere pixel}
}

\newglossaryentry{crc}
{
	name={CRC},
	description={Cyclic redundancy check is een foutdetectie code om te kunnen nagaan of een bestand goed is toegekomen}
}

\newglossaryentry{bitplane}
{
	name={bitplane},
	description={Een term voor een groepering van bits},
	plural={bitplanes}
}

\newglossaryentry{mp4}
{
	name={MP4},
	description={Een container voor onder andere videocodecs. Videocodecs worden verder toegelicht in hoofdstuk \ref{ch:videocompressie}}
}

\newglossaryentry{avi}
{
	name={AVI},
	description={Een container voor onder andere videocodecs. Videocodecs worden verder toegelicht in hoofdstuk \ref{ch:videocompressie}}
}

\newglossaryentry{mkv}
{
	name={MKV},
	description={Een container voor onder andere videocodecs. Videocodecs worden verder toegelicht in hoofdstuk \ref{ch:videocompressie}}
}

\newglossaryentry{mpeg-4}
{
	name={MPEG-4},
	description={Een videocompressie algoritme dat de voorganger was van H.264/AVC, toegelicht in deel \ref{sec:videocompressie-h264-AVC}}
}

\newglossaryentry{hlg}
{
	name={HLG},
	description={Hybrid Log-Gamma is een standaard voor het voorzien van HDR content}
}

\newglossaryentry{hdr}
{
	name={HDR},
	description={High Dynamic Range beelden zijn beelden die een hoog dynamisch bereik hebben}
}

\newglossaryentry{frame}
{
	name={frame},
	description={Een frame is één stilstaande afbeelding binnen een video},
	plural={frames}
}

\newglossaryentry{interlace}
{
	name={interlace},
	description={Een techniek waarbij twee frames tegelijk getoond worden waardoor de kijker een vals beeld krijgt dat de frame rate sneller is dan hij werkelijk is}
}

\newglossaryentry{frame-rate}
{
	name={frame rate},
	description={De snelheid waarmee frames elkaar opvolgen, vaak uitgedruk in frames per seconden (FPS)}
}

\newglossaryentry{pixel}
{
	name={pixel},
	description={Één punt binnen een rasterafbeelding},
	plural={pixels}
}

\newglossaryentry{vp8}
{
	name={VP8},
	description={Een video codec gemaakt door Google. WebP, besproken in deel \ref{sec:afbeeldingscompressie-webp}, is afkomstig van VP8. AV1, besproken in \ref{sec:videocompressie-av1}, is een verre opvolger van VP8}
}