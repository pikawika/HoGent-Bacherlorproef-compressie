%run generate.sh en run hier
\newglossaryentry{datacompressie}
{
	name={datacompressie},
	description={Datacompressie bestaat uit het digitaal opslaan van een bestand met zo weinig mogelijk bits. Enkele belangrijke factoren voor het bepalen van de juiste datacompressie zijn gewenste kwaliteit, bestandsgrootte en snelheid}
}

\newglossaryentry{bit}
{
	name={bit},
	description={Zoals de naam bit, kort voor binary digit, suggereert kan een bit beschouwd worden als een binair signaal. Een bit wordt beschouwd als de kleinste eenheid voor dataopslag. Meer informatie rond bestandsgrootte en dataopslag worden besproken in hoofdstuk \ref{ch:literatuurstudie}, deel \ref{sec:bestandsgrootte-dataopslag}},
	plural={bits}
}

\newglossaryentry{dna-compressie}
{
	name={DNA compressie},
	description={Het menselijke DNA kan digitaal voorgesteld worden door een lange lijst van 5 verschillende karakters, gekend als basen. Deze digitale voorstelling bestaat uit meer dan 3 miljard van deze basen (\cite{dodanaugent2011}). DNA compressie bestaat er uit deze reeks van basen zo efficiënt mogelijk op te slaan zodanig dat performante bewerkingen mogelijk zijn met een zo klein mogelijke bestandsgrootte}
}

\newglossaryentry{afbeeldingscompressie}
{
	name={afbeeldingscompressie},
	description={Afbeeldingscompressie bestaat er uit een afbeelding in zo weinig mogelijk aantal bits op te slaan terwijl een aanvaardbare kwaliteit behouden blijft. Dit kan zowel via lossless als lossy algoritmes. Afbeeldingscompressie wordt uitgebreid besproken in hoofdstuk \ref{ch:afbeeldingscompressie}}
}

\newglossaryentry{videocompressie}
{
	name={videocompressie},
	description={Videocompressie bestaat er uit een videobestand in zo weinig mogelijk aantal bits op te slaan terwijl een accapteerbare kwaliteit behouden blijft. Dit kan zowel via lossless als lossy algoritmes. videcompressie wordt uitgebreid besproken in hoofdstuk \ref{ch:videocompressie}}
}

\newglossaryentry{codec}
{
	name={codec},
	description={De coder-decoder. Binnen datacompressie betekent codec de gebruikte techniek om een bestand te comprimeren. De codec is dus is de technologie verantwoordelijk voor het encoden en decoden van een bestand volgens een bapaald compressiealgoritme. BV; H.265},
	plural={codecs}
}

\newglossaryentry{afbeeldingsformaat}
{
	name={afbeeldingsformaat},
	description={Een afbeeldingsformaat bevat alle gegevens voor het digitaal opslaan van een afbeelding. De vier gekendste categoriën van afbeeldingsformaten zijn: raster, vector, compound en stereo formaten},
	plural={afbeeldingsformaten}
}

\newglossaryentry{container}
{
	name={container},
	description={Binnen datacompressie kan een container vrijwel letterlijk vertaald worden. Het is een verpakking voor alle data die men opslaat en metadata. onder andere de codec plaatst data in deze container. Wanneer er gesproken wordt over bestandsextensies wordt vaak de container bedoeld. Bv; MP4 }
}

\newglossaryentry{jpeg}
{
	name={JPEG},
	description={Ook gekend als JPG. JPEG is een afkorting voor Joint Photographic Experts Group. JPEG is een bestandsformaat voor het opslaan van digitale afbeeldingen via lossy compressie. Afbeeldingscompressie en JPEG worden uitgebreid besproken in hoofdstuk \ref{ch:afbeeldingscompressie}}
}

\newglossaryentry{jpeg2000}
{
	name={JPEG2000},
	description={Ook gekend als JPEG2K. JPEG2000 is een bestandsformaat voor het opslaan van digitale afbeeldingen gemaakt als opvolger van JPEG. Net zoals JPEG maakt het gebruik van lossy compressie. Afbeeldingscompressie en JPEG2000 worden uitgebreid besproken in hoofdstuk \ref{ch:afbeeldingscompressie}}
}

\newglossaryentry{png}
{
	name={PNG},
	description={PNG is een afkorting voor Portable Network Graphics. PNG is een bestandsformaat voor het opslaan van digitale afbeeldingen. PNG maakt gebruik van lossless compressie. Afbeeldingscompressie en PNG worden uitgebreid besproken in hoofdstuk \ref{ch:afbeeldingscompressie}}
}

\newglossaryentry{h264-avc}
{
	name={H.264-AVC},
	description={H.264-AVC is één van de gekenste videocodecs die grootschalig gebruikt wordt. AVC is een afkorting van Advanced Video Coding. Videocompresssie en H.264-AVC worden uitgebreid besproken in hoofdstuk \ref{ch:videocompressie}}
}
% TODO: veruidelijken als deel geschreven is.

\newglossaryentry{h264-svc}
{
	name={H.264-SVC},
	description={H.264-SVC is een videocodec ontwikkelt als extensie van H264-AVC. SVC is een afkorting voor Scalable Video Coding. De nadruk bij deze extensie ligt zoals de naam sugereert op schaalbaarheid. Videocompresssie en H.264-SVC worden uitgebreid besproken in hoofdstuk \ref{ch:videocompressie}}
}
% TODO: veruidelijken als deel geschreven is.

\newglossaryentry{h265}
{
	name={H.265/HEVC},
	description={H.265/HEVC is een videocodec ontwikkelt als opvolger van H.264/AVC. Videocompresssie en H.265/HEVC worden uitgebreid besproken in hoofdstuk \ref{ch:videocompressie}}
}
% TODO: veruidelijken als deel geschreven is.

\newglossaryentry{av1}
{
	name={AV1},
	description={AV1 is een videocodec ontwikkelt als open standaard. AV1 is een afkorting voor AOMedia Video. AV1 is vooral interessant omdat het royalty free is en dus geen licentiekosten heeft. Videocompresssie en AV1 worden uitgebreid besproken in hoofdstuk \ref{ch:videocompressie}}
}
% TODO: veruidelijken als deel geschreven is.

\newglossaryentry{open-source}
{
	name={open source},
	description={Als een programmeer project open source is wilt dit zeggen dat de broncode raadpleegbaar is. Dit wil echter niet gegarandeerd zeggen dat het software programma gratis is in gebruik of de code zomaar aangepast mag worden. Dit hangt af van de licentie.}
}

\newglossaryentry{lossless}
{
	name={lossless},
	description={Binnen datacompressie slaat lossless compressie op het comprimeren van een bestand zonder kwaliteitsverlies. In het geval van video's en afbeeldingen wilt dit zeggen dat een bestand gecomprimeerd met een lossless algoritme identiek is aan het origineel}
}

\newglossaryentry{lossy}
{
	name={lossy},
	description={Binnen datacompressie slaat lossy compressie op het comprimeren van een bestand met kwaliteitsverlies voor het besparen van data. In het geval van video's en afbeeldingen wilt dit zeggen dat een bestand gecomprimeerd met een lossy algoritme een significante hoeveelheid aan data en kwaliteit kan verliezen in vergelijking met het originele bestand}
}

\newglossaryentry{afbeeldingsevaluatietool}
{
	name={afbeeldingsevaluatietool},
	description={Een applicatie gemaakt voor het voeren van een objectief of subjectief onderzoek naar afbeeldingskwaliteit. De mogelijke soorten tools worden verder besproken in hoofdstuk \ref{ch:kwaliteit}. Een subjectieve afbeeldingsevaluatietool werd gebouwd voor deze paper en wordt verder besproken in hoofdstuk \ref{ch:onderzoek}}
}

\newglossaryentry{binaire-voorstelling-bestandsgrootte}
{
	name={Binaire voorstelling},
	description={todo}
}
% TODO

\newglossaryentry{si-voorstelling-bestandsgrootte}
{
	name={SI voorstelling},
	description={todo}
}
% TODO

\newglossaryentry{ieee}
{
	name={IEEE},
	description={todo}
}
% TODO

\newglossaryentry{clustergrootte}
{
	name={clustergrootte},
	description={todo}
}
% TODO

\newglossaryentry{cluster}
{
	name={cluster},
	description={todo},
	plural={clusters}
}
% TODO

\newglossaryentry{byte}
{
	name={byte},
	description={todo},
	plural={bytes}
}
% TODO

\newglossaryentry{leestijd}
{
	name={leestijd},
	description={todo},
	plural={leestijden}
}
% TODO

\newglossaryentry{bandbreedte}
{
	name={bandbreedte},
	description={todo}
}
% TODO

\newglossaryentry{prefix-code}
{
	name={prefix code},
	description={todo},
	plural={prefix codes}
}
% TODO

\newglossaryentry{huffman-coding}
{
	name={Huffman coding},
	description={todo}
}
% TODO

\newglossaryentry{frequency-based}
{
	name={frequency-based},
	description={todo}
}
% TODO

\newglossaryentry{compressie-algoritme}
{
	name={compressie-algoritme},
	description={todo},
	plural={compressie-algoritmen}
}
% TODO

\newglossaryentry{lookup-table}
{
	name={lookup table},
	description={todo}
}
% TODO

\newglossaryentry{dictionary-coding}
{
	name={dictionary coding},
	description={todo}
}
% TODO

\newglossaryentry{prefix-coding}
{
	name={prefix coding},
	description={todo}
}
% TODO

\newglossaryentry{use-case}
{
	name={use case},
	description={todo},
	plural={use cases}
}
% TODO

\newglossaryentry{rle-long}
{
	name={run length encoding},
	description={todo}
}
% TODO

\newglossaryentry{rle-short}
{
	name={RLE},
	description={Kort voor \gls{rle-long}}
}

\newglossaryentry{ascii}
{
	name={ASCII},
	description={todo}
}
% TODO

\newglossaryentry{encoding}
{
	name={encoding},
	description={todo}
}
% TODO

\newglossaryentry{decoding}
{
	name={decoding},
	description={todo}
}
% TODO

\newglossaryentry{meta-data}
{
	name={metadata},
	description={todo}
}

\newglossaryentry{webp}
{
	name={WebP},
	description={todo}
}

\newglossaryentry{heif}
{
	name={HEIF},
	description={todo}
}

\newglossaryentry{ps}
{
	name={Adobe Photoshop},
	description={todo}
}

\newglossaryentry{css}
{
	name={CSS},
	description={todo}
}

\newglossaryentry{minifyen}
{
	name={minifyen},
	description={todo}
}

\newglossaryentry{render}
{
	name={render},
	description={todo}
}

\newglossaryentry{raw}
{
	name={RAW},
	description={todo}
}

\newglossaryentry{compressietool}
{
	name={datacompressietool},
	description={todo}
}

\newglossaryentry{github}
{
	name={GitHub},
	description={todo}
}

\newglossaryentry{hosting}
{
	name={hosting},
	description={todo}
}
% TODO

\newglossaryentry{php}
{
	name={PHP},
	description={todo}
}
% TODO

\newglossaryentry{sql}
{
	name={SQL},
	description={todo}
}
% TODO

\newglossaryentry{xampp}
{
	name={XAMPP},
	description={todo}
}

\newglossaryentry{jquery}
{
	name={JQuery},
	description={todo}
}

\newglossaryentry{drift}
{
	name={Drift},
	description={todo}
}

\newglossaryentry{bootstrap}
{
	name={Bootstrap},
	description={todo}
}
% TODO

\newglossaryentry{js}
{
	name={JavaScript},
	description={todo}
}
% 

\newglossaryentry{yt}
{
	name={YouTube},
	description={todo}
}
% TODO

\newglossaryentry{plug-in}
{
	name={plug-in},
	description={todo},
	plural={plug-ins}
}
% TODO

\newglossaryentry{artefact}
{
	name={artefact},
	description={todo},
	plural={artefacten}
}
% TODO

\newglossaryentry{wavelet}
{
	name={wavelet},
	description={todo},
	plural={wavelets}
}
% TODO

\newglossaryentry{ai}
{
	name={artificiële intelligentie},
	description={todo}
}
% TODO

\newglossaryentry{nef}
{
	name={NEF},
	description={todo}
}
% TODO

\newglossaryentry{dng}
{
	name={DNG},
	description={todo}
}
% TODO

\newglossaryentry{w3c}
{
	name={W3C},
	description={todo}
}
% TODO

\newglossaryentry{gif}
{
	name={GIF},
	description={todo}
}
% TODO

\newglossaryentry{backwards-compatible}
{
	name={backwards compatible},
	description={todo}
}
% TODO

\newglossaryentry{rgb}
{
	name={RGB},
	description={todo}
}
% TODO

\newglossaryentry{cmyk}
{
	name={CMYK},
	description={todo}
}
% TODO

\newglossaryentry{jpeg-exif}
{
	name={JPEG/Exif},
	description={todo}
}
% TODO

\newglossaryentry{jpeg-jfif}
{
	name={JPEG/JFIF},
	description={todo}
}
% TODO

\newglossaryentry{extensie}
{
	name={bestandsextensie},
	description={todo},
	plural={bestandsextensies}
}
% TODO

\newglossaryentry{dct}
{
	name={DCT},
	description={todo}
}
% TODO

\newglossaryentry{wtcq}
{
	name={WTCQ},
	description={todo}
}
% TODO

\newglossaryentry{iso}
{
	name={ISO},
	description={todo}
}
% TODO

\newglossaryentry{jpf}
{
	name={JPF},
	description={todo}
}
% TODO

\newglossaryentry{intra-frame}
{
	name={intra-frame},
	description={todo},
	plural={intra-frames}
}
% TODO

\newglossaryentry{decoder}
{
	name={decoder},
	description={todo},
	plural={decoders}
}
% TODO

\newglossaryentry{encoder}
{
	name={encoder},
	description={todo},
	plural={encoders}
}
% TODO

\newglossaryentry{heic}
{
	name={HEIC},
	description={todo},
}
% TODO

\newglossaryentry{on-premise}
{
	name={on premise},
	description={todo},
}
% TODO

\newglossaryentry{vector}
{
	name={vector},
	description={todo},
}
% TODO

\newglossaryentry{cms}
{
	name={CMS},
	description={todo},
	plural={CMS'en}
}
% TODO

\newglossaryentry{wordpress}
{
	name={WordPress},
	description={todo}
}
% TODO

\newglossaryentry{raster}
{
	name={raster},
	description={todo}
}
% TODO

\newglossaryentry{lbhuffman}
{
	name={lbhuffman},
	description={todo}
}
% TODO

\newglossaryentry{lbrlea}
{
	name={lbrlea},
	description={todo}
}
% TODO

\newglossaryentry{lbrle}
{
	name={lbrle},
	description={todo}
}
% TODO

\newglossaryentry{html}
{
	name={HTML},
	description={todo}
}
% TODO

\newglossaryentry{bootsrap}
{
	name={Bootstrap},
	description={todo}
}
% TODO

\newglossaryentry{library}
{
	name={library},
	description={todo},
	plural={Libraries}
}
% TODO

\newglossaryentry{string}
{
	name={string},
	description={todo},
	plural={strings}
}
% TODO

\newglossaryentry{regex}
{
	name={regex},
	description={todo}
}
% TODO

\newglossaryentry{recursieve-functie}
{
	name={recursieve functie},
	description={todo}
}
% TODO

\newglossaryentry{array}
{
	name={array},
	description={todo}
}
% TODO

\newglossaryentry{json}
{
	name={json},
	description={todo}
}
% TODO

\newglossaryentry{rmse}
{
	name={rmse},
	description={Root mean square error, bespsroken in deel \ref{sec:kwaliteit-rmse}.}
}

\newglossaryentry{ssim}
{
	name={SSIM},
	description={Structural similarity index, bespsroken in deel \ref{sec:kwaliteit-ssim}.}
}

\newglossaryentry{vdp}
{
	name={VDP},
	description={Visual difference predictor is een term voor functies die visuele verschillen tussen afbeeldingen probeert uit te drukken.}
}

\newglossaryentry{dxomark}
{
	name={DxOMark},
	description={Een erkende instelling dat de kwaliteit van, voornamelijk smartphone, camera's beoordeeld. DxOMark wordt verder toegelicht in deel \ref{sec:kwaliteit-dxomark}.}
}

\newglossaryentry{maximum-compression}
{
	name={Maximum Compression},
	description={Een erkende instelling dat de compressieratio van lossles compressie-algoritmes beoordeeld. Maximum Compression wordt verder toegelicht in deel \ref{sec:kwaliteit-maximum-compression}.}
}

\newglossaryentry{compressieratio}
{
	name={compressieratio},
	description={TODO}
}

\newglossaryentry{r}
{
	name={R},
	description={TODO}
}

