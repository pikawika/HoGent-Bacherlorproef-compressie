\chapter{Afbeeldingscompressie}
\label{ch:afbeeldingscompressie}

\Gls{afbeeldingscompressie} is een subdomein van \gls{datacompressie}. \Gls{afbeeldingscompressie} bestaat er uit een afbeelding in zo weinig mogelijk aantal \glspl{bit} op te slaan terwijl een aanvaardbare kwaliteit behouden blijft. Dit kan zowel via \gls{lossless} als \gls{lossy} \glspl{compressie-algoritme}.

\Gls{afbeeldingscompressie} is zeer belangrijk voor de snelheid en schaalbaarheid van IT-projecten alsook voor de gebruikerservaring. Bijna alle IT-projecten bevatten afbeeldingen, dit is zeker bij websites het geval.

In een officiële blog post van Google, \citetitle{googleinternetspeed} (\cite{googleinternetspeed}), wordt besproken dat een efficiënte webpagina steeds onder de twee seconden zou moeten geladen kunnen worden. In diezelfde post wordt zelfs aangehaald dat binnen een halve second de gebruiker reeds inhoud van de webpagina zou moeten zien.

Uit een nog recenter onderzoek door Akamai, \citetitle{akamaiinternetspeed} (\cite{akamaiinternetspeed}), blijkt dat bij een laadtijd van meer dan drie seconden op een mobiele webpagina meer dan de helft van de bezoekers de webpagina verlaat.

Afbeeldingen zijn meestal de grootste bestanden die bij het laden van een webpagina gedownload moeten worden. Het is dan ook met een juiste keuze aan \gls{afbeeldingscompressie} dat het meeste tijd voor de bezoeker kan gespaard worden. 

Dit hoofdstuk licht toe welke soorten \gls{afbeeldingscompressie} er bestaan. Ook enkele \glspl{afbeeldingsformaat} zullen besproken worden samen met hun voordelen en nadelen. De mogelijke problemen en oplossingen bij de implementatie van nieuwe generatie \glspl{afbeeldingsformaat} zal alsook besproken worden.

Hoofdstuk \ref{ch:kwaliteit} bespreekt hoe de kwaliteit van een \gls{afbeeldingsformaat} objectief en subjectief beoordeeld kan worden. In hoofdstuk \ref{ch:onderzoek} wordt voor een bepaalde \gls{use-case} een geschikt \gls{afbeeldingsformaat} gezocht aan de hand van een subjectief onderzoek met een voor deze bachelorproef geschreven \gls{afbeeldingsevaluatietool}.


\section{Afbeeldingsformaten}
\label{sec:afbeeldingscompressie-afbeeldingsformaten}

Het nemen van een foto met een digitale camera komt overeen met het openstellen van de beeldsensor aan licht voor een bepaalde duur (sluitertijd). De gegevens die gedurende die tijd waargenomen worden, kunnen direct verwerkt worden en gecomprimeerd opgeslagen worden als bijvoorbeeld een \gls{jpeg}. Dit is hoe de meeste smartphone camera’s te werk gaan. Bij vele, voornamelijk professionele, toestellen kan ingesteld worden dat er geen gecomprimeerd \gls{afbeeldingsformaat} gebruikt moet worden maar maar een \gls{raw} \gls{afbeeldingsformaat}. 

Hieronder worden enkele \glspl{afbeeldingsformaat} verder toegelicht. Het is belangrijk om te weten dat dit niet de enige \glspl{afbeeldingsformaat} zijn die bestaan. De lijst van \glspl{afbeeldingsformaat} blijft groeien en bestaande \glspl{afbeeldingsformaat} kunnen extensies krijgen om gekende problemen als bepaalde \glspl{artefact} tegen te gaan. 

Een mogelijke manier om deze artefacten tegen te gaan is het gebruik ven een andere \gls{wavelet} zoals besproken in \citetitle{inproceedings} (\cite{inproceedings}). Recente doorbraken binnen \gls{ai} maken het zelfs mogelijk op een nog meer dynamische manier \glspl{artefact} in \glspl{afbeeldingsformaat} tegen te gaan zoals besproken in \citetitle{jpegartefactereductionai} (\cite{jpegartefactereductionai}).


\subsection{RAW}
\label{sec:afbeeldingscompressie-raw}

Een \gls{raw} \gls{afbeeldingsformaat} bevat alle ruwe, onbewerkte en ongecomprimeerde gegevens die de beeldsensor heeft vastgelegd. In een \gls{raw} \gls{afbeeldingsformaat} wordt ook tal van \gls{meta-data}  bijgehouden zoals de gebruikte camera en lens, hun instellingen… De bestandsgrootte van een \gls{raw} bestand is hierdoor aanzienlijk. 

\Gls{raw} is geen afkorting noch een echt \glspl{afbeeldingsformaat} zoals \gls{jpeg} of \gls{png} maar een benaming voor een groep van \glspl{afbeeldingsformaat} die voldoen aan de benoemde eigenschappen. Het effectieve \gls{afbeeldingsformaat} kan verschillen van merk tot merk en zelfs van toestel tot toestel. Zo zijn de \gls{raw} bestanden gebruikt voor het onderzoek in hoofdstuk \ref{ch:onderzoek} afkomstig van een Nikon toestel en zijn ze opgeslagen in het \gls{nef} \gls{afbeeldingsformaat}.

Hoewel er reeds voorstellen zijn gedaan voor een open \gls{raw} standaard om bewerkingen makkelijk te maken zoals \gls{dng} van Adobe is er tot op heden een grote diversiteit aan \gls{raw} \glspl{afbeeldingsformaat} te vinden. Dit vormt binnen \gls{afbeeldingscompressie} en de evaluatie ervan enkele nadelen. Doordat er zo veel verschillende \gls{raw} \glspl{afbeeldingsformaat} zijn en dus veel uiteenlopende licenties en rechten, is het een uitdaging een \gls{compressie-algoritme} voor een \gls{afbeeldingsformaat} te maken dat alle \gls{raw} \glspl{afbeeldingsformaat} ondersteund als input. 

Starten van een \gls{raw} \gls{afbeeldingsformaat} voor het evalueren van \gls{afbeeldingscompressie} is echter wel aangeraden aangezien zelfs het gebruik van een \gls{lossless} \gls{afbeeldingsformaat} voor verlies van \gls{meta-data} kan zorgen zoals eerder besproken. Het weglaten van deze \gls{meta-data} kan onderdeel zijn van het gekozen \gls{afbeeldingsformaat} en bijhorend \gls{compressie-algoritme}. Als deze \gls{meta-data} niet inbegrepen is in het inputbestand wordt het weglaten ervan niet gerepresenteerd in de eindscore wat voor een vals beeld kan zorgen.

\subsection{PNG}
\label{sec:afbeeldingscompressie-png}

Portable Network Graphics is een \gls{lossless} \gls{afbeeldingsformaat}. Het is ontwikkeld door de Portable Network Graphics Development Group met een eerste beta in 1995, een draft versie voor \gls{w3c} eind 1995, een officiële \gls{w3c} voorstelling op 1 juli 1996 en goedgekeurd als \gls{w3c} aanbeveling op 1 oktober 1996. Datums uit \citetitle{pnghistory} (\cite{pnghistory}).

\Gls{png} was gemaakt als vervanger van het toen veelgebruikte \gls{gif} \gls{afbeeldingsformaat} dat net zoals vele andere \gls{compressie-algoritme} een nachtmerrie van licenties en patenten aan het worden was. 

\Gls{png} had als doel een \gls{afbeeldingsformaat} te worden dat zeer flexibel is, gemakkelijk te gebruiken is op het internet en allerlei soorten afbeeldingen ondersteund. Meer dan 20 jaar later slaagt het daar nog altijd in.

\subsubsection{PNG: werking}
\label{sec:afbeeldingscompressie-png-werking}

TODO
%TODO: afbeeldingcompressie


\subsubsection{PNG: voordelen}
\label{sec:afbeeldingscompressie-png-voordelen}

Het \gls{png} \gls{afbeeldingsformaat} bied tal van voordelen ten opzichte van zijn voorgangers en \gls{lossy} tegenstanders. Enkele van deze voordelen zijn:

\begin{itemize}
	\item Beschikt over een alpha kanaal waardoor doorzichtigheid meegegeven kan worden als een getal tussen 0 (volledig doorzichtig) en 100 (geen doorzichtigheid).
	
	\item Aanzien als een van de standaarden voor \gls{lossless} \glspl{afbeeldingsformaat} waardoor het een goede support heeft overheen verschillende hardware en software.
	
	\item \Gls{lossless} \gls{afbeeldingsformaat} waardoor er geen kwaliteit verloren gaat.
	
	\item Goede uitbreidbaarheid waardoor \gls{meta-data} en andere randvariabelen aan een \gls{png} bestand kunnen toegevoegd worden terwijl het bestand  \gls{backwards-compatible} blijft.
	
	\item Een keuze uit meer dan 16 miljoen kleuren dankzij het \gls{rgb} kleurenprofiel met alpha kanaal. Een groot contrast tegenover \gls{gif} dat maar 256 kleuren ondersteund. 
\end{itemize}

\subsubsection{PNG: nadelen}
\label{sec:afbeeldingscompressie-png-nadelen}

\Gls{png} heeft echter ook enkele minpunten, voornamelijk te weiden aan het feit dat \gls{png} ontwikkeld is om te gebruiken op het internet.

\begin{itemize}
	\item Geen ondersteuning voor kleuromgevingen als \gls{cmyk}.
	
	\item Geen standaard ondersteuning voor geanimeerde beelden.
	
	\item Grote bestandsgrootte door zijn \gls{lossless} eigenschap.
\end{itemize}

\subsection{JPEG}
\label{sec:afbeeldingscompressie-jpeg}

Joint Photographic Experts Group is technisch gezien geen \gls{afbeeldingsformaat} maar een \gls{codec}. Het is een \gls{compressie-algoritme} dat zowel \gls{lossless} als \gls{lossy} te werk kan gaan. Wanneer gesproken wordt over het \gls{jpeg} \gls{afbeeldingsformaat} verwijst dit meestal naar \gls{jpeg-exif} of \gls{jpeg-jfif} welke wel een \gls{afbeeldingsformaat} zijn.

\gls{jpeg} wordt doorgaans als \gls{lossy} \gls{compressie-algoritme} gebruikt voor het opslaan van afbeeldingen waarbij een controle over bestandsgrootte en kwaliteit gewenst is. Dit is mogelijk doordat \gls{jpeg} verschillende parameters ondersteund om de werking van het \gls{compressie-algoritme} te beïnvloeden en dus ook het uiteindelijk bestand zijn kwaliteit en bestandsgrootte.

De ontwikkeling van het \gls{jpeg} \gls{compressie-algoritme} is begonnen in 1986 en de \gls{jpeg} standaard is gemaakt in 1992. Deze bestaat uit 7 delen met de laatste officiële revisie in 1994. Uiteraard zijn er tal van uitbreidingen (of extensies zoals deel 3 van de ISO/IEC 10918 standaard ze benoemd) gemaakt tot op heden. Datums overgenomen van de officiële \gls{jpeg} website (\cite{jpegorg}). 

\gls{jpeg} is ook gekend onder de kortere vorm JPG omdat dit de extensie is die het meest gebruikt wordt voor de \gls{jpeg} \gls{codec}. Dit was omdat in oudere versies van het Windows besturingssysteem, zoals bijvoorbeeld Windows 98, een \gls{extensie} maximaal drie karakters lang mocht zijn. In de huidige versies van Windows is deze beperking echter niet meer actief waardoor de JPG en \gls{jpeg} \glspl{extensie} door elkaar gebruikt kunnen worden.

\subsubsection{JPEG: werking}
\label{sec:afbeeldingscompressie-jpeg-werking}

TODO
%TODO: afbeeldingcompressie

\subsubsection{JPEG: voordelen}
\label{sec:afbeeldingscompressie-jpeg-voordelen}

\Gls{jpeg} is één van de meest gebruikte \gls{lossy} \gls{compressie-algoritme} voor afbeeldingen en heeft onder andere daarom enkele belangrijke voordelen zoals:

\begin{itemize}
	\item Uitstekende support overheen verschillende hardware en software.
	
	\item Door de mogelijkheid om als \gls{lossy} \gls{compressie-algoritme} te werken kan \gls{jpeg} een enorm kleine bestandsgrootte aannemen afhankelijk van de instellingen.
	
	\item \Gls{jpeg} werkt snel wat samen met een kleinere bestandsgrootte de mogelijkheid creëert om meer foto's per second te verwerken. Op deze manier kan een digitale camera meer opnames maken in burst wanneer er voor \gls{jpeg-exif} is gekozen als \gls{afbeeldingsformaat} in plaats van een \gls{raw} \gls{afbeeldingsformaat}. 
\end{itemize}

\subsubsection{JPEG: nadelen}
\label{sec:afbeeldingscompressie-jpeg-nadelen}

\begin{itemize}
	\item Door de \gls{lossy} eigenschap aan de hand van clustering kunnen allerlei vormen van \glspl{artefact} voorkomen.
	
	\item Geen mogelijkheid voor doorzichtigheid.
	
	\item Onnatuurlijke afbeeldingen zoals logo's zijn zeer gevoelig aan het verlies van scherpe lijnen en het ontstaan van \glspl{artefact}.
\end{itemize}

\subsection{JPEG2000}
\label{sec:afbeeldingscompressie-jpeg2000}

TODO
%TODO: afbeeldingcompressie

\subsubsection{JPEG2000: werking}
\label{sec:afbeeldingscompressie-jpeg2000-werking}

TODO
%TODO: afbeeldingcompressie


\subsubsection{JPEG2000: voordelen}
\label{sec:afbeeldingscompressie-jpeg2000-voordelen}

TODO
%TODO: afbeeldingcompressie

\subsubsection{JPEG2000: nadelen}
\label{sec:afbeeldingscompressie-jpeg2000-nadelen}

TODO
%TODO: afbeeldingcompressie

\subsection{WEBP}
\label{sec:afbeeldingscompressie-webp}

TODO
%TODO: afbeeldingcompressie

\subsubsection{WEBP: werking}
\label{sec:afbeeldingscompressie-webp-werking}

TODO
%TODO: afbeeldingcompressie


\subsubsection{WEBP: voordelen}
\label{sec:afbeeldingscompressie-webp-voordelen}

TODO
%TODO: afbeeldingcompressie

\subsubsection{WEBP: nadelen}
\label{sec:afbeeldingscompressie-webp-nadelen}

TODO
%TODO: afbeeldingcompressie

\subsection{HEIC}
\label{sec:afbeeldingscompressie-heic}

TODO
%TODO: afbeeldingcompressie

\subsubsection{HEIC: werking}
\label{sec:afbeeldingscompressie-heic-werking}

TODO
%TODO: afbeeldingcompressie


\subsubsection{HEIC: voordelen}
\label{sec:afbeeldingscompressie-heic-voordelen}

TODO
%TODO: afbeeldingcompressie

\subsubsection{HEIC: nadelen}
\label{sec:afbeeldingscompressie-heic-nadelen}

TODO
%TODO: afbeeldingcompressie

\section{De juiste keuze}
\label{sec:afbeeldingscompressie-keuze}

TODO
%TODO: afbeeldingcompressie

\subsection{Lossless of lossy afbeeldingscompressie}
\label{sec:afbeeldingscompressie-lossless-of-lossy}

TODO
%TODO: afbeeldingcompressie

\subsection{Ondersteuning}
\label{sec:afbeeldingscompressie-ondersteuning}

TODO
%TODO: afbeeldingcompressie

\section{Implementatie}
\label{sec:afbeeldingscompressie-implementatie}

TODO
%TODO: afbeeldingcompressie