\chapter{Afbeeldingscompressie}
\label{ch:afbeeldingscompressie}

\Gls{afbeeldingscompressie} is een subdomein van \gls{datacompressie}. \Gls{afbeeldingscompressie} bestaat er uit een afbeelding in zo weinig mogelijk aantal \glspl{bit} op te slaan terwijl een aanvaardbare kwaliteit behouden blijft. Dit kan zowel via \gls{lossless} als \gls{lossy} \glspl{compressie-algoritme}.

\Gls{afbeeldingscompressie} is zeer belangrijk voor de snelheid en schaalbaarheid van IT-projecten alsook voor de gebruikerservaring. Bijna alle IT-projecten bevatten afbeeldingen, dit is zeker bij websites het geval.

In een officiële blog post van Google, \citetitle{googleinternetspeed} (\cite{googleinternetspeed}), wordt besproken dat een efficiënte webpagina steeds onder de twee seconden zou moeten geladen kunnen worden. In diezelfde post wordt zelfs aangehaald dat binnen een halve second de gebruiker reeds inhoud van de webpagina zou moeten zien.

Uit een nog recenter onderzoek door Akamai, \citetitle{akamaiinternetspeed} (\cite{akamaiinternetspeed}), blijkt dat bij een laadtijd van meer dan drie seconden op een mobiele webpagina meer dan de helft van de bezoekers de webpagina verlaat.

Afbeeldingen zijn meestal de grootste bestanden die bij het laden van een webpagina gedownload moeten worden. Het is dan ook met een juiste keuze aan \gls{afbeeldingscompressie} dat het meeste tijd voor de bezoeker kan gespaard worden. 

Dit hoofdstuk licht toe welke soorten \gls{afbeeldingscompressie} er bestaan. Ook enkele \glspl{afbeeldingsformaat} zullen besproken worden samen met hun voordelen en nadelen. De mogelijke problemen en oplossingen bij de implementatie van nieuwe generatie \glspl{afbeeldingsformaat} zal alsook besproken worden.

Hoofdstuk \ref{ch:kwaliteit} bespreekt hoe de kwaliteit van een \gls{afbeeldingsformaat} objectief en subjectief beoordeeld kan worden. In hoofdstuk \ref{ch:onderzoek} wordt voor een bepaalde \gls{use-case} een geschikt \gls{afbeeldingsformaat} gezocht aan de hand van een subjectief onderzoek met een \gls{afbeeldingsevaluatietool}.

\section{Afbeeldingsformaten}
\label{sec:afbeeldingscompressie-afbeeldingsformaten}

TODO
%TODO: afbeeldingcompressie

\subsection{RAW}
\label{sec:afbeeldingscompressie-raw}

TODO
%TODO: afbeeldingcompressie

\subsection{PNG}
\label{sec:afbeeldingscompressie-png}

TODO
%TODO: afbeeldingcompressie

\subsection{JPEG}
\label{sec:afbeeldingscompressie-jpeg}

TODO
%TODO: afbeeldingcompressie

\subsection{JPEG2000}
\label{sec:afbeeldingscompressie-jpeg2000}

TODO
%TODO: afbeeldingcompressie

\subsection{WEBP}
\label{sec:afbeeldingscompressie-webp}

TODO
%TODO: afbeeldingcompressie

\subsection{HEIC}
\label{sec:afbeeldingscompressie-heic}

TODO
%TODO: afbeeldingcompressie

\section{De juiste keuze}
\label{sec:afbeeldingscompressie-keuze}

TODO
%TODO: afbeeldingcompressie

\subsection{Lossless of lossy afbeeldingscompressie}
\label{sec:afbeeldingscompressie-lossless-of-lossy}

TODO
%TODO: afbeeldingcompressie

\subsection{Ondersteuning}
\label{sec:afbeeldingscompressie-ondersteuning}

TODO
%TODO: afbeeldingcompressie

\section{Implementatie}
\label{sec:afbeeldingscompressie-implementatie}

TODO
%TODO: afbeeldingcompressie