%==================
%% Samenvatting
%==================

% De "abstract" of samenvatting is een kernachtige (~ 1 blz. voor een - op het einde
% thesis) synthese van het document.
%
% Deze aspecten moeten zeker aan bod komen:
% - Context: waarom is dit werk belangrijk?
% - Nood: waarom moest dit onderzocht worden?
% - Taak: wat heb je precies gedaan?
% - Object: wat staat in dit document geschreven?
% - Resultaat: wat was het resultaat?
% - Conclusie: wat is/zijn de belangrijkste conclusie(s)?
% - Perspectief: blijven er nog vragen open die in de toekomst nog kunnen
%    onderzocht worden? Wat is een mogelijk vervolg voor jouw onderzoek?
%
% LET OP! Een samenvatting is GEEN voorwoord!


\chapter*{Samenvatting}
\label{ch:samenvatting}

\Gls{datacompressie}: een fundamenteel onderdeel van de IT-wereld waar weinig belanghebbenden een basiskennis van hebben. Waarom is dat? Schrikken de complexe en zeer uitgebreide papers reeds geschreven over dit onderwerp geïnteresseerden af? Is het een te complex onderwerp om te voorzien in meer IT-gerelateerde opleidingen? Staat \gls{datacompressie} stil in de tijd dat standaarden als het \glspl{afbeeldingsformaat} \gls{jpeg} al meer dan twintig jaar het bekendste \gls{afbeeldingsformaat} is? Deze bachelorproef tracht een antwoord te geven op die vragen en de tal van andere onderzoeksvragen besproken in deel \ref{sec:onderzoeksvragen}. 

Dit document is gemaakt met een eenvoudig visie: de nodige basiskennis over het ontstaan van \gls{datacompressie}, de werking van enkele \glspl{compressie-algoritme}, tal van \glspl{afbeeldingsformaat} en video \glspl{codec} en de manieren voor het evalueren van compressiekwaliteit toe te lichten. Na het lezen van deze bachelorproef zal de lezer meer stilstaan bij de keuze voor een geschikt \gls{compressie-algoritme}.

Er is zowel een proof of concept \gls{compressietool} als een uitgebreide \gls{afbeeldingsevaluatietool} geschreven voor de bachelorproef die gratis zijn in gebruik en \gls{open-source} toegankelijk zijn op de \gls{github} repository van deze bachelorproef\urlcite{githubbachelorproef}. Er wordt dieper ingegaan op het ontstaan, de werking en de voordelen en nadelen van volgende \glspl{afbeeldingsformaat}: \gls{png} | \gls{jpeg} | \gls{jpeg2000} | \gls{webp} | \gls{heif}. Hetzelfde wordt gedaan voor de volgende video \glspl{codec}: \gls{h264-avc} | \gls{h264-svc} | \gls{h265} | \gls{av1}.

Er wordt een subjectief onderzoek gevoerd aan de hand van de eerder benoemde \gls{afbeeldingsevaluatietool}. Dit geeft samen met de theoretische kennis die zal verkregen worden door de bachelorproef een antwoord op de hoofdonderzoeksvraag van deze bachelorproef: Waarom moet er stilgestaan worden bij het gebruiken van \glspl{compressie-algoritme}, hoe kies je een geschikt \gls{compressie-algoritme} voor een bepaalde \gls{use-case} en hoe implementeer je dit het best?

Het volledige verloop van deze bachelorproef is beschreven in deel \ref{sec:opzet-bachelorproef}.