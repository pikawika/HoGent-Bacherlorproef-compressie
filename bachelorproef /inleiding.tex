%%=============================================================================
%% Inleiding
%%=============================================================================

\chapter{Inleiding}
\label{ch:inleiding}

Datacompressie en het achterliggende compressie idee is niets nieuw. Integendeel, het is één van de oudste zaken binnen IT dat tot op heden van fundamenteel belang is binnen zowat alle IT-toepassingen. Door de databesparing werkt alles niet alleen veelvouden sneller en goedkoper, maar worden bepaalde zaken die voorheen onmogelijk leken mogelijk. Denk hierbij bijvoorbeeld aan recente doorbraken binnen DNA-compressie dat het mogelijk maken steeds meer onderzoeken met betrekking tot het menselijk genoom uit te voeren.

\section{\IfLanguageName{dutch}{Probleemstelling}{Problem Statement}}
\label{sec:probleemstelling}


%TODO: verder 
%De inleiding moet de lezer net genoeg informatie verschaffen om het onderwerp te begrijpen en in te zien waarom de onderzoeksvraag de moeite waard is om te onderzoeken. In de inleiding ga je literatuurverwijzingen beperken, zodat de tekst vlot leesbaar blijft. Je kan de inleiding verder onderverdelen in secties als dit de tekst verduidelijkt. Zaken die aan bod kunnen komen in de inleiding~\autocite{Pollefliet2011}:
%\begin{itemize}
%\item context, achtergrond
%\item afbakenen van het onderwerp
%\item verantwoording van het onderwerp, methodologie
%\item probleemstelling
%\item onderzoeksdoelstelling
%\item onderzoeksvraag
%\item \ldots
%\end{itemize}

\section{\IfLanguageName{dutch}{Onderzoeksvragen}{Research question}}
\label{sec:onderzoeksvragen}


%TODO: verder 
%Wees zo concreet mogelijk bij het formuleren van je onderzoeksvraag. Een onderzoeksvraag is trouwens iets waar nog niemand op dit moment een antwoord heeft (voor zover je kan nagaan). Het opzoeken van bestaande informatie (bv. ``welke tools bestaan er voor deze toepassing?'') is dus geen onderzoeksvraag. Je kan de onderzoeksvraag verder specifiëren in deelvragen. Bv.~als je onderzoek gaat over performantiemetingen, dan 

\section{\IfLanguageName{dutch}{Onderzoeksdoelstelling}{Research objective}}
\label{sec:onderzoeksdoelstelling}

%TODO: verder 
%Wat is het beoogde resultaat van je bachelorproef? Wat zijn de criteria voor succes? Beschrijf die zo concreet mogelijk. Gaat het bv. om een proof-of-concept, een prototype, een verslag met aanbevelingen, een vergelijkende studie, enz.

\section{\IfLanguageName{dutch}{Opzet van deze bachelorproef}{Structure of this bachelor thesis}}
\label{sec:opzet-bachelorproef}

% Het is gebruikelijk aan het einde van de inleiding een overzicht te
% geven van de opbouw van de rest van de tekst. Deze sectie bevat al een aanzet
% die je kan aanvullen/aanpassen in functie van je eigen tekst.

Deze paper zal zich in het eerste deel focussen op het toelichten van de belangrijke termen binnen datacompressie. Ook het ontstaan van datacompressie en enkele basisprincipes zal hier toegelicht worden. Enkele van deze primitieve technieken zullen aan de hand van een voorbeeld toegelicht worden.
%TODO: nog wat meer over voor die zin Lennert eet veel
 
 In het tweede zal een basis datacompressie tool geïmplementeerd worden om de theorie uit het eerste deel in praktijk te zien.
Dit deel is vooral voor technische lezers (programmeurs ed) interessant.
 %TODO: welke taal tool gemaakt is en wat deze juist doet etc

In het derde deel worden twee sub domeinen van datacompressie verder toegelicht; afbeelding- en videocompressie. Er zal dieper ingegaan worden op volgende afbeeldingcompressie mogelijkheden: JPEG (= JPG), JPEG2000 (= JPEG2K) en PNG. Voor videocompressie zullen we verder ingaan op de huidige standaard H264/AVC en SVC. Ook de opvolger H265 en open source AV1 zullen besproken worden.
%TODO: aanpassen indien andere op input copromoter

In het vierde deel wordt besproken hoe compressiemethoden binnen video en afbeelding objectief en subjectief gemeten kunnen worden en zal een subjectief onderzoek gedaan worden tussen enkele van de besproken afbeeldingscompressie methoden.
%TODO: in staat stelllen als bv prog of content zelf kiezen welk gebruiken

In het vijfde deel zullen de huidige uitdagingen van datacompressie kort toegelicht worden.
%TODO: copromoter zijn job en eerbetoon vermelden

Het laatste deel zal een objectieve conclussie vormen voor deze paper.
%TODO: lezer aanzetten meer etc

%TODO: verder en meer citen met ~\ref{ch:stand-van-zaken}
% TODO: Vul hier aan voor je eigen hoofstukken, één of twee zinnen per hoofdstuk
%In Hoofdstuk~\ref{ch:conclusie}, tenslotte, wordt de conclusie gegeven en een antwoord geformuleerd op de onderzoeksvragen. Daarbij wordt ook een aanzet gegeven voor toekomstig onderzoek binnen dit domein.