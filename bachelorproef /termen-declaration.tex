%run generate.sh en run hier
\newglossaryentry{datacompressie}
{
	name={datacompressie},
	description={De mens wilt vaak zoveel mogelijk output met zo weinig mogelijk input. Datacompressie vervult die wens binnen de computerwereld. Datacompressie draait er om zoveel mogelijk data weer te geven met zo weinig mogelijk bits}
}

\newglossaryentry{bit}
{
	name={bit},
	description={Zoals de naam bit, kort voor binary digit, suggereert kan een bit beschouwd worden als een binair signaal. Een bit wordt beschouwd als de kleinste dataeenheid voor dataopslag},
	plural={bits}
}

\newglossaryentry{dna-compressie}
{
	name={DNA compressie},
	description={Het menselijke DNA kan digitaal voorgesteld worden door een lange lijst van 5 verschillende karakters, gekend als basen. Deze digitale voorstelling bestaat uit meer dan 3 miljard van deze basen (\cite{dodanaugent2011}). DNA compressie bestaat er uit deze reeks van basen zo efficiënt mogelijk op te slaan zodanig dat performante bewerkingen mogelijk zijn met een zo klein mogelijke bestandsgrootte}
}

\newglossaryentry{afbeeldingscompressie}
{
	name={afbeeldingscompressie},
	description={Afbeeldingcompressie bestaat er uit een afbeelding in zo weinig mogelijk aantal bits op te slaan terwijl een accapteerbare kwaliteit behouden blijft. Dit kan zowel via lossless als lossy algoritmes. Afbeeldingcompressie wordt uitgebreid besproken in hoofdstuk \ref{ch:afbeeldingcompressie}}
}

\newglossaryentry{videocompressie}
{
	name={videocompressie},
	description={Videocompressie bestaat er uit een videobstand in zo weinig mogelijk aantal bits op te slaan terwijl een accapteerbare kwaliteit behouden blijft. Dit kan zowel via lossless als lossy algoritmes. videcompressie wordt uitgebreid besproken in hoofdstuk \ref{ch:videocompressie}}
}

\newglossaryentry{codec}
{
	name={codec},
	description={Binnen datacompressie betekent codec de gebruikte techniek om een bestand te comprimeren. De codec is dus is de technologie verantwoordelijk voor het encoden en decoden van een bestand volgens een bapaald compressiealgoritme. BV; H.26},
	plural={codecs}
}

\newglossaryentry{container}
{
	name={container},
	description={Binnen datacompressie kan een container vrijwel letterlijk vertaald worden. Het is een verpakking voor alle data die men opslaat alsook de instructies voor het openen van die data. onder andere de codec plaatst data in deze container. Wanneer er gesproken wordt over bestandsextensies wordt vaak de container bedoeld. Bv; MP4 }
}

\newglossaryentry{jpeg}
{
	name={JPEG},
	description={Ook gekend als JPG. JPEG is een afkorting voor Joint Photographic Experts Group. JPEG is een bestandsformaat voor het opslaan van digitale afbeeldingen via lossy compressie. Afbeeldingcompressie en JPEG worden uitgebreid besproken in hoofdstuk \ref{ch:afbeeldingcompressie}}
}

\newglossaryentry{jpeg2000}
{
	name={JPEG2000},
	description={Ook gekend als JPEG2K. JPEG2000 is een bestandsformaat voor het opslaan van digitale afbeeldingen gemaakt als opvolger van JPEG. Net zoals JPEG maakt het gebruik van lossy compressie. Afbeeldingcompressie en JPEG2000 worden uitgebreid besproken in hoofdstuk \ref{ch:afbeeldingcompressie}}
}

\newglossaryentry{png}
{
	name={PNG},
	description={PNG is een afkorting voor Portable Network Graphics. PNG is een bestandsformaat voor het opslaan van digitale afbeeldingen. PNG maakt gebruik van lossless compressie. Afbeeldingcompressie en PNG worden uitgebreid besproken in hoofdstuk \ref{ch:afbeeldingcompressie}}
}

\newglossaryentry{h264-avc}
{
	name={H.264-AVC},
	description={H.264-AVC is één van de gekenste videocodecs die grootschalig gebruikt wordt. AVC is een afkorting van Advanced Video Coding. Videocompresssie en H.264-AVC worden uitgebreid besproken in hoofdstuk \ref{ch:videocompressie}}
}
% TODO: veruidelijken als deel geschreven is.

\newglossaryentry{h264-svc}
{
	name={H.264-SVC},
	description={H.264-SVC is een videocodec ontwikkelt als extensie van H264-AVC. SVC is een afkorting voor Scalable Video Coding. De nadruk bij deze extensie ligt zoals de naam sugereert op schaalbaarheid. Videocompresssie en H.264-SVC worden uitgebreid besproken in hoofdstuk \ref{ch:videocompressie}}
}
% TODO: veruidelijken als deel geschreven is.

\newglossaryentry{h265}
{
	name={H.265},
	description={H.265 is een videocodec ontwikkelt als opvolger van H.264. Videocompresssie en H.265 worden uitgebreid besproken in hoofdstuk \ref{ch:videocompressie}}
}
% TODO: veruidelijken als deel geschreven is.

\newglossaryentry{av1}
{
	name={AV1},
	description={AV1 is een videocodec ontwikkelt als open standaard. AV1 is een afkorting voor AOMedia Video. AV1 is vooral interessant omdat het royalty free is en dus geen licentiekosten heeft. Videocompresssie en AV1 worden uitgebreid besproken in hoofdstuk \ref{ch:videocompressie}}
}
% TODO: veruidelijken als deel geschreven is.

\newglossaryentry{open-source}
{
	name={open source},
	description={Als een programmeer project open source is wilt dit zeggen dat de broncode raadpleegbaar is en aanpasbaar is door iedereen. Dit wil echter niet gegarandeerd zeggen dat het software programma gratis is in gebruik.}
}

\newglossaryentry{lossless}
{
	name={lossless},
	description={Binnen datacompressie slaat lossless compressie op het comprimeren van een bestand zonder kwaliteitsverlies. In het geval van video's en afbeeldingen wilt dit zeggen dat een bestand gecomprimeerd met een lossless algoritme visueel dezelfde kwaliteit moet hebben}
}

\newglossaryentry{lossy}
{
	name={lossy},
	description={Binnen datacompressie slaat lossy compressie op het comprimeren van een bestand met mogelijks kwaliteitsverlies voor het besparen van data. In het geval van video's en afbeeldingen wilt dit zeggen dat een bestand gecomprimeerd met een lossy algoritme een significante hoeveelheid aan data en kwaliteit kan verliezen in vergelijking met het originele bestand}
}

% TODO: overige termen uit word
