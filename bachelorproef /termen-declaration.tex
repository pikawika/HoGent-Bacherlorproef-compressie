%run generate.sh en run hier
\newglossaryentry{datacompressie}
{
	name={datacompressie},
	description={De mens wilt vaak zoveel mogelijk output met zo weinig mogelijk input. Datacompressie vervult die wens binnen de computerwereld. Datacompressie draait er om zoveel mogelijk data weer te geven met zo weinig mogelijk bits}
}

\newglossaryentry{bit}
{
	name={bit},
	description={Zoals de naam bit, kort voor binary digit, suggereert kan een bit beschouwd worden als een binair signaal. Een bit wordt beschouwd als de kleinste dataeenheid voor dataopslag}
}

\newglossaryentry{dna-compressie}
{
	name={DNA compressie},
	description={Het menselijke DNA kan digitaal voorgesteld worden door een lange lijst van 5 verschillende karakters, gekend als basen. Deze digitale voorstelling bestaat uit meer dan 3 miljard van deze basen. DNA compressie bestaat er uit deze reeks van basen zo efficiënt mogelijk op te slaan zodanig dat performante bewerkingen mogelijk zijn met een zo klein mogelijke bestandsgrootte}
}

\newglossaryentry{afbeeldingscompressie}
{
	name={afbeeldingscompressie},
	description={TODO}
}

\newglossaryentry{videocompressie}
{
	name={videocompressie},
	description={TODO}
}

\newglossaryentry{codec}
{
	name={codec},
	description={TODO}
}

\newglossaryentry{jpeg}
{
	name={JPEG},
	description={TODO}
}

\newglossaryentry{jpeg2000}
{
	name={JPEG2000},
	description={TODO}
}

\newglossaryentry{png}
{
	name={PNG},
	description={TODO}
}

\newglossaryentry{h264}
{
	name={H264},
	description={TODO}
}

\newglossaryentry{h264-avc}
{
	name={H264-AVC},
	description={TODO}
}

\newglossaryentry{h264-svc}
{
	name={H264-SVC},
	description={TODO}
}

\newglossaryentry{h265}
{
	name={H265},
	description={TODO}
}

\newglossaryentry{av1}
{
	name={AV1},
	description={TODO}
}

\newglossaryentry{open-source}
{
	name={open source},
	description={TODO}
}

% TODO: overige termen uit word
