%===============================================================================
% LaTeX sjabloon voor de bachelorproef toegepaste informatica aan HOGENT
% Meer info op https://github.com/HoGentTIN/bachproef-latex-sjabloon
%===============================================================================
\documentclass{bachproef-tin}
\usepackage{hogent-thesis-titlepage} % Titelpagina conform aan HOGENT huisstijl

\usepackage[nonumberlist]{glossaries}
\renewcommand{\glossarysection}[2][]{} % Fix voor glossary met HoGent Theme
\makeglossaries
%run generate.sh en run hier
\newglossaryentry{datacompressie}
{
	name={datacompressie},
	description={Datacompressie bestaat uit het digitaal opslaan van een bestand met zo weinig mogelijk bits. Enkele belangrijke factoren voor het bepalen van de juiste datacompressie zijn gewenste kwaliteit, bestandsgrootte en snelheid}
}

\newglossaryentry{bit}
{
	name={bit},
	description={Zoals de naam bit, kort voor binary digit, suggereert kan een bit beschouwd worden als een binair signaal. Een bit wordt beschouwd als de kleinste eenheid voor dataopslag. Meer informatie rond bestandsgrootte en dataopslag worden besproken in hoofdstuk \ref{ch:literatuurstudie}, deel \ref{sec:bestandsgrootte-dataopslag}},
	plural={bits}
}

\newglossaryentry{dna-compressie}
{
	name={DNA compressie},
	description={Het menselijke DNA kan digitaal voorgesteld worden door een lange lijst van 5 verschillende karakters, gekend als basen. Deze digitale voorstelling bestaat uit meer dan 3 miljard van deze basen (\cite{dodanaugent2011}). DNA compressie bestaat er uit deze reeks van basen zo efficiënt mogelijk op te slaan zodanig dat performante bewerkingen mogelijk zijn met een zo klein mogelijke bestandsgrootte}
}

\newglossaryentry{afbeeldingscompressie}
{
	name={afbeeldingscompressie},
	description={Afbeeldingscompressie bestaat er uit een afbeelding in zo weinig mogelijk aantal bits op te slaan terwijl een aanvaardbare kwaliteit behouden blijft. Dit kan zowel via lossless als lossy algoritmes. Afbeeldingscompressie wordt uitgebreid besproken in hoofdstuk \ref{ch:afbeeldingscompressie}}
}

\newglossaryentry{videocompressie}
{
	name={videocompressie},
	description={Videocompressie bestaat er uit een videobestand in zo weinig mogelijk aantal bits op te slaan terwijl een accapteerbare kwaliteit behouden blijft. Dit kan zowel via lossless als lossy algoritmes. videcompressie wordt uitgebreid besproken in hoofdstuk \ref{ch:videocompressie}}
}

\newglossaryentry{codec}
{
	name={codec},
	description={De coder-decoder. Binnen datacompressie betekent codec de gebruikte techniek om een bestand te comprimeren. De codec is dus is de technologie verantwoordelijk voor het encoden en decoden van een bestand volgens een bapaald compressiealgoritme. BV; H.265},
	plural={codecs}
}

\newglossaryentry{afbeeldingsformaat}
{
	name={afbeeldingsformaat},
	description={Een afbeeldingsformaat bevat alle gegevens voor het digitaal opslaan van een afbeelding. De vier gekendste categoriën van afbeeldingsformaten zijn: raster, vector, compound en stereo formaten},
	plural={afbeeldingsformaten}
}

\newglossaryentry{container}
{
	name={container},
	description={Binnen datacompressie kan een container vrijwel letterlijk vertaald worden. Het is een verpakking voor alle data die men opslaat en metadata. onder andere de codec plaatst data in deze container. Wanneer er gesproken wordt over bestandsextensies wordt vaak de container bedoeld. Bv; MP4}
}

\newglossaryentry{jpeg}
{
	name={JPEG},
	description={Ook gekend als JPG. JPEG is een afkorting voor Joint Photographic Experts Group. JPEG is een bestandsformaat voor het opslaan van digitale afbeeldingen via lossy compressie. Afbeeldingscompressie en JPEG worden uitgebreid besproken in hoofdstuk \ref{ch:afbeeldingscompressie}}
}

\newglossaryentry{jpeg2000}
{
	name={JPEG2000},
	description={Ook gekend als JPEG2K. JPEG2000 is een bestandsformaat voor het opslaan van digitale afbeeldingen gemaakt als opvolger van JPEG. Net zoals JPEG maakt het gebruik van lossy compressie. Afbeeldingscompressie en JPEG2000 worden uitgebreid besproken in hoofdstuk \ref{ch:afbeeldingscompressie}}
}

\newglossaryentry{png}
{
	name={PNG},
	description={PNG is een afkorting voor Portable Network Graphics. PNG is een bestandsformaat voor het opslaan van digitale afbeeldingen. PNG maakt gebruik van lossless compressie. Afbeeldingscompressie en PNG worden uitgebreid besproken in hoofdstuk \ref{ch:afbeeldingscompressie}}
}

\newglossaryentry{h264-avc}
{
	name={H.264-AVC},
	description={H.264-AVC is één van de gekenste videocodecs die grootschalig gebruikt wordt. AVC is een afkorting van Advanced Video Coding. H.264-AVC wordt uitgebreid besproken in deel \ref{sec:videocompressie-h264-AVC}}
}

\newglossaryentry{h264-svc}
{
	name={H.264-SVC},
	description={H.264-SVC is een videocodec ontwikkelt als extensie op H264-AVC. SVC is een afkorting voor Scalable Video Coding. De nadruk bij deze extensie ligt zoals de naam sugereert op schaalbaarheid. H.264-SVC wordt uitgebreid besproken in deel \ref{sec:videocompressie-h264-SVC}}
}

\newglossaryentry{h265}
{
	name={H.265/HEVC},
	description={H.265/HEVC is een videocodec ontwikkelt als opvolger van H.264/AVC. H.265/HEVC wordt uitgebreid besproken in deel \ref{sec:videocompressie-h265}}
}

\newglossaryentry{av1}
{
	name={AV1},
	description={AV1 is een videocodec ontwikkelt als open standaard. AV1 is een afkorting voor AOMedia Video. AV1 is vooral interessant omdat het royalty free is en dus geen licentiekosten heeft. AV1  wordt uitgebreid besproken in deel \ref{sec:videocompressie-av1}}
}

\newglossaryentry{open-source}
{
	name={open source},
	description={Als een programmeer project open source is wilt dit zeggen dat de broncode raadpleegbaar is. Dit wil echter niet gegarandeerd zeggen dat het software programma gratis is in gebruik of de code zomaar aangepast mag worden. Dit hangt af van de licentie}
}

\newglossaryentry{lossless}
{
	name={lossless},
	description={Binnen datacompressie slaat lossless compressie op het comprimeren van een bestand zonder kwaliteitsverlies. In het geval van video's en afbeeldingen wilt dit zeggen dat een bestand gecomprimeerd met een lossless algoritme identiek is aan het origineel}
}

\newglossaryentry{lossy}
{
	name={lossy},
	description={Binnen datacompressie slaat lossy compressie op het comprimeren van een bestand met kwaliteitsverlies voor het besparen van data. In het geval van video's en afbeeldingen wilt dit zeggen dat een bestand gecomprimeerd met een lossy algoritme een significante hoeveelheid aan data en kwaliteit kan verliezen in vergelijking met het originele bestand}
}

\newglossaryentry{afbeeldingsevaluatietool}
{
	name={afbeeldingsevaluatietool},
	description={Een applicatie gemaakt voor het voeren van een objectief of subjectief onderzoek naar afbeeldingskwaliteit. De mogelijke soorten tools worden verder besproken in hoofdstuk \ref{ch:kwaliteit}. Een subjectieve afbeeldingsevaluatietool werd gebouwd voor deze paper en wordt verder besproken in hoofdstuk \ref{ch:onderzoek}}
}

\newglossaryentry{binaire-voorstelling-bestandsgrootte}
{
	name={Binaire voorstelling},
	description={De binaire voorstelling voor bestandsgroottes maakt gebruik van 1024 als basis wat overeen komt met $ 2^{10} $. Dit wordt verder besproken in deel \ref{sec:bestandsgrootte-dataopslag-voorvoegsels-binair}}
}

\newglossaryentry{si-voorstelling-bestandsgrootte}
{
	name={SI voorstelling},
	description={De SI voorstelling voor bestandsgroottes maakt gebruik van 1000 als basis wat overeen komt met $ 10^{3} $. Dit wordt verder besproken in deel \ref{sec:bestandsgrootte-dataopslag-voorvoegsels-si}}
}

\newglossaryentry{ieee}
{
	name={IEEE},
	description={Institute of Electrical and Electronics Engineers. Het is een non-profit instituut dat zich inzet voor technologische vooruitgang}
}

\newglossaryentry{clustergrootte}
{
	name={clustergrootte},
	description={De grootte van een cluster. Dit is verder besproken in deel \ref{sec:bestandsgrootte-dataopslag-clustergrootte}}
}

\newglossaryentry{cluster}
{
	name={cluster},
	description={Het kleinste deel van een oplsagmedium waar data in voorzien kan worden. Dit is verder besproken in deel \ref{sec:bestandsgrootte-dataopslag-clustergrootte}},
	plural={clusters}
}

\newglossaryentry{byte}
{
	name={byte},
	description={Een term voor acht bits},
	plural={bytes}
}

\newglossaryentry{leestijd}
{
	name={leestijd},
	description={De tijd die verstrekt tussen het aanvragen van een bestand op een opslagmedium tot het effectief verkrijgen van dat bestand},
	plural={leestijden}
}

\newglossaryentry{bandbreedte}
{
	name={bandbreedte},
	description={Een term waarmee de maximale beschikbare hoeveelheid data dat over een netwerkverbinding verstuurd kan worden bedoeld wordt}
}

\newglossaryentry{prefix-code}
{
	name={prefix code},
	description={Een korte waarde die een langere waarde voorstelt binnen prefix coding. Dit is verder beschreven in deel \ref{sec:ontstaan-datacompressie-primitieve-technieken-binnen-it}},
	plural={prefix codes}
}

\newglossaryentry{huffman-coding}
{
	name={Huffman coding},
	description={Een compressie-algoritme gebasseerd op prefix code. Huffman coding wordt verder toegelicht aan de hand van een voorbeeld in deel \ref{sec:primitieve-technieken-voorbeeld-huffman-encoding}. Een implementatie wordt voorzien in hoofdstuk \ref{ch:compressietool}}
}

\newglossaryentry{compressie-algoritme}
{
	name={compressie-algoritme},
	description={Een set instructies dat het mogelijk maakt om bepaalde informatie met minder resources voor te stellen. Binnen deze bachelorproef wilt dit code voorstellen die een een bestand zijn bestandsgrootte kan verkleinen. Dit kan zowel lossless als lossy},
	plural={compressie-algoritmen}
}

\newglossaryentry{lookup-table}
{
	name={lookup table},
	description={Een tabel waar waardes in opgezocht kunnen worden. Bij prefix code is dit de tabel waar de lange waarde staat die de korte prefix code voorstelt}
}

\newglossaryentry{prefix-coding}
{
	name={prefix coding},
	description={Een vorm van datacompressie waar bij een waarde wordt toegekend aan iets dat verwijst naar een andere, langere waarde. Dit is verder beschreven in deel \ref{sec:ontstaan-datacompressie-primitieve-technieken-binnen-it}}
}

\newglossaryentry{use-case}
{
	name={use case},
	description={Representeert een bepaald doel en de manier waarop dat doel bereken wenst te worden},
	plural={use cases}
}

\newglossaryentry{rle-long}
{
	name={run length encoding},
	description={Een compressietechniek die verder besproken wordt in deel \ref{sec:primitieve-technieken-voorbeeld-rle} en geïmplementeerd wordt in deel \ref{ch:compressietool}}
}

\newglossaryentry{rle-short}
{
	name={RLE},
	description={Kort voor run length encoding}
}

\newglossaryentry{ascii}
{
	name={ASCII},
	description={Een tekenset die voorgesteld wordt door acht bits}
}

\newglossaryentry{encoding}
{
	name={encoding},
	description={Het proces dat een encoder uitvoert}
}

\newglossaryentry{decoding}
{
	name={decoding},
	description={het process dat een decoder uitvoert}
}

\newglossaryentry{meta-data}
{
	name={metadata},
	description={Data over data}
}

\newglossaryentry{webp}
{
	name={WebP},
	description={Een afbeeldingsformaat dat uitgebreid besproken wordt in deel \ref{sec:afbeeldingscompressie-webp}}
}

\newglossaryentry{heif}
{
	name={HEIF},
	description={Een codec dat uitgebreid besproken wordt in deel \ref{sec:afbeeldingscompressie-heif}}
}

\newglossaryentry{ps}
{
	name={Adobe Photoshop},
	description={Computersoftware voor het aanmaken en bewerken van rasterafbeeldingen}
}

\newglossaryentry{css}
{
	name={CSS},
	description={Programmeertaal die verantwoordelijk is voor het opmaken van een webpagina}
}

\newglossaryentry{minifyen}
{
	name={minifyen},
	description={Een proces waarbij code zodanig herschreven wordt dat dit de minste hoveelheid ruimte in beslag neemt}
}

\newglossaryentry{render}
{
	name={render},
	description={het genereren van een digitale afbeelding of video met behulp van computersoftware}
}

\newglossaryentry{raw}
{
	name={RAW},
	description={Een benaming voor afbeeldingsformaten dat in geen enkele vorm gecomprimeerd worden. RAW wordt uitgebreid besproken in deel \ref{sec:afbeeldingscompressie-raw}}
}

\newglossaryentry{compressietool}
{
	name={datacompressietool},
	description={Een kleine applicatie voor het comprimeren van data. In hoofdstuk \ref{ch:compressietool} wordt een proof of concept compressietool geïmplementeerd}
}

\newglossaryentry{github}
{
	name={GitHub},
	description={Een online platform voor versiebeheer van code}
}

\newglossaryentry{hosting}
{
	name={hosting},
	description={Het delen van een bepaalde webpagina met anderen}
}

\newglossaryentry{php}
{
	name={PHP},
	description={Een programmeertaal die voornamelijk de logica bij websites voorziet}
}

\newglossaryentry{sql}
{
	name={SQL},
	description={Een standaard programmertaal voor communicatie met een relationele databank}
}

\newglossaryentry{xampp}
{
	name={XAMPP},
	description={Een software pakket waarmee eenvoudig een hosting omgeving kan opgezet worden}
}

\newglossaryentry{jquery}
{
	name={JQuery},
	description={Een uitbreiding op JavaScript}
}

\newglossaryentry{drift}
{
	name={Drift},
	description={Een JavaScript library dat het eenvoudig maakt om in te zoomen op afbeeldingen zonder manipulaties}
}

\newglossaryentry{bootstrap}
{
	name={Bootstrap},
	description={Een toolkit voor het eenvoudig programmeren met HTML, CSS, JS en jQuery}
}

\newglossaryentry{js}
{
	name={JavaScript},
	description={Een programmeertaal waarmee onder andere de inhoud van een webpagina gemanipuleerd kan worden}
}

\newglossaryentry{yt}
{
	name={YouTube},
	description={Een online website waar gebruikers gratis video's kunnen uploaden. YouTube is een dochterbedrijf van Google en is één van de grootste streaming diensten}
}

\newglossaryentry{plug-in}
{
	name={plug-in},
	description={Een downloadbare collectie code die bepaalde functionaliteit voorziet},
	plural={plug-ins}
}

\newglossaryentry{artefact}
{
	name={artefact},
	description={Een kunstmatig verschijnsel. Binnen datacompressie zijn artefacten zichtbare of hoorbare fouten},
	plural={artefacten}
}

\newglossaryentry{wavelet}
{
	name={wavelet},
	description={Een golfvormige soort data},
	plural={wavelets}
}

\newglossaryentry{ai}
{
	name={artificiële intelligentie},
	description={Wanneer een apparaat kan reageren op binnenkomende data en op basis daarvan een eigen beslissing kan maken zonder dit expliciet geprogrammeerd is spreek men van artificiële intelligentie}
}

\newglossaryentry{nef}
{
	name={NEF},
	description={Het RAW afbeeldingsformaat van Nikon}
}

\newglossaryentry{dng}
{
	name={DNG},
	description={Digital Negative Specification is een RAW formaat dat Adobe heeft uitgevonden met de bedoelding dat dit wereldwijd geadopteerd zou worden als standaard voor RAW afbeeldingsformaten}
}

\newglossaryentry{w3c}
{
	name={W3C},
	description={World Wide Web Consortium is een organisatie die standaarden voor het web voorziet}
}

\newglossaryentry{gif}
{
	name={GIF},
	description={Graphics interchange format is een afbeeldingsformaat dat op heden voornamelijk gebruikt wordt om bewegende delen voor te stellen}
}

\newglossaryentry{rgb}
{
	name={RGB},
	description={Red Green Blue is een kleurcodering systeem voor het voorstellen van kleuren aan de hand van deze drie basiskleuren}
}

\newglossaryentry{cmyk}
{
	name={CMYK},
	description={Cyan, Magenta, Yellow, Key is een kleurcodering systeem voor het voorstellen van kleuren aan de hand van deze drie basiskleuren}
}

\newglossaryentry{jpeg-exif}
{
	name={JPEG/Exif},
	description={Een bestandsformaat verder toegelicht in deel \ref{sec:afbeeldingscompressie-jpeg}}
}

\newglossaryentry{jpeg-jfif}
{
	name={JPEG/JFIF},
	description={Een bestandsformaat verder toegelicht in deel \ref{sec:afbeeldingscompressie-jpeg}}
}

\newglossaryentry{extensie}
{
	name={bestandsextensie},
	description={Een toevoeging op het einde van een bestandsnaam om weer te geven wat voor soort bestand het is},
	plural={bestandsextensies}
}

\newglossaryentry{dct}
{
	name={DCT},
	description={Discrete cosine transform is een wiskundige formule dat binnen datacompressie voornamelijk gebruikt wordt om een pixels te kunnen voorstellen als een makkelijk comprimeerbaar getal}
}

\newglossaryentry{wtcq}
{
	name={WTCQ},
	description={Wavelet/Trellis Coded Quantization is een compressie algoritme dat de start vormde voor JPEG2000. JPEG2000 wordt uigebreid besproken in deel \ref{sec:afbeeldingscompressie-jpeg2000}}
}

\newglossaryentry{iso}
{
	name={ISO},
	description={De International Organization for Standardization is een organisatie die internationale standaarden oplecht voor bijvoorbeeld compressie-algoritmen}
}

\newglossaryentry{jpf}
{
	name={JPF},
	description={JPf is de bestandsexentsie van JPEG2000. JPEG2000 wordt uigebreid besproken in deel \ref{sec:afbeeldingscompressie-jpeg2000}}
}

\newglossaryentry{intra-frame}
{
	name={intra-frame},
	description={intra-frame compressie is een techniek van afbeeldingen comprimeren binnen videocompressie dat uitgebreid besproken wordt in deel \ref{sec:videocompressie-intra-inter}},
	plural={intra-frames}
}

\newglossaryentry{inter-frame}
{
	name={inter-frame},
	description={inter-frame compressie is een techniek van afbeeldingen comprimeren binnen videocompressie dat uitgebreid besproken wordt in deel \ref{sec:videocompressie-intra-inter}},
	plural={inter-frames}
}

\newglossaryentry{decoder}
{
	name={decoder},
	description={Een softwareapplicatie dat een gecomprimeerd bestand terug omzet naar zijn originele vorm, al dan niet met kwaliteitsverlies },
	plural={decoders}
}

\newglossaryentry{encoder}
{
	name={encoder},
	description={Een softwareapplicatie dat een bestand omzet naar een gecomprimeerde vorm, al dan niet met kwaliteitsverlies},
	plural={encoders}
}

\newglossaryentry{heic}
{
	name={HEIC},
	description={Een afbeeldingsformaat dat uitgebreid besproken wordt in \ref{sec:afbeeldingscompressie-heif}},
}

\newglossaryentry{on-premise}
{
	name={on premise},
	description={Software is on-premise wanneer deze lokaal geïnstalleerd en gedraaid wordt op het toestel van de gebruiker},
}

\newglossaryentry{vector}
{
	name={vector},
	description={Een categorie voor afbeeldingsformaten dat verder toegelicht wordt in deel \ref{sec:afbeeldingscompressie-raster-vector}},
}

\newglossaryentry{cms}
{
	name={CMS},
	description={Een content management system is een systeem dat het voor een eindgebruiker eenvoudig moet maken om de inhoud binnen een IT-applicatie eenvoudig te beheren},
	plural={CMS'en}
}

\newglossaryentry{wordpress}
{
	name={WordPress},
	description={Een CMS voor web development}
}

\newglossaryentry{raster}
{
	name={raster},
	description={Een categorie voor afbeeldingsformaten dat verder toegelicht wordt in deel \ref{sec:afbeeldingscompressie-raster-vector}}
}

\newglossaryentry{lbhuffman}
{
	name={lbhuffman},
	description={Een bestandsextensie gebruikt voor binnen de datacompressietool verder besproken in deel \ref{sec:compressietool-opslaan}}
}

\newglossaryentry{lbrlea}
{
	name={lbrlea},
	description={Een bestandsextensie gebruikt voor binnen de datacompressietool verder besproken in deel \ref{sec:compressietool-opslaan}}
}

\newglossaryentry{lbrle}
{
	name={lbrle},
	description={Een bestandsextensie gebruikt voor binnen de datacompressietool verder besproken in deel \ref{sec:compressietool-opslaan}}
}

\newglossaryentry{html}
{
	name={HTML},
	description={Een programmeertaal voor het opbouwen van webpagina's}
}

\newglossaryentry{library}
{
	name={library},
	description={Een verzameling code dat door een programma gebruikt kan worden},
	plural={Libraries}
}

\newglossaryentry{string}
{
	name={string},
	description={Een term binnen programmeren dat slaat op een tekstfragment zonder opmaak},
	plural={strings}
}

\newglossaryentry{regex}
{
	name={regex},
	description={Regular Expressions zijn uitdrukkingen waarmee op zoek kan gegaan worden naar een tekst met een bepaalde structuur}
}

\newglossaryentry{recursieve-functie}
{
	name={recursieve functie},
	description={een recursieve functie is een functie die zichelf (al dan niet meermaals)oproept}
}

\newglossaryentry{array}
{
	name={array},
	description={Een term binnen programmeren dat slaat op een verzameling van variabelen}
}

\newglossaryentry{json}
{
	name={json},
	description={Een techniek voor het opslaan van variabelen in een string}
}

\newglossaryentry{rmse}
{
	name={rmse},
	description={Root mean square error, bespsroken in deel \ref{sec:kwaliteit-rmse}}
}

\newglossaryentry{ssim}
{
	name={SSIM},
	description={Structural similarity index, bespsroken in deel \ref{sec:kwaliteit-ssim}}
}

\newglossaryentry{vdp}
{
	name={VDP},
	description={Visual difference predictor is een term voor functies die visuele verschillen tussen afbeeldingen probeert uit te drukken}
}

\newglossaryentry{dxomark}
{
	name={DxOMark},
	description={Een erkende instelling dat de kwaliteit van, voornamelijk smartphone, camera's beoordeeld. DxOMark wordt verder toegelicht in deel \ref{sec:kwaliteit-dxomark}}
}

\newglossaryentry{maximum-compression}
{
	name={Maximum Compression},
	description={Een erkende instelling dat de compressieratio van lossles compressie-algoritmes beoordeeld. Maximum Compression wordt verder toegelicht in deel \ref{sec:kwaliteit-maximum-compression}}
}

\newglossaryentry{compressieratio}
{
	name={compressieratio},
	description={De verhouding waarmee een bestand gecomprimeerd is ten opzichte van het originele bestand}
}

\newglossaryentry{python}
{
	name={Python},
	description={Een programmeertaal dat voornamelijk gebruikt wordt voor het verwerken van data in de vorm van grote datascheets}
}

\newglossaryentry{pandas}
{
	name={Pandas},
	description={Een uitbreiding voor Python}
}

\newglossaryentry{illustrator}
{
	name={Adobe Illustrator},
	description={Computerssoftware voor het bewerken van vector afbeeldingen}
}

\newglossaryentry{svg}
{
	name={SVG},
	description={Scalable Vector Graphics zijn is een schaalbare vorm van vector afbeeldingen. Vector wordt verder toegelicht in deel \ref{sec:afbeeldingscompressie-raster-vector}}
}

\newglossaryentry{deflate}
{
	name={deflate},
	description={Een datacompressie techniek die voornamelijk binnen videcompressie gebruikt wordt}
}

\newglossaryentry{pixel-prediction}
{
	name={pixel prediction},
	description={Een datacompressie techniek die de pixel van een afbeelding gaat voorspellen aan de hand van een andere pixel}
}

\newglossaryentry{crc}
{
	name={CRC},
	description={Cyclic redundancy check is een foutdetectie code om te kunnen nagana of een bestand goed is toegekomen}
}

\newglossaryentry{bitplane}
{
	name={bitplane},
	description={Een term voor een groepering van bits},
	plural={bitplanes}
}

\newglossaryentry{mp4}
{
	name={MP4},
	description={Een container voor onder andere videocodecs. Videocodecs worden verder toegelicht in hoofdstuk \ref{ch:videocompressie}}
}

\newglossaryentry{avi}
{
	name={AVI},
	description={Een container voor onder andere videocodecs. Videocodecs worden verder toegelicht in hoofdstuk \ref{ch:videocompressie}}
}

\newglossaryentry{mkv}
{
	name={MKV},
	description={Een container voor onder andere videocodecs. Videocodecs worden verder toegelicht in hoofdstuk \ref{ch:videocompressie}}
}

\newglossaryentry{mpeg-4}
{
	name={MPEG-4},
	description={Een videocompressie algoritme dat de voorganger was van H.264/AVC toegelicht in deel \ref{sec:videocompressie-h264-AVC}}
}

\newglossaryentry{hlg}
{
	name={HLG},
	description={Hybrid Log-Gamma is een standaard voor het voorzien van HDR content}
}

\newglossaryentry{hdr}
{
	name={HDR},
	description={High Dynamic Range beelden zijn beelden die een hoog dynamisch bereik hebben}
}

\newglossaryentry{frame}
{
	name={frame},
	description={Een frame is één stilstaande afbeelding binnen een video},
	plural={frames}
}

\newglossaryentry{interlace}
{
	name={interlace},
	description={Een techniek waarbij twee frames tegelijk getoond worden waardoor de kijker een vals beeld krijgt dat de frame rate sneller is dan hij werkelijk is}
}

\newglossaryentry{frame-rate}
{
	name={frame rate},
	description={De snelheid waarmee frames elkaar opvolgen, vaak uitgedruk in frames per seconden (FPS)}
}

\newglossaryentry{pixel}
{
	name={pixel},
	description={Één punt binnen een rasterafbeelding},
	plural={pixels}
}

\newglossaryentry{vp8}
{
	name={VP8},
	description={Een video codec gemaakt door Google. WebP, besproken in deel \ref{sec:afbeeldingscompressie-webp}, is afkomstig van VP8. AV1 besproken in \ref{sec:videocompressie-av1} is een verre opvolger van VP8}
}

%%---------- Documenteigenschappen ---------------------------------------------
% De titel van het rapport/bachelorproef
\title{Je kijkt er naar, maar ziet het niet: datacompressie	principes - JPEG en PNG - H.264/AVC en H.264/SVC}

% Je eigen naam
\author{Bontinck Lennert}

% De naam van je promotor (lector van de opleiding)
\promotor{Wim De Bruyn}

% De naam van je co-promotor. Als je promotor ook je opdrachtgever is en je
% dus ook inhoudelijk begeleidt (en enkel dan!), mag je dit leeg laten.
\copromotor{Tom Paridaens}

% Indien je bachelorproef in opdracht van/in samenwerking met een bedrijf of
% externe organisatie geschreven is, geef je hier de naam. Zoniet laat je dit
% zoals het is.
\instelling{---}

% Academiejaar
\academiejaar{2018-2019}

% Examenperiode
%  - 1e semester = 1e examenperiode => 1
%  - 2e semester = 2e examenperiode => 2
%  - tweede zit  = 3e examenperiode => 3
\examenperiode{2}

%===============================================================================
% Inhoud document
%===============================================================================

\begin{document}

%---------- Taalselectie -------------------------------------------------------
% Als je je bachelorproef in het Engels schrijft, haal dan onderstaande regel
% uit commentaar. Let op: de tekst op de voorkaft blijft in het Nederlands, en
% dat is ook de bedoeling!

%\selectlanguage{english}

%---------- Titelblad ----------------------------------------------------------
\inserttitlepage

%---------- Samenvatting, voorwoord --------------------------------------------
\usechapterimagefalse
%==================
%% Voorwoord
%==================

\chapter*{Woord vooraf}
\label{ch:voorwoord}

De zoektocht naar een leuk, leerrijk en bruikbaar bachelorproefonderwerp is niet makkelijk. Wanneer ik echter op zoek was naar een geschikt \gls{afbeeldingsformaat} voor het bewaren van mijn steeds groeiende afbeeldingscollectie was ik stomverbaasd hoe weinig ik over dit onderwerp wist. Ook mijn vrienden, waarvan vele medestudenten Toegepaste Informatica te HoGent, konden mij geen antwoord geven op de vraag welk \gls{afbeeldingsformaat} een goede keuze zou zijn en kenden buiten het feit dat \gls{png} wel een transparante achtergrond kan hebben en \gls{jpeg} niet, geen echte verschillen tussen deze twee bekendste \glspl{afbeeldingsformaat}.

Ook op mijn stageplaats, waar websites en webshops gemaakt en onderhouden worden, waren maar weinig mensen zich echt bewust van de voordelen en nadelen van de verschillende \glspl{afbeeldingsformaat}. Meestal exporteerden ze afbeeldingen vanuit \gls{ps} met een kwaliteitsinstelling zodanig het eindbestand kleiner was dan 100kb om de performance van een website hoog te houden. De keuze voor een \gls{afbeeldingsformaat} anders dan \gls{jpeg} of \gls{png} wordt hier en in het algemeen te weinig overwegen ondanks dat hier een grote kwaliteitswinst gedaan kan worden met dezelfde of kleinere bestandsgrootte. 

Deze onwetendheid deed mij beseffen dat een onderzoek naar de verschillende soorten \gls{datacompressie} interessant kon worden. Zowel voor mede programmeurs die hun \gls{css} \gls{minifyen} om enkele kilobytes te besparen maar niet stilstaan hoeveel data ze kunnen besparen door het kiezen van een gepast \gls{afbeeldingsformaat} met de juiste \gls{render}instellingen en andere \gls{datacompressie}technieken. 

De achterliggende wiskunde en boeiende uitdagingen zoals \gls{dna-compressie} wisten mij ook te overtuigen dat dit onderwerp zeer verreikend zou zijn voor mij. Het vinden van een geweldige co-promotor, vakexpert Tom Paridaens, was dan ook de kers op de taart.

\pagebreak

Graag bedank ik dan ook mijn co-promotor, Tom Paridaens, voor de nauwe samenwerking binnen deze bachelorproef. Ook bedank ik graag Wim De Bruyn voor de leerrijke lessen onderzoekstechnieken die ons de nodige kennis brachten voor het voeren van een gegrond onderzoek. Bovendien is Wim De Bruyn de promotor van deze bachelorproef waarvoor ik hem ook wil danken.

Ook wil ik Mayté Bogaert, van MaytéB fotografie, bedanken voor het aanleveren van meerdere \gls{raw} bestanden die ik heb gebruikt voor het uitvoeren van mijn onderzoek naar een geschikt \gls{afbeeldingsformaat} voor een bepaalde \gls{use-case} besproken in deze bachelorproef.

Bert Van Vreckem en collega's verdienen ook een dankwoord voor het opstellen van een \LaTeX{} sjabloon voor deze bachelorproef. Tot slot bedank ik graag de deelnemers die aan de hand van mijn \gls{afbeeldingsevaluatietool} hebben bijgedragen naar het onderzoek van een geschikt \gls{afbeeldingsformaat} voor de \gls{use-case} besproken in deze bachelorproef.
%==================
%% Samenvatting
%==================

% De "abstract" of samenvatting is een kernachtige (~ 1 blz. voor een - op het einde
% thesis) synthese van het document.
%
% Deze aspecten moeten zeker aan bod komen:
% - Context: waarom is dit werk belangrijk?
% - Nood: waarom moest dit onderzocht worden?
% - Taak: wat heb je precies gedaan?
% - Object: wat staat in dit document geschreven?
% - Resultaat: wat was het resultaat?
% - Conclusie: wat is/zijn de belangrijkste conclusie(s)?
% - Perspectief: blijven er nog vragen open die in de toekomst nog kunnen
%    onderzocht worden? Wat is een mogelijk vervolg voor jouw onderzoek?
%
% LET OP! Een samenvatting is GEEN voorwoord!


\chapter*{Samenvatting}
\label{ch:samenvatting}

\Gls{datacompressie}: een fundamenteel onderdeel van de IT-wereld waar weinig belanghebbenden een basiskennis van hebben. Waarom is dat? Schrikken de complexe en zeer uitgebreide papers reeds geschreven over dit onderwerp geïnteresseerden af? Is het een te complex onderwerp om te voorzien in meer IT-gerelateerde opleidingen? Staat \gls{datacompressie} stil in de tijd dat standaarden als het \glspl{afbeeldingsformaat} \gls{jpeg} al meer dan twintig jaar het bekendste \gls{afbeeldingsformaat} is? Deze bachelorproef tracht een antwoord te geven op die vragen en de tal van andere onderzoeksvragen besproken in deel \ref{sec:onderzoeksvragen}. 

Dit document is gemaakt met een eenvoudig visie: de nodige basiskennis over het ontstaan van \gls{datacompressie}, de werking van enkele \glspl{compressie-algoritme}, tal van \glspl{afbeeldingsformaat} en video \glspl{codec} en de manieren voor het evalueren van compressiekwaliteit toe te lichten. Na het lezen van deze bachelorproef zal de lezer meer stilstaan bij de keuze voor een geschikt \gls{compressie-algoritme}.

Er is zowel een proof of concept \gls{compressietool} als een uitgebreide \gls{afbeeldingsevaluatietool} geschreven voor de bachelorproef die gratis zijn in gebruik en \gls{open-source} toegankelijk zijn op de \gls{github} repository van deze bachelorproef\urlcite{githubbachelorproef}. Er wordt dieper ingegaan op het ontstaan, de werking en de voordelen en nadelen van volgende \glspl{afbeeldingsformaat}: \gls{png} | \gls{jpeg} | \gls{jpeg2000} | \gls{webp} | \gls{heif}. Hetzelfde wordt gedaan voor de volgende video \glspl{codec}: \gls{h264-avc} | \gls{h264-svc} | \gls{h265} | \gls{av1}.

Er wordt een subjectief onderzoek gevoerd aan de hand van de eerder benoemde \gls{afbeeldingsevaluatietool}. Dit geeft samen met de theoretische kennis die zal verkregen worden door de bachelorproef een antwoord op de hoofdonderzoeksvraag van deze bachelorproef: Waarom moet er stilgestaan worden bij het gebruiken van \glspl{compressie-algoritme}, hoe kies je een geschikt \gls{compressie-algoritme} voor een bepaalde \gls{use-case} en hoe implementeer je dit het best?

Het volledige verloop van deze bachelorproef is beschreven in deel \ref{sec:opzet-bachelorproef}.

%---------- Inhoudstafel -------------------------------------------------------
\pagestyle{empty} % Geen hoofding
\tableofcontents  % Voeg de inhoudstafel toe
\cleardoublepage  % Zorg dat volgende hoofstuk op een oneven pagina begint
\pagestyle{fancy} % Zet hoofding opnieuw aan

%---------- Kern ---------------------------------------------------------------

% De eerste hoofdstukken van een bachelorproef zijn meestal een inleiding op
% het onderwerp, literatuurstudie en verantwoording methodologie.
% Aarzel niet om een meer beschrijvende titel aan deze hoofstukken te geven of
% om bijvoorbeeld de inleiding en/of stand van zaken over meerdere hoofdstukken
% te verspreiden!

%%=============================================================================
%% Inleiding
%%=============================================================================

\chapter{Inleiding}
\label{ch:inleiding}

\Gls{datacompressie} en het achterliggende compressie idee is niets nieuw. Integendeel, het is één van de oudste zaken binnen IT dat tot op heden van fundamenteel belang is voor zowat alle IT-toepassingen. Door de databesparing werkt alles niet alleen veelvouden sneller en goedkoper, maar worden bepaalde zaken die voorheen onmogelijk leken mogelijk. Denk hierbij bijvoorbeeld aan recente doorbraken binnen \gls{dna-compressie} dat het mogelijk maken steeds meer onderzoeken met betrekking tot het menselijk genoom uit te voeren.



\section{Probleemstelling}
\label{sec:probleemstelling}


%TODO: verder 
%De inleiding moet de lezer net genoeg informatie verschaffen om het onderwerp te begrijpen en in te zien waarom de onderzoeksvraag de moeite waard is om te onderzoeken. In de inleiding ga je literatuurverwijzingen beperken, zodat de tekst vlot leesbaar blijft. Je kan de inleiding verder onderverdelen in secties als dit de tekst verduidelijkt. Zaken die aan bod kunnen komen in de inleiding~\autocite{Pollefliet2011}:
%\begin{itemize}
%\item context, achtergrond
%\item afbakenen van het onderwerp
%\item verantwoording van het onderwerp, methodologie
%\item probleemstelling
%\item onderzoeksdoelstelling
%\item onderzoeksvraag
%\item \ldots
%\end{itemize}

\section{Onderzoeksvragen}
\label{sec:onderzoeksvragen}


%TODO: verder 
%Wees zo concreet mogelijk bij het formuleren van je onderzoeksvraag. Een onderzoeksvraag is trouwens iets waar nog niemand op dit moment een antwoord heeft (voor zover je kan nagaan). Het opzoeken van bestaande informatie (bv. ``welke tools bestaan er voor deze toepassing?'') is dus geen onderzoeksvraag. Je kan de onderzoeksvraag verder specifiëren in deelvragen. Bv.~als je onderzoek gaat over performantiemetingen, dan 

\section{Onderzoeksdoelstelling}
\label{sec:onderzoeksdoelstelling}

%TODO: verder 
%Wat is het beoogde resultaat van je bachelorproef? Wat zijn de criteria voor succes? Beschrijf die zo concreet mogelijk. Gaat het bv. om een proof-of-concept, een prototype, een verslag met aanbevelingen, een vergelijkende studie, enz.

\section{Opzet van deze bachelorproef}
\label{sec:opzet-bachelorproef}

% Het is gebruikelijk aan het einde van de inleiding een overzicht te
% geven van de opbouw van de rest van de tekst. Deze sectie bevat al een aanzet
% die je kan aanvullen/aanpassen in functie van je eigen tekst.

\subsection{Deel 1: situering en literatuurstudie}
\label{sec:opzet-bachelorproef-deel-1}

Deze paper zal zich in het eerste deel focussen op het toelichten van de belangrijke termen binnen \gls{datacompressie}. In hoofdstuk~\ref{ch:termen} is een lijst met belangrijker termen te vinden die binnen \gls{datacompressie} en deze paper vaak voorkomen. Doorheen deze paper zullen tal van referenties naar deze termen gelegd worden. 

Hoofdstuk~\ref{ch:methodologie} licht de gebruikte methodologie voor deze paper toe. Hieruit wordt duidelijk dat deze paper zo objectief mogelijk is opgesteld met een focus op duidelijkheid en reproduceerbaarheid.

Hoofdstuk~\ref{ch:literatuurstudie} behoort ook tot het situerende eerste deel en zal het ontstaan van \gls{datacompressie} en enkele basisprincipes toelichten. Een reeks van deze primitieve technieken zullen aan de hand van een voorbeeld toegelicht worden.
%todo: link naar voorbeeld en welke manieren

\subsection{Deel 2: compressie tool ontwikkelen}
\label{sec:opzet-bachelorproef-deel-2}
 
 In het tweede zal een basis \gls{datacompressie} tool programmatisch geïmplementeerd worden om de theorie uit het eerste deel in praktijk te zien.
 
Hoofdstuk~\ref{ch:compressietool} is hierdoor gericht voor technische lezers als programmeurs. Er is echter telkens voldoende randinformatie gegeven zodanig ook de minder technische lezers een blik achter de schermen kunnen verkrijgen.
 %TODO: welke taal tool gemaakt is en wat deze juist doet etc

\subsection{Deel 3: afbeelding- en videocompressie}
\label{sec:opzet-bachelorproef-deel-3}

In het derde deel worden twee sub domeinen van \gls{datacompressie} verder toegelicht; \gls{afbeeldingscompressie} en \gls{videocompressie}. 

In hoofdstuk~\ref{ch:afbeeldingcompressie} zal er dieper ingegaan worden op volgende afbeelding \glspl{codec}: \gls{jpeg}, \gls{jpeg2000} en \gls{png}. 
%TODO: aanpassen indien andere op input copromoter

In hoofdstuk~\ref{ch:videocompressie} zal er verder ingegaan worden op de gekende \gls{videocompressie} standaard: \gls{h264-avc} en \gls{h264-svc}. Ook de opvolger \gls{h265} en open source \gls{av1} zullen besproken worden.
%TODO: aanpassen indien andere op input copromoter

\subsection{Deel 4: onderzoek afbeelding compressie}
\label{sec:opzet-bachelorproef-deel-4}

In het vierde deel wordt besproken hoe compressiemethoden binnen video en afbeelding geëvalueerd worden. Hoofdstuk~\ref{ch:kwaliteit} zal enkele veel gebruikte tools en methoden voor objectieve en subjectieve metingen toelichten.

In hoofdstuk~\ref{ch:onderzoek} wordt een subjectieve test voor het evalueren van afbeeldingskwaliteit opgesteld voor portretfoto's. Hierbij zullen enkele van de besproken  \glspl{codec} uit hoofdstuk~\ref{ch:afbeeldingcompressie} tegen elkaar concurreren. De gebruikte tool is voor deze paper opgesteld en zal \gls{open-source} toegankelijk zijn wat het eenvoudig mogelijk maakt om een gelijkaardig onderzoek uit te voeren.
%TODO: in staat stelllen als bv prog of content zelf kiezen welk gebruiken

\subsection{Deel 5: uitdagingen en conclusie}
\label{sec:opzet-bachelorproef-deel-5}

In het vijfde deel zullen de huidige uitdagingen van \gls{datacompressie} kort toegelicht worden. Zo zal hoofdstuk~\ref{ch:uitdagingen} een beeld geven van de taken die mensen als Tom Paridaens, co-promoter voor deze paper, krijgen.
%TODO: copromoter zijn job en eerbetoon vermelden

In hoofdstuk~\ref{ch:conclusie} wordt kort teruggeblikt op de paper en worden enkele besluiten uit het onderzoek van hoofdstuk~\ref{ch:onderzoek} opgesomd. Daarbij wordt ook een aanzet gegeven om zelf meer na te denken over het gebruik van \gls{datacompressie} en bepaalde  \glspl{codec} in projecten, of nog beter, zelf een onderzoek uit te voeren!
%TODO: lezer aanzetten meer etc
%deel 1
\chapter{Belangrijke termen in datacompressie}
\label{ch:termen}

Om deze paper vlot te kunnen lezen, zijn er enkele termen die de lezer moet kennen. Termen die vaak voorkomen in datacompressie en doorheen deze paper worden hier opgesomd. Er zal een referentie voorzien zijn naar deze lijst bij het gebruik van een term uit de lijst.

\glsaddall
\printglossary[title=Woordenlijst]

%%=============================================================================
%% Methodologie
%%=============================================================================

\chapter{Methodologie}
\label{ch:methodologie}

%% TODO: Hoe ben je te werk gegaan? Verdeel je onderzoek in grote fasen, en
%% licht in elke fase toe welke stappen je gevolgd hebt. Verantwoord waarom je
%% op deze manier te werk gegaan bent. Je moet kunnen aantonen dat je de best
%% mogelijke manier toegepast hebt om een antwoord te vinden op de
%% onderzoeksvraag.

%TODO: zeggen dat je literatuurstudie van betrouwbare bronnen komt 
%TODO: zeggen dat je tool open source is en zelfgeschreven met validatie van co-promoter 
%TODO: zeggen dat je opzoekingswerk afb en vid compressie goed is door co-promoter en zijn collega's
%TODO: zeggen dat eval tool open source zelfgemaakt is met zelfde pc etc en dat daarom betrouwbare resultaten zijn die kunnen nagebootst worden
%TODO: zeggen dat door werkwijze toelichten zelf een test opgezet kan worden zodanig dit besluit gevalideerd kan worden of zelf verder onderzoek
\chapter{Literatuurstudie}
\label{ch:literatuurstudie}

% Tip: Begin elk hoofdstuk met een paragraaf inleiding die beschrijft hoe
% dit hoofdstuk past binnen het geheel van de bachelorproef. Geef in het
% bijzonder aan wat de link is met het vorige en volgende hoofdstuk.

Deze literatuurstudie zal samen met de lijst van termen uit hoofdstuk \ref{ch:termen} de nodige achtergrondinformatie bieden om de volgende delen van de paper te begrijpen. Er zal verwezen worden naar meerdere papers van andere instellingen zodanig dat u zich verder kan inlezen waar gewenst.

% Pas na deze inleidende paragraaf komt de eerste sectiehoofding.


%Dit hoofdstuk bevat je literatuurstudie. De inhoud gaat verder op de inleiding, maar zal het onderwerp van de bachelorproef *diepgaand* uitspitten. De bedoeling is dat de lezer na lezing van dit hoofdstuk helemaal op de hoogte is van de huidige stand van zaken (state-of-the-art) in het onderzoeksdomein. Iemand die niet vertrouwd is met het onderwerp, weet nu voldoende om de rest van het verhaal te kunnen volgen, zonder dat die er nog andere informatie moet over opzoeken \autocite{Pollefliet2011}.

%Je verwijst bij elke bewering die je doet, vakterm die je introduceert, enz. naar je bronnen. In \LaTeX{} kan dat met het commando \texttt{$\backslash${textcite\{\}}} of \texttt{$\backslash${autocite\{\}}}. Als argument van het commando geef je de ``sleutel'' van een ``record'' in een bibliografische databank in het Bib\LaTeX{}-formaat (een tekstbestand). Als je expliciet naar de auteur verwijst in de zin, gebruik je \texttt{$\backslash${}textcite\{\}}.
%Soms wil je de auteur niet expliciet vernoemen, dan gebruik je \texttt{$\backslash${}autocite\{\}}. In de volgende paragraaf een voorbeeld van elk.


%\textcite{Knuth1998} schreef een van de standaardwerken over sorteer- en zoekcompressie-algoritmen. Experten zijn het erover eens dat cloud computing een interessante opportuniteit vormen, zowel voor gebruikers als voor dienstverleners op vlak van informatietechnologie~\autocite{Creeger2009}.

\section{Bestandsgrootte en dataopslag}
\label{sec:bestandsgrootte-dataopslag}
Zoals in de definitie van \gls{datacompressie} besproken is, bestaat \gls{datacompressie} uit het digitaal opslaan van een bestand met zo weinig mogelijk \glspl{bit}.

\Gls{bit} staat kort voor binary digit. Een bit wordt beschouwd als de kleinste data eenheid voor dataopslag. Een bit kan twee waarden aannemen, deze worden voorgesteld door 1 of 0 (binair talstelsel) maar kunnen ook geïnterpreteerd worden als aan of uit, ja of nee…

\subsection{Voorvoegsels voor het uitdrukken van bestandsgrootte}
\label{sec:bestandsgrootte-dataopslag-voorvoegsels}

Bestandsgroottes worden meestal uitgedrukt in bytes (8 bits), al dan niet met voorvoegsel dat een veelvoud voorstelt. Deze voorvoegsels en het door elkaar gebruik van de \gls{si-voorstelling-bestandsgrootte} en \gls{binaire-voorstelling-bestandsgrootte} kan voor enige verwarring zorgen. Denk hierbij aan het fenomeen dat harde schijven die geadverteerd zijn als 1TB (\gls{si-voorstelling-bestandsgrootte}) overeen komt met 931GiB (\gls{binaire-voorstelling-bestandsgrootte}) op de meeste besturingssystemen. Doorheen deze paper zal de \gls{si-voorstelling-bestandsgrootte} gebruikt worden.

\subsubsection{SI voorstelling}
\label{sec:bestandsgrootte-dataopslag-voorvoegsels-si}

De \gls{si-voorstelling-bestandsgrootte} gebruikt als basis 1000 wat overeen komt met $ 10^{3} $. SI staat voor International System of Units en beschrijft. Het wordt beschouwd als een moderne vorm op het metrisch stelsel. SI beschrijft onder IEC 60027 het gebruik van bepaalde voorvoegsels voor het uitdrukken van machten op 10. (\cite{iec60027}) Een conversietabel is hieronder raadpleegbaar. 

\FloatBarrier
\begin{table}[h]
	\begin{tabular}{lll}
		Voorvoegsel & symbool & waarde \\
		kilo & Ki & $ 10^{3} = 1000^{1} $  = 1 000 \\
		mega & Mi & $ 10^{6} = 1000^{2} $  = 1 000 000 \\
		giga & Gi & $ 10^{9} = 1000^{3} $  = 1 000 000 000
	\end{tabular}
\end{table}
\FloatBarrier

\subsubsection{binaire voorstelling}
\label{sec:bestandsgrootte-dataopslag-voorvoegsels-binair}

De \gls{binaire-voorstelling-bestandsgrootte} gebruikt als basis 1024 wat overeen komt met $ 2^{10} $. Deze voorstelling is een standaardisatie opgelegd door \gls{ieee} 1541-2002 (\cite{ieee15412002}). Een conversietabel is hieronder raadpleegbaar.

\FloatBarrier
\begin{table}[h]
	\begin{tabular}{lll}
		Voorvoegsel & symbool & waarde \\
		kibi & Ki & $ 2^{10} = 1024^{1} $  = 1 024 \\
		mebi & Mi & $ 2^{20} = 1024^{2} $ = 1048 576 \\
		gibi & Gi & $ 2^{30} = 1024^{3} $ = 1 073 741 824
	\end{tabular}
\end{table}
\FloatBarrier

\subsubsection{clustergrootte}
\label{sec:bestandsgrootte-dataopslag-clustergrootte}

Een andere belangrijke term bij dataopslag is de \gls{clustergrootte}. Data moet namelijk bijgehouden worden in één of meerdere \glspl{cluster} op een opslagmedium zodat naar deze \glspl{cluster} kan verwezen worden voor het lezen van de data. 

Aangezien alle data steeds minstens in één \gls{cluster} staat en twee verschillende databestanden nooit een \gls{cluster} kunnen delen kan dit voor opslagruimteverlies zorgen. 

Neem bijvoorbeeld een \gls{clustergrootte} van 4096 \glspl{byte}, een vaak voorkomende \gls{clustergrootte}. Als in deze situatie een bestand van 2000 \glspl{byte} groot zou opgeslagen worden, zijn de overige 2096 \glspl{byte} aan opslagcapaciteit op die schijf verloren. Een bestand van 4097 \glspl{byte} zou twee \glspl{cluster} in beslag nemen waardoor 4095 \glspl{byte} verloren gaan. 

Dit speelt vooral een rol wanneer \gls{datacompressie} gebruikt wordt voor het besparen van opslagruimte op een opslagmedium. Een gecomprimeerd bestand met een kleinere bestandsgrootte dat dezelfde hoeveelheid \glspl{clustergrootte} nodig heeft op het medium zal dus niet voor plaats besparing zorgen op dat opslagmedium. 

Theoretisch gezien zal er wel een verbetering te zien zijn in \glspl{leestijd} en de gebruikte \gls{bandbreedte} bij een bestandsoverdracht omdat de effectieve bestandsgrootte kleiner is. 

Er zijn tal van reden waarom een andere \gls{clustergrootte} aangeraden is, een recente discussie is terug te vinden in een blogpost van Microsoft\urlcite{microsoftblogcluster}

\section{Ontstaan datacompressie en primitieve technieken}
\label{sec:ontstaan-datacompressie-primitieve-technieken}
\subsection{Eerste vorm van datacompressie}
\label{sec:ontstaan-datacompressie-primitieve-technieken-eerste-vorm}
Vele onderzoekers zijn het erover eens dat \gls{datacompressie} dateert van voor de uitvinding van de computer. Vele onderzoekers zijn het er over eens dat morsecode de eerste vorm was van \gls{datacompressie}. Morsecode is uitgevonden in 1832 door Samuel F.B. Morse en wordt aanzien als eerste vorm van datacompressie doordat veel voorkomende letters een kortere audiotoon kregen dan minder gebruikte letters. (\cite{morsecode})

\subsection{Ontwikkeling datacompressie binnen IT}
\label{sec:ontstaan-datacompressie-primitieve-technieken-binnen-it}
Bij de prille opkomst van mainframe eind de jaren 40 en begin de jaren 50 zijn twee belangrijke doorbraken binnen \gls{datacompressie} gemaakt. Beiden maken gebruik van \gls{prefix-code}. De originele uitvinder van dit soort compressie was Shannon Claude dat Shannon coding uitvond, een proof of concept voor zijn artikel \citetitle{shannon1948} (\cite{shannon1948}). In diezelfde periode werd ook Shannon-Fano coding voorgesteld, een project samen met Robert Fano dat verschillende \glspl{use-case} had. Geen van beide technieken waren echter optimaal aangezien de \glspl{compressie-algoritme} niet gegarandeerd de korst mogelijke prefix codes gaven. 

\Gls{huffman-coding}, voorgesteld in  \citetitle{huffman} (\cite{huffman}) was een optimale variant op deze techniek. Dit is een \gls{compressie-algoritme} door David Huffman gemaakt als academie opdracht dat bijna 70 jaar na publicatie nog steeds de basis legt voor vele \gls{lossless} \gls{datacompressie} \glspl{compressie-algoritme}. Deze soort \glspl{compressie-algoritme} worden \gls{frequency-based} \glspl{compressie-algoritme} genoemd. Het exacte verschil tussen \gls{huffman-coding} en Shannon-Fano coding en meer informatie over deze \glspl{compressie-algoritme} zijn beschreven in \citetitle{lelewer87datacompression} (\cite{lelewer87datacompression}). In deel \ref{sec:primitieve-technieken-voorbeeld-huffman-encoding} wordt een praktisch voorbeeld van \gls{huffman-coding} uitgewerkt. In hoofdstuk  \ref{ch:compressietool} zal onder andere \gls{huffman-coding} gebruikt worden voor het maken van de compressietool.

De jaren 70 en 80 zorgden voor tal van belangrijke doorbraken binnen \gls{datacompressie}. Dit was te weiden aan de opkomst van het internet en de steeds groter wordende bestanden. Ook werd hardware matige compressie (zoals \gls{prefix-code} met vaste \gls{lookup-table} voor tekstbestanden) steeds meer vervangen door dynamische compressie (codegewijs). 

Deze eerste softwareoplossingen waren veelal implementatie van \gls{huffman-coding}, eventueel met kleine aanpassingen. Eind de jaren 70 waren de eerste Lempel-Ziv \gls{compressie-algoritme} uitgevonden: LZ77 en LZ78. Dit zijn de grondleggers van \gls{dictionary-coding}. Een veelgebruikte variant van LZ798 is LZW (1984). Net zoals \gls{huffman-coding} de basis legde voor vele van de eerste softwareoplossingen zorgde de grondleggers van \gls{dictionary-coding} voor vele nieuwe softwareoplossingen. De doorgroei van deze \glspl{compressie-algoritme} is zichtbaar in figuur \ref{fig:lossles-datacompressie-overzicht}.

\begin{figure}
	\includegraphics{img/literatuurstudie/lossles_datacompressie_overzicht.png}
	\caption{Lossless datacompressie overzicht (\cite{ethwcompressionhistory})}
	\label{fig:lossles-datacompressie-overzicht}
\end{figure}

Het grootste verschil tussen \gls{prefix-coding} en \gls{dictionary-coding} zit in de naam zelf. Bij \gls{prefix-coding} wordt elk karakter vervangen door een \gls{prefix-code} terwijl bij \gls{dictionary-coding} een reeks van karakters vervangen kunnen worden door één enkele \gls{prefix-code}.  

Eind de jaren 80 en begin de jaren 90, door de digitalisering van afbeeldingen en muziek, begonnen \gls{lossy} \glspl{compressie-algoritme} steeds meer op te komen. Het verschil tussen \gls{lossless} en \gls{lossy} \glspl{compressie-algoritme} wordt in deel \ref{sec:ontstaan-datacompressie-lossless-lossy} verder besproken.


\subsection{Primitieve technieken: een voorbeeld}
\label{sec:primitieve-technieken-voorbeeld}
TODO

\subsubsection{Situering}
\label{sec:primitieve-technieken-voorbeeld-situering}
TODO

\subsubsection{ASCII encoding en decoding}
\label{sec:primitieve-technieken-voorbeeld-ascii}
TODO

\paragraph{ASCII Probleemstelling 1: 8 bits per karakter}
\label{par:primitieve-technieken-voorbeeld-ascii-probleem-1}
TODO

\subsubsection{RLE: run length encoding en decoding}
\label{sec:primitieve-technieken-voorbeeld-rle}
TODO

\paragraph{RLE Probleemstelling 1: gecomprimeerd bestand groter dan bron}
\label{par:primitieve-technieken-voorbeeld-rle-probleem-1}
TODO

\subsubsection{Huffman coding}
\label{sec:primitieve-technieken-voorbeeld-huffman-encoding}
TODO

\paragraph{Huffman encoding stap 1: meerdere bomen}
\label{par:primitieve-technieken-voorbeeld-huffman-encoding-1}
TODO

\paragraph{Huffman encoding stap 2: bomen samenvoegen}
\label{par:primitieve-technieken-voorbeeld-huffman-encoding-2}
TODO

\paragraph{stap 3: prefix tabel (optioneel)}
\label{par:primitieve-technieken-voorbeeld-huffman-encoding-3}
TODO

\paragraph{stap 4: encoding}
\label{par:primitieve-technieken-voorbeeld-huffman-encoding-4}
TODO

\paragraph{Huffman decoding}
\label{par:primitieve-technieken-voorbeeld-huffman-decoding}
TODO

\paragraph{Huffman coding Probleemstelling 1: binaire boom niet opgeslagen}
\label{par:primitieve-technieken-voorbeeld-huffman-probleem-1}
TODO

\paragraph{Huffman coding Probleemstelling 2: gecomprimeerd bestand groter dan bron}
\label{par:primitieve-technieken-voorbeeld-huffman-probleem-2}
TODO

\paragraph{Huffman coding Probleemstelling 3: overlappende prefix codes}
\label{par:primitieve-technieken-voorbeeld-huffman-probleem-3}
TODO

\section{Lossless vs lossy datacompressie}
\label{sec:ontstaan-datacompressie-lossless-lossy}

%deel 2
\chapter{Proof of concept compressietool}
\label{ch:compressietool}

In deel \ref{sec:primitieve-technieken-voorbeeld-rle} en \ref{sec:primitieve-technieken-voorbeeld-huffman-encoding} wordt uitgelegd hoe \gls{rle-long} en \gls{huffman-coding} werken. Dit hoofdstuk focust zich op de implementatie van deze \gls{compressie-algoritme}. Er zal een proof of concept \gls{compressietool} gebouwd worden in \gls{php} en de werking zal toegelicht worden. Deze is in staat om input bestanden onder de vorm van simpele tekst in een txt bestand om te zetten naar hen gecomprimeerde vorm.
 
\section{Gebruikte technologie}
\label{sec:compressietool-gebruikte-technologie}

Deze \gls{compressietool} is geschreven in \gls{php}. Dit maakt het mogelijk te tool eenvoudig lokaal te runnen door het gebruik van een webserver omgeving als \gls{xampp} of hem online te zetten op een \gls{hosting} platform. Deze tool is online raadpleegbaar op de website van Lennert Bontinck\urlcite{compressietool}.

\section{Run length encoding}
\label{sec:compressietool-rle}

TODO
%TODO: compressietool

\section{Huffman Coding}
\label{sec:compressietool-huffman}

TODO
%TODO: compressietool

\section{Patronen}
\label{sec:compressietool-patronen}

TODO
%TODO: compressietool

\section{Resultaten}
\label{sec:compressietool-resultaten}

TODO
%TODO: compressietool
%deel 3
\chapter{Afbeeldingcompressie}
\label{ch:afbeeldingcompressie}

%TODO: afbeeldingcompressie
\chapter{Videocompressie}
\label{ch:videocompressie}

TODO

%TODO: videocompressie
%deel 4
\chapter{Kwaliteit beoordelen}
\label{ch:kwaliteit}

TODO

%TODO: kwaliteit
\chapter{Onderzoek}
\label{ch:onderzoek}

TODO

%TODO: onderzoek

\section{Waarom een subjectief onderzoek}
\label{sec:onderzoek-waarom-subjectief}

TODO
%TODO: onderzoek

\section{Use case}
\label{sec:onderzoek-use-case}

TODO
%TODO: onderzoek

TODO
%TODO: onderzoek

\section{Uitvoering}
\label{sec:onderzoek-uitvoering}

TODO
%TODO: onderzoek

\subsection{Geëvalueerde afbeeldingsformaten}
\label{sec:onderzoek-afbeeldingsformaten}

TODO
%TODO: onderzoek

\subsection{Evaluatietool}
\label{sec:onderzoek-evaluatietool}

TODO
%TODO: onderzoek

\subsection{Deelnemers}
\label{sec:onderzoek-deelnemers}

TODO
%TODO: onderzoek

\section{Resultaten}
\label{sec:onderzoek-resultaten}

TODO
%TODO: onderzoek

\section{Besluit}
\label{sec:onderzoek-besluit}

TODO
%TODO: onderzoek
%deel 5
\chapter{Huidige en toekomstige uitdagingen}
\label{ch:uitdagingen}

%TODO: uitdagingen
%%=============================================================================
%% Conclusie
%%=============================================================================

\chapter{Conclusie}
\label{ch:conclusie}

\Gls{datacompressie}, en compressie in het algemeen is niets nieuw. Integendeel, het is één van de oudste concepten binnen IT en tot op heden van fundamenteel belang voor zowat alle IT-toepassingen. Een basiskennis over \gls{datacompressie} en de belangrijkste \glspl{afbeeldingsformaat} en video \gls{codec} is dan ook geen luxe binnen de IT-wereld. Veel vakgerelateerde opleidingen, zoals de opleiding Toegepaste Informatica te HoGent, voorzien echter geen lessen rond \gls{datacompressie}, waardoor deze basiskennis voor velen onbestaande is.

Deze bachelorproef biedt een oplossing voor dat probleem. Het vormt een gegronde basiskennis over \gls{datacompressie} zonder onnodig complex te zijn, wat het geschikt maakt voor de grote variatie van belanghebbenden. Vanaf het voorstel waren de doelstellingen van deze bachelorproef, een antwoord bieden op zeven onderzoeksvragen en één hoofdonderzoeksvraag. Deze onderzoeksvragen worden hieronder nog eens kort aangegaan met een terugblik naar de kennis verworven in deze bachelorproef.

\subsection*{Hoe is datacompressie binnen IT ontstaan?}
\label{sec:conclussie-onderzoeksvraag-1}

Vele onderzoekers zijn het erover eens dat \gls{datacompressie} dateert van voor de uitvinding van de computer. Zo kan morsecode gezien worden als een vorm van \gls{datacompressie}. Morsecode is uitgevonden voor het computertijdperk, in 1832, door Samuel F.B. Morse. Het kan gezien worden als een vorm van datacompressie doordat veelvoorkomende letters een kortere audiotoon kregen dan minder gebruikte letters (\cite{morsecode}).

\subsection*{Wat waren enkele van de eerste compressie-algoritmen?}
\label{sec:conclussie-onderzoeksvraag-2}

Enkele van de eerste \glspl{compressie-algoritme} komen aan bod in deel \ref{sec:ontstaan-datacompressie-primitieve-technieken-binnen-it} van deze bachelorproef. Twee belangrijke \glspl{compressie-algoritme} die al meer dan vijftig jaar bestaan, maar tot heden de basis vormen voor vele toepassingen binnen \gls{datacompressie} zijn \gls{rle-long} en \gls{huffman-coding}. De werking van deze \glspl{compressie-algoritme} is dan ook uitgebreid aan bod gekomen in deze bachelorproef. Een theoretische uitleg met een eenvoudig voorbeeld is voorzien in deel \ref{sec:primitieve-technieken-voorbeeld}. In de proof of concept \gls{compressietool} gemaakt voor deze bachelorproef zijn het ook deze twee \glspl{compressie-algoritme} die gebruikt worden. Deze \gls{compressietool} is verder toegelicht in hoofdstuk \ref{ch:compressietool}.

\subsection*{Waar zitten de verschillen tussen de afbeeldingsformaten en video codecs?}
\label{sec:conclussie-onderzoeksvraag-3}

\Glspl{afbeeldingsformaat} en video \glspl{codec} hebben meer gemeen dan oorspronkelijk gedacht zou worden. Vele \glspl{afbeeldingsformaat} vormen de basis voor een goed presterende video \glspl{codec} en de besproken \glspl{afbeeldingsformaat} \gls{webp} en \gls{heic} vinden juist hun ontstaan in \gls{videocompressie}. De onderlinge verschillen tussen de verschillende besproken \glspl{afbeeldingsformaat} en video \gls{codec} is af te leiden uit de voordelen en nadelen te vinden in hoofdstuk \ref{ch:afbeeldingscompressie} en \ref{ch:videocompressie}. De delen over het maken van een juiste keuze van \gls{afbeeldingsformaat} (deel \ref{sec:afbeeldingscompressie-keuze}) en video \gls{codec} (deel \ref{sec:videocompressie-keuze}) bieden aan de hand van enkele overzichten ook een duidelijk antwoord op deze vraag.

\subsection*{Hoe kan datacompressie correct geïmplementeerd worden?}
\label{sec:conclussie-onderzoeksvraag-4}

Hoe \gls{datacompressie} correct geïmplementeerd kan worden, is terug te vinden in verschillende delen van deze bachelorproef. De \gls{compressietool} en achterliggende code wordt uitgebreid besproken in hoofdstuk \ref{ch:compressietool}. Deze is \gls{open-source} ter beschikking gesteld op \gls{github} en kan zonder enige licenties gebruikt en aangepast worden. Er worden ook enkele beperkingen met deze \gls{compressietool} toegelicht en mogelijke oplossingen wat een geïnteresseerde lezer kan aanzetten deze beperkingen zelf weg te werken. Er wordt ook toegelicht hoe nieuwe generatie \glspl{afbeeldingsformaat} geïmplementeerd kunnen worden met ondersteuning voor alle internetbrowsers in gedachten. Dit is verder toegelicht in deel \ref{sec:afbeeldingscompressie-implementatie}.

\subsection*{Wat is het verschil tussen de afbeeldingsformaten: PNG, JPEG, JPEG2000, WebP en HEIF?}
\label{sec:onderzoeksvraag-5}

Elk \gls{afbeeldingsformaat} wordt toegelicht in hoofdstuk \ref{ch:afbeeldingscompressie}. Hier wordt zowel het ontstaan, de werking en voordelen en nadelen van de verschillende \glspl{afbeeldingsformaat} uitgelegd. Dit biedt samen met de resultaten van het onderzoek besproken in deel \ref{sec:onderzoek-resultaten} en \ref{sec:onderzoek-besluit} een uitgebreid inzicht in de verschillen tussen deze \glspl{afbeeldingsformaat}.

\subsection*{Wat is het verschil tussen de video codecs: H.264/AVC, H.264/SVC, H.265/HEVC en AV1?}
\label{sec:conclussie-onderzoeksvraag-6}

Elke video \glspl{codec} wordt toegelicht in hoofdstuk \ref{ch:videocompressie}. Hier wordt zowel het ontstaan als de voordelen en nadelen van de verschillende video \glspl{codec} aangekaart. Zoals in de overzichten van deel \ref{sec:videocompressie-keuze} duidelijk is weergegeven is er binnen \gls{videocompressie} een enorm probleem van complexe licenties. Het is ook daarom dat Google samenwerkt met tal van andere grote bedrijven als Mozilla en Microsoft en zo \gls{av1} op de markt heeft gebracht. Deze veelbelovende video \gls{codec} wordt ook in hoofdstuk \ref{ch:videocompressie} uitgebreid besproken.

\subsection*{Wat is DNA compressie en wat zijn andere uitdagingen binnen datacompressie?}
\label{sec:onderzoeksvraag-7}

Als afsluitend hoofdstuk (\ref{ch:uitdagingen}) is een korte vermelding van enkele huidige uitdagingen binnen \gls{datacompressie} toegelicht. Dit is bewust zeer beknopt gehouden zodat de lezer warm gemaakt wordt verder opzoekingswerk naar de interessante wereld van \gls{datacompressie} te verrichten!

\subsection{Hoofdonderzoeksvraag}
\label{sec:conclussie-hoofdonderzoeksvraag}

Door het beantwoorden van alle sub onderzoeksvragen kan de hoofdonderzoeksvraag, 'waarom moet er stilgestaan worden bij het gebruiken van \glspl{compressie-algoritme}, hoe kies je een geschikt \gls{compressie-algoritme} voor een bepaalde \gls{use-case} en hoe implementeer je dit best', door de lezer zelf beantwoord worden. Deze bachelorproef bevat namelijk alle nodige informatie om op een gegronde manier op zoek te gaan naar een \gls{compressie-algoritme} voor een bepaalde \gls{use-case}.

\subsection{Mogelijke uitbreidingen}
\label{sec:conclussie-uitbreidingen}

Desondanks deze bachelorproef reeds uit meer dan honderd pagina's bestaat, is er nog altijd ruimte voor uitbreidingen. Zo kan het onderzoek herwerkt worden zodat de verschillende afbeeldingen een gelijke bestandsgrootte hebben, wat een conclusie trekken makkelijker zal maken. Deze uitbreiding is relatief eenvoudig te voorzien doordat \gls{afbeeldingsevaluatietool}, die gemaakt is voor deze bachelorproef, \gls{open-source} is en gratis gebruikt mag worden.

Maar ook uitbreidingen op de gemaakte proof of concept \gls{compressietool} zijn mogelijk. Denk hierbij aan het combineren van \gls{rle-long} en \gls{huffman-coding} of het implementeren van een compleet nieuw \gls{compressie-algoritme}.



% Voeg hier je eigen hoofdstukken toe die de ``corpus'' van je bachelorproef
% vormen. De structuur en titels hangen af van je eigen onderzoek. Je kan bv.
% elke fase in je onderzoek in een apart hoofdstuk bespreken.

%\input{...}
%\input{...}
%...



%%=============================================================================
%% Bijlagen
%%=============================================================================
\appendix
\renewcommand{\chaptername}{Appendix}

%%---------- Onderzoeksvoorstel -----------------------------------------------

\chapter{Onderzoeksvoorstel}

Het onderwerp van deze bachelorproef is gebaseerd op een onderzoeksvoorstel dat vooraf werd beoordeeld door de promotor. Dat voorstel is opgenomen in deze bijlage.

% Verwijzing naar het bestand met de inhoud van het onderzoeksvoorstel
%---------- Inleiding ---------------------------------------------------------

\section{Introductie} % The \section*{} command stops section numbering
\label{sec:introductie}

Compressie is overal, bij katten video’s op YouTube, vakantiefoto’s op Instagram, zelfs bij het digitaal raadplegen van iemands DNA. Een wereld zonder compressie is ondenkbaar, er zouden enorme veelvouden van de huidige data opslag, brandbreedte en hardware capaciteit nodig moeten zijn om dezelfde data van vandaag te verwerken.

Bij DNA compressie is een verkleining van bestandsgrootte van meer dan 99 \% niet ongewoonlijk \autocite{Afify2011}. Bij afbeelding- en videocompressie kan een andere codec die een visueel gelijkaardig resultaat geeft een bestandsgrootte van factor tien hebben. Dit wilt zeggen dat compressie één van de belangrijkste factoren is, zeker vanuit het perspectief van de eindgebruiker, voor het optimaliseren van snelheid en kost bij applicatieontwikkeling en meer.

Bij een kleine bevraging van een tiental toegepaste informatica studenten te HoGent, één digital content team, twee mobile app developers en drie web developers bleek echter dat geen enkel van deze intensief bezig was met het bepalen van welke codec ze zullen gebruiken voor de afbeeldingen en video’s binnen hen project. Vrijwel iedereen wist wel dat het belangrijk was afbeeldingen en video’s te uploaden tegen een lagere resolutie maar de gebruikte codec verdedigen ging voor velen niet verder dan “het is voorgesteld door dit tooltje” of “bij JPEG heb je geen doorzichtige achtergrond”. 

Deze vaststelling was de triggerende factor voor deze bachelorproef. Een onderzoek naar waarom standaarden als JPEG en PNG nog niet vervangen zijn, welke interessanter is voor welk gebruik en hoeveel tijd en geld bespaard kan worden door minimale inspanning van de juiste codec keuze. 

Concreet zal het ontstaan van compressie en de fundamentele wiskunde achter de primitieve vormen van compressie besproken worden. De werking van JPEG en PNG compressie toegelicht worden om zo te kunnen bepalen welke beter is voor welke doeleinden. Videocompressie uitgelegd worden a.d.h.v. een vergelijkende studie tussen H.264/AVC en H.264/SVC. Ook zal er een blik geworpen worden op huidige en toekomstige uitdagingen voor compressie zoals DNA-compressie.

%---------- Stand van zaken ---------------------------------------------------

\section{Stand van zaken}
\label{sec:stand-van-zaken}




Hier beschrijf je de \emph{state-of-the-art} rondom je gekozen onderzoeksdomein. Dit kan bijvoorbeeld een literatuurstudie zijn. Je mag de titel van deze sectie ook aanpassen (literatuurstudie, stand van zaken, enz.). Zijn er al gelijkaardige onderzoeken gevoerd? Wat concluderen ze? Wat is het verschil met jouw onderzoek? Wat is de relevantie met jouw onderzoek?

Verwijs bij elke introductie van een term of bewering over het domein naar de vakliteratuur, bijvoorbeeld~\textcite{Salomon2006}! Denk zeker goed na welke werken je refereert en waarom.

% Voor literatuurverwijzingen zijn er twee belangrijke commando's:
% \autocite{KEY} => (Auteur, jaartal) Gebruik dit als de naam van de auteur
%   geen onderdeel is van de zin.
% \textcite{KEY} => Auteur (jaartal)  Gebruik dit als de auteursnaam wel een
%   functie heeft in de zin (bv. ``Uit onderzoek door Doll & Hill (1954) bleek
%   ...'')

Je mag gerust gebruik maken van subsecties in dit onderdeel.

%---------- Methodologie ------------------------------------------------------
\section{Methodologie}
\label{sec:methodologie}

Hier beschrijf je hoe je van plan bent het onderzoek te voeren. Welke onderzoekstechniek ga je toepassen om elk van je onderzoeksvragen te beantwoorden? Gebruik je hiervoor experimenten, vragenlijsten, simulaties? Je beschrijft ook al welke tools je denkt hiervoor te gebruiken of te ontwikkelen.

%---------- Verwachte resultaten ----------------------------------------------
\section{Verwachte resultaten}
\label{sec:verwachte_resultaten}

Hier beschrijf je welke resultaten je verwacht. Als je metingen en simulaties uitvoert, kan je hier al mock-ups maken van de grafieken samen met de verwachte conclusies. Benoem zeker al je assen en de stukken van de grafiek die je gaat gebruiken. Dit zorgt ervoor dat je concreet weet hoe je je data gaat moeten structureren.

%---------- Verwachte conclusies ----------------------------------------------
\section{Verwachte conclusies}
\label{sec:verwachte_conclusies}

Hier beschrijf je wat je verwacht uit je onderzoek, met de motivatie waarom. Het is \textbf{niet} erg indien uit je onderzoek andere resultaten en conclusies vloeien dan dat je hier beschrijft: het is dan juist interessant om te onderzoeken waarom jouw hypothesen niet overeenkomen met de resultaten.



%%---------- Andere bijlagen --------------------------------------------------
% TODO: Voeg hier eventuele andere bijlagen toe
%\input{...}

%%---------- Referentielijst --------------------------------------------------

%voorzie alle bronnen uit de db in referenties
\nocite{*}
%\phantomsection
\printbibliography[heading=bibintoc]

\end{document}
