%===============================================================================
% LaTeX sjabloon voor de bachelorproef toegepaste informatica aan HOGENT
% Meer info op https://github.com/HoGentTIN/bachproef-latex-sjabloon
%===============================================================================

\documentclass{bachproef-tin}

\usepackage{hogent-thesis-titlepage} % Titelpagina conform aan HOGENT huisstijl

%%---------- Documenteigenschappen ---------------------------------------------
% De titel van het rapport/bachelorproef
\title{Je kijkt er naar, maar ziet het niet: datacompressie	principes - JPEG en PNG - H.264/AVC en H.264/SVC}

% Je eigen naam
\author{Bontinck Lennert}

% De naam van je promotor (lector van de opleiding)
\promotor{Wim De Bruyn}

% De naam van je co-promotor. Als je promotor ook je opdrachtgever is en je
% dus ook inhoudelijk begeleidt (en enkel dan!), mag je dit leeg laten.
\copromotor{Tom Paridaens}

% Indien je bachelorproef in opdracht van/in samenwerking met een bedrijf of
% externe organisatie geschreven is, geef je hier de naam. Zoniet laat je dit
% zoals het is.
\instelling{---}

% Academiejaar
\academiejaar{2018-2019}

% Examenperiode
%  - 1e semester = 1e examenperiode => 1
%  - 2e semester = 2e examenperiode => 2
%  - tweede zit  = 3e examenperiode => 3
\examenperiode{2}

%===============================================================================
% Inhoud document
%===============================================================================

\begin{document}

%---------- Taalselectie -------------------------------------------------------
% Als je je bachelorproef in het Engels schrijft, haal dan onderstaande regel
% uit commentaar. Let op: de tekst op de voorkaft blijft in het Nederlands, en
% dat is ook de bedoeling!

%\selectlanguage{english}

%---------- Titelblad ----------------------------------------------------------
\inserttitlepage

%---------- Samenvatting, voorwoord --------------------------------------------
\usechapterimagefalse
%%=============================================================================
%% Voorwoord
%%=============================================================================

\chapter*{\IfLanguageName{dutch}{Woord vooraf}{Preface}}
\label{ch:voorwoord}

%% TODO:
%% Het voorwoord is het enige deel van de bachelorproef waar je vanuit je
%% eigen standpunt (``ik-vorm'') mag schrijven. Je kan hier bv. motiveren
%% waarom jij het onderwerp wil bespreken.
%% Vergeet ook niet te bedanken wie je geholpen/gesteund/... heeft


%%=============================================================================
%% Samenvatting
%%=============================================================================

% TODO: De "abstract" of samenvatting is een kernachtige (~ 1 blz. voor een
% thesis) synthese van het document.
%
% Deze aspecten moeten zeker aan bod komen:
% - Context: waarom is dit werk belangrijk?
% - Nood: waarom moest dit onderzocht worden?
% - Taak: wat heb je precies gedaan?
% - Object: wat staat in dit document geschreven?
% - Resultaat: wat was het resultaat?
% - Conclusie: wat is/zijn de belangrijkste conclusie(s)?
% - Perspectief: blijven er nog vragen open die in de toekomst nog kunnen
%    onderzocht worden? Wat is een mogelijk vervolg voor jouw onderzoek?
%
% LET OP! Een samenvatting is GEEN voorwoord!

%%---------- Nederlandse samenvatting -----------------------------------------
%
% TODO: Als je je bachelorproef in het Engels schrijft, moet je eerst een
% Nederlandse samenvatting invoegen. Haal daarvoor onderstaande code uit
% commentaar.
% Wie zijn bachelorproef in het Nederlands schrijft, kan dit negeren, de inhoud
% wordt niet in het document ingevoegd.

\IfLanguageName{english}{%
\selectlanguage{dutch}
\chapter*{Samenvatting}
\lipsum[1-4]
\selectlanguage{english}
}{}

%%---------- Samenvatting -----------------------------------------------------
% De samenvatting in de hoofdtaal van het document

\chapter*{\IfLanguageName{dutch}{Samenvatting}{Abstract}}

\lipsum[1-4]


%---------- Inhoudstafel -------------------------------------------------------
\pagestyle{empty} % Geen hoofding
\tableofcontents  % Voeg de inhoudstafel toe
\cleardoublepage  % Zorg dat volgende hoofstuk op een oneven pagina begint
\pagestyle{fancy} % Zet hoofding opnieuw aan

%---------- Lijst figuren, afkortingen, ... ------------------------------------

% Indien gewenst kan je hier een lijst van figuren/tabellen opgeven. Geef in
% dat geval je figuren/tabellen altijd een korte beschrijving:
%
%  \caption[korte beschrijving]{uitgebreide beschrijving}
%
% De korte beschrijving wordt gebruikt voor deze lijst, de uitgebreide staat bij
% de figuur of tabel zelf.

\listoffigures
\listoftables

% Als je een lijst van afkortingen of termen wil toevoegen, dan hoort die
% hier thuis. Gebruik bijvoorbeeld de ``glossaries'' package.
% https://www.overleaf.com/learn/latex/Glossaries

%---------- Kern ---------------------------------------------------------------

% De eerste hoofdstukken van een bachelorproef zijn meestal een inleiding op
% het onderwerp, literatuurstudie en verantwoording methodologie.
% Aarzel niet om een meer beschrijvende titel aan deze hoofstukken te geven of
% om bijvoorbeeld de inleiding en/of stand van zaken over meerdere hoofdstukken
% te verspreiden!

%%=============================================================================
%% Inleiding
%%=============================================================================

\chapter{Inleiding}
\label{ch:inleiding}

\Gls{datacompressie}, en compressie in het algemeen is niets nieuw. Integendeel, het is één van de oudste concepten binnen IT en tot op heden van fundamenteel belang voor zowat alle IT-toepassingen. Door de databesparing kan veelvouden sneller en goedkoper gewerkt worden. Datacompressie en de altijd groeiende waaier aan \glspl{compressie-algoritme} maken het ook mogelijk mt data te verwerken waar voorheen geen beginnen aan was. Denk hierbij bijvoorbeeld aan recente doorbraken binnen \gls{dna-compressie} die het mogelijk maken steeds meer onderzoeken met betrekking tot het menselijk genoom uit te voeren.

Er zijn reeds tal van uitgebreide en professionele papers geschreven rond zowel \gls{datacompressie} als \gls{afbeeldingscompressie} en \gls{videocompressie} in het bijzonder. Dit is logisch, want het zijn onderwerpen die van fundamenteel belang zijn binnen IT. De focus van deze paper ligt dan ook niet op het herschrijven van papers die reeds bestaan maar aan het maken van een nuttig document voor iedereen binnen de IT-wereld. 

Door de verworven kennis zal je als programmeur, content creator of eender welk andere belanghebber inzicht krijgen hoe \gls{datacompressie} ontstaan is en werkt. Er zal verklaring komen rond de verschillende mogelijkheden voor \gls{afbeeldingscompressie} en \gls{videocompressie}. Dit zal inzicht geven tot waarom het zo belangrijk is voor de juiste \glspl{codec} en \glspl{afbeeldingsformaat} te kiezen.

Er zal een onderzoek uitgevoerd worden welk \gls{afbeeldingsformaat} het best is voor een bepaalde \gls{use-case}: portret foto's op sociale media. De hiervoor ontwikkelde en gratis te gebruiken open source  \gls{afbeeldingsevaluatietool} zal je dan ook de mogelijkheid bieden zelf een subjectief onderzoek te voeren binnen je doelpubliek.

%todo verder en mss andere locatie etc



\section{Probleemstelling}
\label{sec:probleemstelling}
TODO

%TODO: verder 
%De inleiding moet de lezer net genoeg informatie verschaffen om het onderwerp te begrijpen en in te zien waarom de onderzoeksvraag de moeite waard is om te onderzoeken. In de inleiding ga je literatuurverwijzingen beperken, zodat de tekst vlot leesbaar blijft. Je kan de inleiding verder onderverdelen in secties als dit de tekst verduidelijkt. Zaken die aan bod kunnen komen in de inleiding~\autocite{Pollefliet2011}:
%\begin{itemize}
%\item context, achtergrond
%\item afbakenen van het onderwerp
%\item verantwoording van het onderwerp, methodologie
%\item probleemstelling
%\item onderzoeksdoelstelling
%\item onderzoeksvraag
%\item \ldots
%\end{itemize}

\section{Onderzoeksvragen}
\label{sec:onderzoeksvragen}
TODO

%TODO: verder 
%Wees zo concreet mogelijk bij het formuleren van je onderzoeksvraag. Een onderzoeksvraag is trouwens iets waar nog niemand op dit moment een antwoord heeft (voor zover je kan nagaan). Het opzoeken van bestaande informatie (bv. ``welke tools bestaan er voor deze toepassing?'') is dus geen onderzoeksvraag. Je kan de onderzoeksvraag verder specifiëren in deelvragen. Bv.~als je onderzoek gaat over performantiemetingen, dan 

\section{Onderzoeksdoelstelling}
\label{sec:onderzoeksdoelstelling}
TODO

%TODO: verder 
%Wat is het beoogde resultaat van je bachelorproef? Wat zijn de criteria voor succes? Beschrijf die zo concreet mogelijk. Gaat het bv. om een proof-of-concept, een prototype, een verslag met aanbevelingen, een vergelijkende studie, enz.

\section{Opzet van deze bachelorproef}
\label{sec:opzet-bachelorproef}
TODO

% Het is gebruikelijk aan het einde van de inleiding een overzicht te
% geven van de opbouw van de rest van de tekst. Deze sectie bevat al een aanzet
% die je kan aanvullen/aanpassen in functie van je eigen tekst.

\subsection{Deel 1: situering en literatuurstudie}
\label{sec:opzet-bachelorproef-deel-1}

Deze paper zal zich in het eerste deel focussen op het toelichten van de belangrijke termen binnen \gls{datacompressie}. In hoofdstuk \ref{ch:termen} is een lijst met belangrijker termen te vinden die binnen \gls{datacompressie} en deze paper vaak voorkomen. Doorheen deze paper zullen tal van referenties naar deze termen gelegd worden. 

Hoofdstuk \ref{ch:methodologie} licht de gebruikte methodologie voor deze paper toe. Hieruit wordt duidelijk dat deze paper zo objectief mogelijk is opgesteld met een focus op duidelijkheid en reproduceerbaarheid.

Hoofdstuk \ref{ch:literatuurstudie} behoort ook tot het situerende eerste deel en zal het ontstaan van \gls{datacompressie} en enkele basisprincipes toelichten. Een reeks van deze primitieve technieken zullen aan de hand van een voorbeeld toegelicht worden.
%todo: link naar voorbeeld en welke manieren

\subsection{Deel 2: compressie tool ontwikkelen}
\label{sec:opzet-bachelorproef-deel-2}
 
 In het tweede zal een basis \gls{datacompressie} tool programmatisch geïmplementeerd worden om de theorie uit het eerste deel in praktijk te brengen.
 
Hoofdstuk \ref{ch:compressietool} is hierdoor gericht voor technische lezers als programmeurs. Er is echter telkens voldoende randinformatie gegeven zodanig ook de minder technische lezers een blik achter de schermen kunnen verkrijgen.
 %TODO: welke taal tool gemaakt is en wat deze juist doet etc

\subsection{Deel 3: afbeelding- en videocompressie}
\label{sec:opzet-bachelorproef-deel-3}

In het derde deel worden twee subdomeinen van \gls{datacompressie} verder toegelicht: \gls{afbeeldingscompressie} en \gls{videocompressie}. 

In hoofdstuk \ref{ch:afbeeldingscompressie} zal er dieper ingegaan worden op volgende \glspl{afbeeldingsformaat} voor \gls{afbeeldingscompressie}: \gls{jpeg}, \gls{jpeg2000} en \gls{png}. 
%TODO: aanpassen indien andere op input copromoter

In hoofdstuk \ref{ch:videocompressie} zal er verder ingegaan worden op de gekende \gls{videocompressie} standaarden: \gls{h264-avc} en \gls{h264-svc}. Ook de opvolger \gls{h265} en het open source alternatief \gls{av1} zullen besproken worden.
%TODO: aanpassen indien andere op input copromoter

\subsection{Deel 4: onderzoek afbeelding compressie}
\label{sec:opzet-bachelorproef-deel-4}

In het vierde deel wordt besproken hoe compressiemethoden voor video's en afbeeldingen geëvalueerd worden. Hoofdstuk \ref{ch:kwaliteit} zal enkele veel gebruikte tools en methoden voor objectieve en subjectieve vergelijkingen toelichten.

In hoofdstuk \ref{ch:onderzoek} werd een subjectieve test voor het evalueren van afbeeldingskwaliteit uitgevoerd. Deze test focust zich op portretfoto's. Hierbij zullen enkele van de besproken \glspl{afbeeldingsformaat} uit hoofdstuk \ref{ch:afbeeldingscompressie} tegen elkaar concurreren. De gebruikte tool is voor deze paper opgesteld en is gratis \gls{open-source} toegankelijk gesteld wat het eenvoudig mogelijk maakt om een gelijkaardig onderzoek uit te voeren.
%TODO: in staat stelllen als bv prog of content zelf kiezen welk gebruiken

\subsection{Deel 5: uitdagingen en conclusie}
\label{sec:opzet-bachelorproef-deel-5}

In het vijfde deel zullen de huidige uitdagingen van \gls{datacompressie} kort toegelicht worden. Zo zal hoofdstuk \ref{ch:uitdagingen} een beeld geven van de taken die mensen als Tom Paridaens, co-promoter voor deze paper, krijgen.
%TODO: copromoter zijn job en eerbetoon vermelden

In hoofdstuk \ref{ch:conclusie} wordt kort teruggeblikt op de paper en worden enkele besluiten uit het onderzoek van hoofdstuk \ref{ch:onderzoek} opgesomd. Daarbij wordt ook een aanzet gegeven om zelf meer na te denken over het gebruik van \gls{datacompressie} en bepaalde  \glspl{codec} in projecten, of nog beter, zelf een onderzoek uit te voeren!
%TODO: lezer aanzetten meer etc
\chapter{\IfLanguageName{dutch}{Stand van zaken}{State of the art}}
\label{ch:stand-van-zaken}

% Tip: Begin elk hoofdstuk met een paragraaf inleiding die beschrijft hoe
% dit hoofdstuk past binnen het geheel van de bachelorproef. Geef in het
% bijzonder aan wat de link is met het vorige en volgende hoofdstuk.

% Pas na deze inleidende paragraaf komt de eerste sectiehoofding.

Dit hoofdstuk bevat je literatuurstudie. De inhoud gaat verder op de inleiding, maar zal het onderwerp van de bachelorproef *diepgaand* uitspitten. De bedoeling is dat de lezer na lezing van dit hoofdstuk helemaal op de hoogte is van de huidige stand van zaken (state-of-the-art) in het onderzoeksdomein. Iemand die niet vertrouwd is met het onderwerp, weet nu voldoende om de rest van het verhaal te kunnen volgen, zonder dat die er nog andere informatie moet over opzoeken \autocite{Pollefliet2011}.

Je verwijst bij elke bewering die je doet, vakterm die je introduceert, enz. naar je bronnen. In \LaTeX{} kan dat met het commando \texttt{$\backslash${textcite\{\}}} of \texttt{$\backslash${autocite\{\}}}. Als argument van het commando geef je de ``sleutel'' van een ``record'' in een bibliografische databank in het Bib\LaTeX{}-formaat (een tekstbestand). Als je expliciet naar de auteur verwijst in de zin, gebruik je \texttt{$\backslash${}textcite\{\}}.
Soms wil je de auteur niet expliciet vernoemen, dan gebruik je \texttt{$\backslash${}autocite\{\}}. In de volgende paragraaf een voorbeeld van elk.

\textcite{Knuth1998} schreef een van de standaardwerken over sorteer- en zoekalgoritmen. Experten zijn het erover eens dat cloud computing een interessante opportuniteit vormen, zowel voor gebruikers als voor dienstverleners op vlak van informatietechnologie~\autocite{Creeger2009}.

\lipsum[7-20]

%%=============================================================================
%% Methodologie
%%=============================================================================

\chapter{Methodologie}
\label{ch:methodologie}

Bij het schrijven van een wetenschappelijk document is een gegronde methodologie vereist. Dit hoofdstuk licht de gekozen methodologie voor deze bachelorproef toe en bespreekt die keuze per deel van dit document. Ook de gebruikte \LaTeX{} packages worden kort toegelicht.

\section{Aanpak van deze bachelorproef}
\label{sec:aanpak-bachelorproef}

Deze bachelorproef streeft naar een correcte en verantwoordelijke manier van werken om de betrouwbaarheid van dit document te bewaren. Hiervoor is dit document meermaals gevalideerd door en veranderd naar input van vakexpert en co-promotor Tom Paridaens. 

Om dit te garanderen is er ook uitsluitend gebruik gemaakt van primaire of secundaire bronnen en geen tertiaire bronnen. De kennis verworven uit primaire bronnen is steeds gevalideerd met secundaire bronnen. De volledig bronnenlijst is ter beschikking gesteld op het einde van dit document. Wanneer data uit een bron overgenomen is naar dit document is steeds een verwijzing naar de oorspronkelijke bron voorzien. 

De kennis uit primaire brommen komen voornamelijk uit de vijf jaar informatica gerelateerde studies en twee jaar fotografie gerelateerde studies dat door de auteur van deze bachelorproef gevolgd zijn.

Dit document is geschreven in \LaTeX{} en is voorzien van een BibTeX bibliografische databank. Er is onder andere gebruik gemaakt van volgende packages: 

\begin{itemize}
	
	\item Glossary voor het voorzien van een woordenlijst. Woorden die voorkomen in de woordenlijst uit hoofdstuk \ref{ch:termen} bevatten een verwijzing naar dit overzicht wanneer er op geklikt wordt.
	
	\item Listings en colors voor het voorzien van code met de gepaste highlighting. 
	
	\item Xcolor voor het voorzien van kleuren achtergrondkleuren in de cellen van een tabel.
	
	\item Placeins voor meer controle over de plaatsing van tabellen en andere figuren door \LaTeX{}.
	
\end{itemize}

In de volgende secties zal voor de vijf verschillende delen van dit document kort toegelicht worden wat de gekozen methodologie is en waarom.

\subsection{Deel 1: situering en literatuurstudie}
\label{sec:aanpak-bachelorproef-deel-1}

In hoofdstuk \ref{ch:termen} is er voor gekozen een lijst van belangrijke termen te voorzien. Deze lijst helpt de lezer de bachelorproef vlot te lezen. Er is gebruik gemaakt van de Glossary package omdat deze de mogelijkheid voor referenties en een alfabetisch gerangschikte woordenlijst voorziet.

De literatuurstudie (hoofdstuk \ref{ch:literatuurstudie}) is kort gehouden maar volstaat samen met de woordenlijst uit hoofdstuk \ref{ch:termen} voor het begrijpen van de overige besproken zaken uit dit document. Deze literatuurstudie licht ook enkele primitieve \glspl{compressie-algoritme} toe en maakt daarvoor gebruik van de originele documenten omtrent de uitgave van deze \glspl{compressie-algoritme}. 

In dit deel worden de volgende onderzoekvragen (deels) beantwoord: 
\begin{itemize}
	\item Hoe is \gls{datacompressie} binnen IT ontstaan?
	\item Wat waren enkele van de eerste \glspl{compressie-algoritme}?
	\item Waar zitten de verschillen tussen de \glspl{afbeeldingsformaat} en video \glspl{codec}?
	\item Hoe kan \gls{datacompressie} correct geïmplementeerd worden?
\end{itemize}

\subsection{Deel 2: datacompressietool ontwikkelen}
\label{sec:aanpak-bachelorproef-deel-2}

Aan de hand van de verworven kennis uit de hoofdstuk \ref{ch:literatuurstudie} is een proof of concept \gls{compressietool} geschreven in \gls{php} met een grafische interface in \gls{html}, \gls{css} en \gls{bootsrap}. Er is gekozen om de \gls{compressietool} in deze talen te schrijven aangezien deze eenvoudig online te hosten zijn of lokaal te runnen. De tool is dan ook online ter beschikking gesteld op de website van Lennert Bontinck\urlcite{compressietool}. Dit maakt het voor de lezer eenvoudig om de \gls{compressietool} zelf te testen.

De code van de \gls{compressietool} is publiek ter beschikking gesteld op de \gls{github} repository van deze bachelorproef\urlcite{githubbachelorproef}. Deze is vrijgegeven onder de GNU GPLv3 licentie en mag dus gratis aangepast en gebruikt worden voor alle doeleinden. Dit maakt het eenvoudig voor de lezer om de broncode raad te plegen en, aan de hand van de uitleg in hoofdstuk \ref{ch:compressietool}, aan te passen.

Er is bewust gekozen om de code achter deze tool simpel te houden en te werken met Nederlandse variabelen zodanig de code, mits de voorziene uitleg in hoofdstuk \ref{ch:compressietool}, ook voor minder technische lezers verstaanbaar is. Hiervoor zijn ook tal van links naar de geziene theorie uit deel \ref{sec:primitieve-technieken-voorbeeld} voorzien.

De verworven bestanden na compressie worden met het origineel vergeleken op basis van het aantal karakters nodig om de tekst op te slaan voor de \gls{rle-long} gebaseerde \glspl{compressie-algoritme} en het aantal bits nodig om de tekst op te slaan voor het \gls{huffman-coding} gebaseerde \gls{compressie-algoritme}. Dit geeft een duidelijker beeld van de prestatie van het \gls{compressie-algoritme} dan de zuiver de bestandsgrootte.

Aangezien het om een proof of concept \gls{compressietool} gaat zijn er enkele beperkingen, deze worden dan ook toegelicht in deel \ref{sec:compressietool-beperkingen}. Dit doet de gebruiker stilstaan over mogelijke (ongewenste) beperkingen die kunnen voorkomen bij het implementeren van een \gls{compressie-algoritme} en nadenken over mogelijke oplossingen.

In dit deel worden de volgende onderzoekvragen (deels) beantwoord: 
\begin{itemize}
	\item Wat waren enkele van de eerste \glspl{compressie-algoritme}?
	\item Hoe kan \gls{datacompressie} correct geïmplementeerd worden?
\end{itemize}

\subsection{Deel 3: afbeelding- en videocompressie}
\label{sec:aanpak-bachelorproef-deel-3}

In dit deel is er voor gekozen om de volgende \glspl{afbeeldingsformaat} toe te lichten: \gls{png} | \gls{jpeg} | \gls{jpeg2000} | \gls{webp} | \gls{heif}. \Gls{png} en \gls{jpeg} zijn namelijk de bekendste en op het web meest gebruikte \glspl{afbeeldingsformaat}. \Gls{jpeg2000}, \gls{webp} en \gls{heif} zijn dan weer enkele van de bekendste nieuwe generatie \glspl{afbeeldingsformaat}. Dit zorgt er voor dat de besproken \glspl{afbeeldingsformaat} diegene zijn dat het meeste potentieel hebben om een goede keuze te zijn voor het doelpubliek van deze bachelorproef.

Voor elk \gls{afbeeldingsformaat} is kort het ontstaan toegelicht en aan de hand van de bijhorende \gls{iso} de werking (voor \gls{png} en \gls{jpeg} diepgaander dan de anderen) uitgelegd. Buiten de \gls{iso} is ook veel informatie over de werking van enkele van de besproken \glspl{afbeeldingsformaat} uit het uitgebreide boek rond \gls{datacompressie}: \citetitle{Salomon2006} (\cite{Salomon2006}) gehaald. Deze \gls{iso} is door de maker mee opgesteld en garandeert dus dat de werking van het \gls{afbeeldingsformaat} juist beschreven is. De belangrijkste voordelen en nadelen zijn ook steeds toegelicht zodanig de lezer zelf een beeld kan scheppen welk \gls{afbeeldingsformaat} geschikt is voor zijn \gls{use-case}. De overzichtstabellen in deel \ref{sec:afbeeldingscompressie-functievereisten} en \ref{sec:afbeeldingscompressie-ondersteuning} omtrent functievereisten en ondersteuning helpen de lezer hier ook bij.

Ook de betekenis van \gls{raw} afbeeldingen wordt hier toegelicht en waarom ze belangrijk zijn om op een objectieve manier een onderzoek te voeren naar de prestatie van een \gls{afbeeldingsformaat}.

Om te vermeiden dat lezers schrik hebben om nieuwe \glspl{afbeeldingsformaat} te gebruiken is in deel \ref{sec:afbeeldingscompressie-implementatie} besproken hoe je deze nieuwe \glspl{afbeeldingsformaat} eenvoudig kan implementeren. Ook oplossingen voor situaties waar de gebruiker geen ondersteuning heeft voor deze nieuwe generatie \glspl{afbeeldingsformaat} wordt besproken. Er zijn ook enkele geautomatiseerde oplossingen voor het voorzien van deze nieuwe \glspl{afbeeldingsformaat} besproken in deel \ref{sec:afbeeldingscompressie-implementatie-web-automated}. Dit helpt de lezer inzicht te geven hoe de implementatie zal verlopen voor zijn \gls{use-case}.

In hoofdstuk \ref{ch:videocompressie} zijn \gls{h264-avc}, \gls{h264-svc}, \gls{h265} en \gls{av1} besproken. Dit omdat \gls{h264-avc}, \gls{h264-svc} en \gls{h265} de bekendste video \glspl{codec} zijn en \gls{av1} een gekende \gls{open-source} alternatief is voor vrij gebruik. Ook hier is er gekozen om kort het ontstaan toe te lichten alsook de voordelen en de nadelen. Ook de werking wordt hier kort aangegaan met ondersteuning van de bijhorende \gls{iso}. De belangrijkste voordelen en nadelen worden voor elke video \gls{codec} toegelicht en helpen de lezer samen met de punten uit deel \ref{sec:videocompressie-keuze} een juiste keuze te maken.

In dit deel worden de volgende onderzoekvragen (deels) beantwoord: 
\begin{itemize}
	\item Waar zitten de verschillen tussen de \glspl{afbeeldingsformaat} en video \glspl{codec}?
	\item Hoe kan \gls{datacompressie} correct geïmplementeerd worden?
	\item Wat is het verschil tussen de \glspl{afbeeldingsformaat}: \gls{png}, \gls{jpeg}, \gls{jpeg2000}, \gls{webp} en \gls{heif}
	\item Wat is het verschil tussen de video \glspl{codec}: \gls{h264-avc}, \gls{h264-svc}, \gls{h265} en \gls{av1}?
	\item Hoe kan \gls{datacompressie} correct geïmplementeerd worden?
\end{itemize}

\subsection{Deel 4: onderzoek afbeeldingscompressie}
\label{sec:aanpak-bachelorproef-deel-4}

In hoofdstuk \ref{ch:onderzoek} wordt een subjectief onderzoek naar de afbeeldingskwaliteit van \gls{png}, \gls{jpeg}, \gls{jpeg2000} en \gls{webp} gevoerd. Er is gekozen voor een subjectief onderzoek omdat dit voor de \gls{use-case} de aangeraden manier van werken is. Voor dit subjectieve onderzoek is een \gls{afbeeldingsevaluatietool} geschreven in \gls{php}, \gls{sql}, \gls{html}, \gls{css}, \gls{js} en \gls{drift}. Net zoals de \gls{compressietool} is deze publiek ter beschikking gesteld op de \gls{github} repository van deze bachelorproef\urlcite{githubbachelorproef}. Deze is vrijgegeven onder de GNU GPLv3 licentie en mag dus gratis aangepast en gebruikt worden voor alle doeleinden. Dit maakt het eenvoudig voor de lezer om zelf, aan de hand van de uitleg in deel \ref{sec:onderzoek-evaluatietool}, een subjectief onderzoek te voeren.

De resultaten van het onderzoek worden met \gls{r} scripts in overzichtelijke grafieken en tabellen gezet. Deze \gls{r} scripts zijn samen met de resultaten beschikbaar op de \gls{github} van deze bachelorproef. Er is bewust gekomen om de lezer zelf de kans te geven een besluit te trekken aan de hand van de geziene theorie en de resultaten.

%todo: verder aanvullen wanneer onderzoek en bespreking gedaan.

In dit deel worden de volgende onderzoekvragen (deels) beantwoord: 
\begin{itemize}
	\item Waar zitten de verschillen tussen de \glspl{afbeeldingsformaat} en video \glspl{codec}?
	\item Wat is het verschil tussen de \glspl{afbeeldingsformaat}: \gls{png}, \gls{jpeg}, \gls{jpeg2000}, \gls{webp} en \gls{heif}
\end{itemize}

\subsection{Deel 5: uitdagingen en conclusie}
\label{sec:aanpak-bachelorproef-deel-5}

Tot slot is er gekozen om nog enkele van de uitdagingen binnen \gls{datacompressie} toe te lichten aan de lezer. Dit is deels als ode voor vakexpert en co-promotor Tom Paridaens maar ook om de lezer warm te maken zich verder te verdiepen in deze uitdagingen.

In de conclusie (hoofdstuk \ref{ch:conclusie}) wordt op een objectieve manier teruggeblikt naar de antwoorden dat deze bachelorproef biedt alsook naar interessante vragen die ze opwekt. Het grote doel is om de lezer te overtuigen zich verder te verdiepen in \gls{datacompressie} en meer stil te staan bij de keuze van een \gls{compressie-algoritme}.

In dit deel wordt het antwoord op de volgende onderzoekvragen gegeven: 'Wat is \gls{dna-compressie} en wat zijn andere uitdagingen binnen \gls{datacompressie}?'. In de conclusie (hoofdstuk \ref{ch:conclusie}) wordt nog eens teruggeblikt op alle delen waardoor ook het antwoord op de hoofdonderzoeksvraag in dit deel gegeven wordt.

% Voeg hier je eigen hoofdstukken toe die de ``corpus'' van je bachelorproef
% vormen. De structuur en titels hangen af van je eigen onderzoek. Je kan bv.
% elke fase in je onderzoek in een apart hoofdstuk bespreken.

%\input{...}
%\input{...}
%...

%%=============================================================================
%% Conclusie
%%=============================================================================

\chapter{Conclusie}
\label{ch:conclusie}

\Gls{datacompressie}, en compressie in het algemeen is niets nieuw. Integendeel, het is één van de oudste concepten binnen IT en tot op heden van fundamenteel belang voor zowat alle IT-toepassingen. Een basiskennis van \gls{datacompressie} en de belangrijkste \glspl{afbeeldingsformaat} en video \gls{codec} is dan ook geen luxe binnen de IT-wereld. Veel vakgerelateerde opleidingen, zoals de opleiding Toegepaste Informatica te HoGent, voorzien echter geen lessen rond \gls{datacompressie} waardoor deze basiskennis voor velen onbestaande is.

Deze bachelorproef bied een oplossing voor dat probleem. Het vormt een gegronde basiskennis van \gls{datacompressie} zonder onnodig complex te zijn wat het geschikt maakt voor de grote variatie van belanghebbende. Vanaf het voorstel waren de doelstellingen van deze bachelorproef, een antwoord bieden op zeven onderzoeksvragen en één hoofdonderzoeksvraag. Deze onderzoeksvragen worden hieronder nog eens kort aangegaan met een terugblik naar de kennis verworven in deze bachelorproef.

\subsection*{Hoe is datacompressie binnen IT ontstaan?}
\label{sec:conclussie-onderzoeksvraag-1}

Vele onderzoekers zijn het erover eens dat \gls{datacompressie} dateert van voor de uitvinding van de computer. Zo kan morsecode gezien worden als een vorm van \gls{datacompressie}. Morsecode is uitgevonden voor het computertijdperk, in 1832, door Samuel F.B. Morse. Het kan aanzien worden als een vorm van datacompressie doordat veel voorkomende letters een kortere audiotoon kregen dan minder gebruikte letters (\cite{morsecode}).

\subsection*{Wat waren enkele van de eerste compressie-algoritmen?}
\label{sec:conclussie-onderzoeksvraag-2}

Enkele van de eerste \glspl{compressie-algoritme} komen aan bod in deel \ref{sec:ontstaan-datacompressie-primitieve-technieken-binnen-it} van deze bachelorproef. Twee belangrijke \glspl{compressie-algoritme} dat al meer dan vijftig jaar bestaan maar tot heden de basis vormen voor vele toepassingen binnen \gls{datacompressie} zijn \gls{rle-long} en \gls{huffman-coding}. De werking van deze \glspl{compressie-algoritme} is dan ook uitgebreid aan bod gekomen in deze bachelorproef. Een theoretische uitleg met een eenvoudig voorbeeld is voorzien in deel \ref{sec:primitieve-technieken-voorbeeld}. In de proof of concept \gls{compressietool} gemaakt voor deze bachelorproef zijn het ook deze twee \glspl{compressie-algoritme} dat gebruikt worden. Deze \gls{compressietool} is verder toegelicht in hoofdstuk \ref{ch:compressietool}.

\subsection*{Waar zitten de verschillen tussen de afbeeldingsformaten en video codecs?}
\label{sec:conclussie-onderzoeksvraag-3}

\Glspl{afbeeldingsformaat} en video \glspl{codec} hebben meer gemeen dan oorspronkelijk gedacht zo worden. Vele \glspl{afbeeldingsformaat} vormen de basis voor een goed presterende video \glspl{codec} en de besproken \glspl{afbeeldingsformaat} \gls{webp} en \gls{heic} vinden juist hun ontstaan in \gls{videocompressie}. De onderlinge verschillen tussen de verschillende besproken \glspl{afbeeldingsformaat} en video \glspl{codec} is af te leiden uit de voordelen en nadelen te vinden in hoofdstuk \ref{ch:afbeeldingscompressie} en \ref{ch:videocompressie}. De delen over het maken van een juiste keuze van \gls{afbeeldingsformaat} (deel \ref{sec:afbeeldingscompressie-keuze}) en video \gls{codec} (deel \ref{sec:videocompressie-keuze}) bieden aan de hand van enkele overzichten ook een duidelijk antwoord op deze vraag.

\subsection*{Hoe kan datacompressie correct geïmplementeerd worden?}
\label{sec:conclussie-onderzoeksvraag-4}

Hoe \gls{datacompressie} correct geïmplementeerd kan worden is terug te vinden in verschillende porties van deze bachelorproef. De \gls{compressietool} en achterliggende code wordt uitgebreid besproken in hoofdstuk \ref{ch:compressietool}. Deze is \gls{open-source} ter beschikking gesteld op \gls{github} en kan zonder enige licenties gebruikt en aangepast worden. Er worden ook enkele beperkingen met deze \gls{compressietool} toegelicht en mogelijke oplossingen wat een geïnteresseerde lezer kan aanzetten deze beperkingen zelf weg te werken. Er wordt ook toegelicht hoe nieuwe generatie \glspl{afbeeldingsformaat} geïmplementeerd kunnen worden met ondersteuning voor alle internetbrowsers in gedachten. Dit is verder toegelicht in deel \ref{sec:afbeeldingscompressie-implementatie}.

\subsection*{Wat is het verschil tussen de afbeeldingsformaten: PNG, JPEG, JPEG2000, WebP en HEIF?}
\label{sec:onderzoeksvraag-5}

Elk \gls{afbeeldingsformaat} wordt toegelicht in hoofdstuk \ref{ch:afbeeldingscompressie}. Hier wordt zowel het ontstaan, de werking en voordelen en nadelen van de verschillende \glspl{afbeeldingsformaat} uitgelegd. Dit bied samen met de resultaten van het onderzoek besproken in deel \ref{sec:onderzoek-resultaten} en \ref{sec:onderzoek-besluit} een uitgebreid inzicht van de verschillen tussen deze \glspl{afbeeldingsformaat}.

\subsection*{Wat is het verschil tussen de video codecs: H.264/AVC, H.264/SVC, H.265/HEVC en AV1?}
\label{sec:conclussie-onderzoeksvraag-6}

Elke video \glspl{codec} wordt toegelicht in hoofdstuk \ref{ch:videocompressie}. Hier wordt zowel het ontstaan als de voordelen en nadelen van de verschillende video \glspl{codec} aangekaart. Zoals in de overzichten van deel \ref{sec:videocompressie-keuze} duidelijk is weergegeven is er binnen \gls{videocompressie} een enorm probleem van complexe licenties. Het is ook daarom dat Google samenwerkt met tal van andere grote bedrijven als Mozilla en Microsoft \gls{av1} op de markt heeft gebracht. Deze veelbelovende video \gls{codec} wordt ook in hoofdstuk \ref{ch:videocompressie} uitgebreid besproken.

\subsection*{Wat is DNA compressie en wat zijn andere uitdagingen binnen datacompressie?}
\label{sec:onderzoeksvraag-7}

Als afsluitend hoofdstuk (\ref{ch:uitdagingen}) is een korte vermelding van enkele huidige uitdagingen binnen \gls{datacompressie} toegelicht. Dit is bewust zeer beknopt gehouden zodanig de lezer warm gemaakt wordt verder opzoekingswerk naar de interessante wereld van \gls{datacompressie} te verrichten!

\subsection{Hoofdonderzoeksvraag}
\label{sec:conclussie-hoofdonderzoeksvraag}

Door het beantwoorden van alle sub onderzoeksvragen kan de hoofdonderzoeksvraag, Waarom moet er stilgestaan worden bij het gebruiken van \glspl{compressie-algoritme}, hoe kies je een geschikt \gls{compressie-algoritme} voor een bepaalde \gls{use-case} en hoe implementeer je dit het best, door de lezer zelf beantwoord worden. Deze bachelorproef bevat namelijk alle nodige informatie om op een gegronde manier op zoek te gaan naar een \gls{compressie-algoritme} voor een bepaalde \gls{use-case}.

\subsection{Mogelijke uitbreidingen}
\label{sec:conclussie-uitbreidingen}

Desondanks deze bachelorproef reeds uit meer dan honderd pagina's bestaat is er nog altijd ruimte voor uitbreidingen. Zo kan het onderzoek herwerkt worden zodanig de verschillende afbeeldingen een gelijke bestandsgrootte hebben wat een conclusie trekken makkelijker zal maken. Deze uitbreiding is relatief simpel te voorzien door de \gls{open-source} en gratis in gebruik \gls{afbeeldingsevaluatietool} dat gemaakt is voor deze bachelorproef.

Maar ook uitbreidingen op de gemaakte proof of concept \gls{compressietool} zijn mogelijk. Denk hierbij aan het combineren van \gls{rle-long} en \gls{huffman-coding} of het implementeren van een compleet nieuw \gls{compressie-algoritme}.



%%=============================================================================
%% Bijlagen
%%=============================================================================

\appendix
\renewcommand{\chaptername}{Appendix}

%%---------- Onderzoeksvoorstel -----------------------------------------------

\chapter{Onderzoeksvoorstel}

Het onderwerp van deze bachelorproef is gebaseerd op een onderzoeksvoorstel dat vooraf werd beoordeeld door de promotor. Dat voorstel is opgenomen in deze bijlage.

% Verwijzing naar het bestand met de inhoud van het onderzoeksvoorstel
%---------- Inleiding ---------------------------------------------------------

\section{Introductie} % The \section*{} command stops section numbering
\label{sec:introductie}

Datacompressie is overal, van vakantiefoto's op Instagram tot DNA compressie voor medisch onderzoek. Een wereld zonder datacompressie is ondenkbaar, er zouden enorme veelvouden van de huidige data opslag, bandbreedte en hardware capaciteit nodig moeten zijn om dezelfde data van vandaag te kunnen verwerken.

Bij DNA compressie is reeds een verkleining van meer dan 99 \% behaald in sommige use cases \autocite{Afify2011}. Bij afbeelding- en videocompressie kan een andere codec, die een visueel gelijkaardig resultaat geeft, een bestandsgrootte van factor tien hebben. Dit wilt zeggen dat compressie één van de belangrijkste factoren is, zeker vanuit het perspectief van de eindgebruiker, voor het optimaliseren van snelheid en kostprijs bij applicatieontwikkeling en meer.

Bij een kleine bevraging van een tiental studenten toegepaste informatica te HoGent, één digital content team, twee mobile app developers en drie web developers bleek echter dat niemand van hen intensief bezig was met het bepalen van welke codec ze zullen gebruiken voor de afbeeldingen en video’s binnen hun project. Vrijwel iedereen wist wel dat het belangrijk was afbeeldingen en video’s te uploaden tegen een lagere resolutie, maar de gebruikte codec verdedigen ging voor velen niet verder dan “het is voorgesteld door dit tooltje” of “bij JPEG heb je geen doorzichtige achtergrond”. 

Deze vaststelling was de doorslaggevende factor voor het opstellen van deze bachelorproef. Door de grote diversiteit binnen de doelgroep voor wie dit onderzoek nuttig is, zal extra belang gehecht worden aan het eenvoudig uitleggen van complexe materie.

Deze bachelorproef en de bijhorende onderzoeken zullen trachten een antwoord te geven op volgende vragen: 
\begin{itemize}
    \item{Hoe is datacompressie binnen IT ontstaan en wat waren enkele van de primitieve algoritmes?}
     \item{Waarom is er een groot verschil tussen de diverse afbeeldingcodecs en videocodecs?}
     \item{Wat is het verschil tussen JPEG en PNG?}
      \item{Wat is het verschil tussen H.264/AVC en H.264/SVC?}
      \item{Wat is DNA compressie en waarom is het de volgende uitdaging binnen datacompressie?}
\end{itemize}
Hierdoor zou de hoofdonderzoeksvraag moeten kunnen beantwoorden worden, zijnde:  
\begin{itemize}
    \item{Waarom moet stilgestaan worden bij het kiezen van een afbeelding- en/of videocodec?}
\end{itemize}
%---------- Stand van zaken ---------------------------------------------------

\section{Stand van zaken}
\label{sec:stand-van-zaken}

Datacompressie bestaat al veel langer dan computers. Zo hebben bij morsecode, ontstaan in 1838, veelgebruikte letters een kortere code. Ook bij computers bestaat datacompressie al enige tijd, zo zijn LZ77 en opvolgers afkomstig uit 1977 en later. \autocite{Riha2011} 

Dit wil ook zeggen dat er reeds een overweldigende hoeveelheid informatie te vinden is omtrent datacompressie en specifieke vormen van afbeelding- en videocompressie. Een heel goed boek over datacompressie is: 'Data compression, the complete reference' door  ~\textcite{Salomon2006}. Dit boek vereist, net zoals vele andere boeken omtrent datacompressie,  een grondige kennis van algoritmes en wiskunde om de volledige 1017 pagina's te begrijpen.

Overigens zijn er al enkele interessante thesissen geschreven omtrent afbeeldingscompressie (bijvoorbeeld over JPEG optimalisatie door ~\textcite{Wahlstrom2015} en videocompressie (bijvoorbeeld over de artefacten die H.264 compressie met zich meebrengt door  ~\textcite{Rakesh2013}). . Ook over het nog vrij recente topic, DNA compressie, zijn reeds tal van uitgebreide documenten beschikbaar. Enkele interessante artikels zijn die van ~\textcite{Afify2011} en ~\textcite{Kuruppu2012}. 

Bij deze documenten zijn echter enkele terugkerende problemen. Zelden of nooit wordt uitgelegd hoe de vergeleken bestanden verkregen zijn. Dit maakt het onmogelijk de experimenten te reproduceren of een soortgelijk onderzoek uit te voeren. Ook zijn de artikels vaak zeer complex (maar mathematisch correct) uitgelegd, wat het moeilijk maakt voor de doorsnee lezer alles te begrijpen. Of juist te simplistisch waardoor de verworven informatie niet volledig correct is. Ook ontbreekt er vaak een besluit om aan te tonen wat moet onthouden worden en waarom al dan niet moet gekozen worden voor een bepaald datacompressie algoritme. 

%---------- Methodologie ------------------------------------------------------
\section{Methodologie}
\label{sec:methodologie}

Deze bachelorproef zal bestaan uit zowel theoretische als praktische onderzoeken. De theoretische onderzoeken zullen bestaan uit enkele interviews met app en web development bedrijven alsook een uitgebreide literatuurstudie.

Het praktisch gedeelte zal bestaan uit het uitleggen van de onderliggende werking van JPEG en PNG en ook H.264/AVC en H.264/SVC. Er zal ook een vergelijkende studie gedaan worden door het comprimeren van dezelfde bestanden met de verschillende codecs. Een datacompressietool zal ook geschreven worden om de werking van primitieve compressietechnieken te verduidelijken.

%---------- Verwachte resultaten ----------------------------------------------
\section{Verwachte resultaten}
\label{sec:verwachte_resultaten}

De theoretische onderzoeken zullen een beeld geven van de kennis van datacompressie bij developers. Uit dit deel zal ook het ontstaan en de toekomst van datacompressie duidelijk worden. Dit gedeelte zal ook aantonen dat compressie meer is dan incrementele verbeteringen van oude technieken a.d.h.v. een bespreking van DNA compressie.

De praktische onderzoeken zullen inzicht proberen geven in de impact die het gebruik van een andere codec kan hebben op bestandsgrootte en kwaliteit. Dit zal gebeuren aan de hand van duidelijke grafieken waarop de bestandsgrootte in kb is af te lezen alsook de laadtijd in ms.

%---------- Verwachte conclusies ----------------------------------------------
\section{Verwachte conclusies}
\label{sec:verwachte_conclusies}

Het doel van deze bachelorproef is de lezer een beeld te geven hoe belangrijk het is stil te staan bij het kiezen van een gepaste codec voor de afbeeldingen en video's binnen een bepaald project. Dit omdat ook voor de eindgebruiker er een enorme tijdswinst en bandbreedte-/opslagbesparing tegenover kan staan. Ook de user experience is zeer belangrijk binnen deze bachelorproef: hoe bepaal je het middelpunt tussen kwaliteit en bestandsgrootte. 

Deze bachelorproef zal de lezer in staat stellen een geschikte keuze te maken tussen JPEG en PNG afbeeldingcompressie alsook H.264/AVC en H.264/SVC videocompressie. Het zal de lezer ook een universele kennis geven van datacompressie en de huidige uitdagingen om hem aan te zetten tot verder onderzoek naar het voordeel van andere compressietechnieken. 

%%---------- Andere bijlagen --------------------------------------------------
% TODO: Voeg hier eventuele andere bijlagen toe
%\input{...}

%%---------- Referentielijst --------------------------------------------------

\printbibliography[heading=bibintoc]

\end{document}
