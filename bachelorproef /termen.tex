\chapter{Belangrijke termen in datacompressie}
\label{ch:termen}

Hoewel deze paper op een zo eenvoudig mogelijke doch correcte en uitgebreide manier is geschreven, zijn er enkele termen die voor de lezer reeds gekend dienen te zijn. Enkele van de meest voorkomende termen binnen deze paper zien hier kort besproken.


% Om te updaten run : makeglossaries -pdf "bachelorproef_bontinck_lennert"
% Controleer .gls en verwijder   \setcounter{page}{df} indien bestaande
% latexmk -pdf "bachelorproef_bontinck_lennert"
% Run hier


\newglossaryentry{datacompressie}
{
	name={Datacompressie},
	description={De mens wilt vaak zoveel mogelijk output met zo weinig mogelijk input. Datacompressie vervult die wens binnen de computerwereld. Datacompressie draait er om zoveel mogelijk data weer te geven met zo weinig mogelijk bits}
}

\newglossaryentry{bit}
{
	name={Bit},
	description={Zoals de naam bit, kort voor binary digit, suggereert kan een bit beschouwd worden als een binair signaal. Een bit wordt beschouwd als de kleinste dataeenheid voor dataopslag}
}

% TODO: overige termen uit word


\glsaddall
\printglossary[title=Woordenlijst]
