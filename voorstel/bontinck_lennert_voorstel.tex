%==============================================================================
% Sjabloon onderzoeksvoorstel bachelorproef
%==============================================================================
% Gebaseerd op LaTeX-sjabloon ‘Stylish Article’ (zie voorstel.cls)
% Auteur: Jens Buysse, Bert Van Vreckem

\documentclass[fleqn,10pt]{voorstel}

%------------------------------------------------------------------------------
% Metadata over het voorstel
%------------------------------------------------------------------------------

\JournalInfo{HoGent Bedrijf en Organisatie}
\Archive{Bachelorproef 2018 - 2019}

%---------- Titel & auteur ----------------------------------------------------

\PaperTitle{Je kijkt er naar maar ziet het niet: afbeelding/ videocompressie en compressie in het algemeen}
\PaperType{Onderzoeksvoorstel Bachelorproef} % Type document

\Authors{Bontinck Lennert\textsuperscript{1}} % Authors
\CoPromotor{Tom Paridaens\textsuperscript{2} (Universiteit Gent - postdoctoraal onderzoeker)}
\affiliation{\textbf{Contact:}
  \textsuperscript{1} \href{mailto:lennert.bontinck.y9785@student.hogent.be }{lennert.bontinck.y9785@student.hogent.be };
  \textsuperscript{2} \href{mailto:tom.paridaens@ugent.be}{tom.paridaens@ugent.be};
}

%---------- Abstract ----------------------------------------------------------

% TODO: samenvatting:
\Abstract{Hier schrijf je de samenvatting van je voorstel, als een doorlopende tekst van één paragraaf. Wat hier zeker in moet vermeld worden: \textbf{Context} (Waarom is dit werk belangrijk?); \textbf{Nood} (Waarom moet dit onderzocht worden?); \textbf{Taak} (Wat ga je (ongeveer) doen?); \textbf{Object} (Wat staat in dit document geschreven?); \textbf{Resultaat} (Wat verwacht je van je onderzoek?); \textbf{Conclusie} (Wat verwacht je van van de conclusies?); \textbf{Perspectief} (Wat zegt de toekomst voor dit werk?).

Bij de sleutelwoorden geef je het onderzoeksdomein, samen met andere sleutelwoorden die je werk beschrijven.

Vergeet ook niet je co-promotor op te geven.
}

%---------- Onderzoeksdomein en sleutelwoorden --------------------------------
% TODO: Sleutelwoorden:
%
% Het eerste sleutelwoord beschrijft het onderzoeksdomein. Je kan kiezen uit
% deze lijst:
%
% - Mobiele applicatieontwikkeling
% - Webapplicatieontwikkeling
% - Applicatieontwikkeling (andere)
% - Systeembeheer
% - Netwerkbeheer
% - Mainframe
% - E-business
% - Databanken en big data
% - Machineleertechnieken en kunstmatige intelligentie
% - Andere (specifieer)
%
% De andere sleutelwoorden zijn vrij te kiezen

\Keywords{Onderzoeksdomein. Keyword1 --- Keyword2 --- Keyword3} % Keywords
\newcommand{\keywordname}{Sleutelwoorden} % Defines the keywords heading name

%---------- Titel, inhoud -----------------------------------------------------

\begin{document}

\flushbottom % Makes all text pages the same height
\maketitle % Print the title and abstract box
\tableofcontents % Print the contents section
\thispagestyle{empty} % Removes page numbering from the first page

%------------------------------------------------------------------------------
% Hoofdtekst
%------------------------------------------------------------------------------

% De hoofdtekst van het voorstel zit in een apart bestand, zodat het makkelijk
% kan opgenomen worden in de bijlagen van de bachelorproef zelf.
%---------- Inleiding ---------------------------------------------------------

\section{Introductie} % The \section*{} command stops section numbering
\label{sec:introductie}

Datacompressie is overal, van vakantiefoto's op Instagram tot DNA compressie voor medisch onderzoek. Een wereld zonder datacompressie is ondenkbaar, er zouden enorme veelvouden van de huidige data opslag, bandbreedte en hardware capaciteit nodig moeten zijn om dezelfde data van vandaag te kunnen verwerken.

Bij DNA compressie is reeds een verkleining van meer dan 99 \% behaald in sommige use cases \autocite{Afify2011}. Bij afbeelding- en videocompressie kan een andere codec, die een visueel gelijkaardig resultaat geeft, een bestandsgrootte van factor tien hebben. Dit wilt zeggen dat compressie één van de belangrijkste factoren is, zeker vanuit het perspectief van de eindgebruiker, voor het optimaliseren van snelheid en kostprijs bij applicatieontwikkeling en meer.

Bij een kleine bevraging van een tiental studenten toegepaste informatica te HoGent, één digital content team, twee mobile app developers en drie web developers bleek echter dat niemand van hen intensief bezig was met het bepalen van welke codec ze zullen gebruiken voor de afbeeldingen en video’s binnen hun project. Vrijwel iedereen wist wel dat het belangrijk was afbeeldingen en video’s te uploaden tegen een lagere resolutie, maar de gebruikte codec verdedigen ging voor velen niet verder dan “het is voorgesteld door dit tooltje” of “bij JPEG heb je geen doorzichtige achtergrond”. 

Deze vaststelling was de doorslaggevende factor voor het opstellen van deze bachelorproef. Door de grote diversiteit binnen de doelgroep voor wie dit onderzoek nuttig is, zal extra belang gehecht worden aan het eenvoudig uitleggen van complexe materie.

Deze bachelorproef en de bijhorende onderzoeken zullen trachten een antwoord te geven op volgende vragen: 
\begin{itemize}
    \item{Hoe is datacompressie binnen IT ontstaan en wat waren enkele van de primitieve algoritmes?}
     \item{Waarom is er een groot verschil tussen de diverse afbeeldingcodecs en videocodecs?}
     \item{Wat is het verschil tussen JPEG en PNG?}
      \item{Wat is het verschil tussen H.264/AVC en H.264/SVC?}
      \item{Wat is DNA compressie en waarom is het de volgende uitdaging binnen datacompressie?}
\end{itemize}
Hierdoor zou de hoofdonderzoeksvraag moeten kunnen beantwoorden worden, zijnde:  
\begin{itemize}
    \item{Waarom moet stilgestaan worden bij het kiezen van een afbeelding- en/of videocodec?}
\end{itemize}
%---------- Stand van zaken ---------------------------------------------------

\section{Stand van zaken}
\label{sec:stand-van-zaken}

Datacompressie bestaat al veel langer dan computers. Zo hebben bij morsecode, ontstaan in 1838, veelgebruikte letters een kortere code. Ook bij computers bestaat datacompressie al enige tijd, zo zijn LZ77 en opvolgers afkomstig uit 1977 en later. \autocite{Riha2011} 

Dit wil ook zeggen dat er reeds een overweldigende hoeveelheid informatie te vinden is omtrent datacompressie en specifieke vormen van afbeelding- en videocompressie. Een heel goed boek over datacompressie is: 'Data compression, the complete reference' door  ~\textcite{Salomon2006}. Dit boek vereist, net zoals vele andere boeken omtrent datacompressie,  een grondige kennis van algoritmes en wiskunde om de volledige 1017 pagina's te begrijpen.

Overigens zijn er al enkele interessante thesissen geschreven omtrent afbeeldingscompressie (bijvoorbeeld over JPEG optimalisatie door ~\textcite{Wahlstrom2015} en videocompressie (bijvoorbeeld over de artefacten die H.264 compressie met zich meebrengt door  ~\textcite{Rakesh2013}). . Ook over het nog vrij recente topic, DNA compressie, zijn reeds tal van uitgebreide documenten beschikbaar. Enkele interessante artikels zijn die van ~\textcite{Afify2011} en ~\textcite{Kuruppu2012}. 

Bij deze documenten zijn echter enkele terugkerende problemen. Zelden of nooit wordt uitgelegd hoe de vergeleken bestanden verkregen zijn. Dit maakt het onmogelijk de experimenten te reproduceren of een soortgelijk onderzoek uit te voeren. Ook zijn de artikels vaak zeer complex (maar mathematisch correct) uitgelegd, wat het moeilijk maakt voor de doorsnee lezer alles te begrijpen. Of juist te simplistisch waardoor de verworven informatie niet volledig correct is. Ook ontbreekt er vaak een besluit om aan te tonen wat moet onthouden worden en waarom al dan niet moet gekozen worden voor een bepaald datacompressie algoritme. 

%---------- Methodologie ------------------------------------------------------
\section{Methodologie}
\label{sec:methodologie}

Deze bachelorproef zal bestaan uit zowel theoretische als praktische onderzoeken. De theoretische onderzoeken zullen bestaan uit enkele interviews met app en web development bedrijven alsook een uitgebreide literatuurstudie.

Het praktisch gedeelte zal bestaan uit het uitleggen van de onderliggende werking van JPEG en PNG en ook H.264/AVC en H.264/SVC. Er zal ook een vergelijkende studie gedaan worden door het comprimeren van dezelfde bestanden met de verschillende codecs. Een datacompressietool zal ook geschreven worden om de werking van primitieve compressietechnieken te verduidelijken.

%---------- Verwachte resultaten ----------------------------------------------
\section{Verwachte resultaten}
\label{sec:verwachte_resultaten}

De theoretische onderzoeken zullen een beeld geven van de kennis van datacompressie bij developers. Uit dit deel zal ook het ontstaan en de toekomst van datacompressie duidelijk worden. Dit gedeelte zal ook aantonen dat compressie meer is dan incrementele verbeteringen van oude technieken a.d.h.v. een bespreking van DNA compressie.

De praktische onderzoeken zullen inzicht proberen geven in de impact die het gebruik van een andere codec kan hebben op bestandsgrootte en kwaliteit. Dit zal gebeuren aan de hand van duidelijke grafieken waarop de bestandsgrootte in kb is af te lezen alsook de laadtijd in ms.

%---------- Verwachte conclusies ----------------------------------------------
\section{Verwachte conclusies}
\label{sec:verwachte_conclusies}

Het doel van deze bachelorproef is de lezer een beeld te geven hoe belangrijk het is stil te staan bij het kiezen van een gepaste codec voor de afbeeldingen en video's binnen een bepaald project. Dit omdat ook voor de eindgebruiker er een enorme tijdswinst en bandbreedte-/opslagbesparing tegenover kan staan. Ook de user experience is zeer belangrijk binnen deze bachelorproef: hoe bepaal je het middelpunt tussen kwaliteit en bestandsgrootte. 

Deze bachelorproef zal de lezer in staat stellen een geschikte keuze te maken tussen JPEG en PNG afbeeldingcompressie alsook H.264/AVC en H.264/SVC videocompressie. Het zal de lezer ook een universele kennis geven van datacompressie en de huidige uitdagingen om hem aan te zetten tot verder onderzoek naar het voordeel van andere compressietechnieken. 

%------------------------------------------------------------------------------
% Referentielijst
%------------------------------------------------------------------------------
% de gerefereerde werken moeten in BibTeX-bestand ``voorstel.bib''
% voorkomen. Gebruik JabRef om je bibliografie bij te houden en vergeet niet
% om compatibiliteit met Biber/BibLaTeX aan te zetten (File > Switch to
% BibLaTeX mode)

\phantomsection
\printbibliography[heading=bibintoc]

\end{document}
