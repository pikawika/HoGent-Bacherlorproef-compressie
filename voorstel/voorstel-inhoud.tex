%---------- Inleiding ---------------------------------------------------------

\section{Introductie} % The \section*{} command stops section numbering
\label{sec:introductie}

Compressie is overal, bij katten video’s op YouTube, vakantiefoto’s op Instagram, zelfs bij het digitaal raadplegen van iemands DNA. Een wereld zonder compressie is ondenkbaar, er zouden enorme veelvouden van de huidige data opslag, brandbreedte en hardware capaciteit nodig moeten zijn om dezelfde data van vandaag te verwerken.

Bij DNA compressie is een verkleining van bestandsgrootte van meer dan 99 \% niet ongewoonlijk \autocite{Afify2011}. Bij afbeelding- en videocompressie kan een andere codec die een visueel gelijkaardig resultaat geeft een bestandsgrootte van factor tien hebben. Dit wilt zeggen dat compressie één van de belangrijkste factoren is, zeker vanuit het perspectief van de eindgebruiker, voor het optimaliseren van snelheid en kost bij applicatieontwikkeling en meer.

Bij een kleine bevraging van een tiental toegepaste informatica studenten te HoGent, één digital content team, twee mobile app developers en drie web developers bleek echter dat geen enkel van deze intensief bezig was met het bepalen van welke codec ze zullen gebruiken voor de afbeeldingen en video’s binnen hen project. Vrijwel iedereen wist wel dat het belangrijk was afbeeldingen en video’s te uploaden tegen een lagere resolutie maar de gebruikte codec verdedigen ging voor velen niet verder dan “het is voorgesteld door dit tooltje” of “bij JPEG heb je geen doorzichtige achtergrond”. 

Deze vaststelling was de triggerende factor voor deze bachelorproef. Een onderzoek naar waarom standaarden als JPEG en PNG nog niet vervangen zijn, welke interessanter is voor welk gebruik en hoeveel tijd en geld bespaard kan worden door minimale inspanning van de juiste codec keuze. 

Concreet zal het ontstaan van compressie en de fundamentele wiskunde achter de primitieve vormen van compressie besproken worden. De werking van JPEG en PNG compressie toegelicht worden om zo te kunnen bepalen welke beter is voor welke doeleinden. Videocompressie uitgelegd worden a.d.h.v. een vergelijkende studie tussen H.264/AVC en H.264/SVC. Ook zal er een blik geworpen worden op huidige en toekomstige uitdagingen voor compressie zoals DNA-compressie.

%---------- Stand van zaken ---------------------------------------------------

\section{Stand van zaken}
\label{sec:stand-van-zaken}




Hier beschrijf je de \emph{state-of-the-art} rondom je gekozen onderzoeksdomein. Dit kan bijvoorbeeld een literatuurstudie zijn. Je mag de titel van deze sectie ook aanpassen (literatuurstudie, stand van zaken, enz.). Zijn er al gelijkaardige onderzoeken gevoerd? Wat concluderen ze? Wat is het verschil met jouw onderzoek? Wat is de relevantie met jouw onderzoek?

Verwijs bij elke introductie van een term of bewering over het domein naar de vakliteratuur, bijvoorbeeld~\textcite{Salomon2006}! Denk zeker goed na welke werken je refereert en waarom.

% Voor literatuurverwijzingen zijn er twee belangrijke commando's:
% \autocite{KEY} => (Auteur, jaartal) Gebruik dit als de naam van de auteur
%   geen onderdeel is van de zin.
% \textcite{KEY} => Auteur (jaartal)  Gebruik dit als de auteursnaam wel een
%   functie heeft in de zin (bv. ``Uit onderzoek door Doll & Hill (1954) bleek
%   ...'')

Je mag gerust gebruik maken van subsecties in dit onderdeel.

%---------- Methodologie ------------------------------------------------------
\section{Methodologie}
\label{sec:methodologie}

Hier beschrijf je hoe je van plan bent het onderzoek te voeren. Welke onderzoekstechniek ga je toepassen om elk van je onderzoeksvragen te beantwoorden? Gebruik je hiervoor experimenten, vragenlijsten, simulaties? Je beschrijft ook al welke tools je denkt hiervoor te gebruiken of te ontwikkelen.

%---------- Verwachte resultaten ----------------------------------------------
\section{Verwachte resultaten}
\label{sec:verwachte_resultaten}

Hier beschrijf je welke resultaten je verwacht. Als je metingen en simulaties uitvoert, kan je hier al mock-ups maken van de grafieken samen met de verwachte conclusies. Benoem zeker al je assen en de stukken van de grafiek die je gaat gebruiken. Dit zorgt ervoor dat je concreet weet hoe je je data gaat moeten structureren.

%---------- Verwachte conclusies ----------------------------------------------
\section{Verwachte conclusies}
\label{sec:verwachte_conclusies}

Hier beschrijf je wat je verwacht uit je onderzoek, met de motivatie waarom. Het is \textbf{niet} erg indien uit je onderzoek andere resultaten en conclusies vloeien dan dat je hier beschrijft: het is dan juist interessant om te onderzoeken waarom jouw hypothesen niet overeenkomen met de resultaten.

